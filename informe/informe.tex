\documentclass[12pt]{extarticle}
\usepackage[utf8]{inputenc}
\usepackage{caption}
\usepackage{subcaption}
\usepackage[spanish]{babel}
\usepackage{multicol}
\usepackage{fancyhdr}
\usepackage{longtable}
\usepackage{enumitem}

\usepackage{url}
\usepackage[a4paper]{geometry}
\usepackage{float}
\usepackage{setspace}
\usepackage{color}   %May be necessary if you want to color links
\usepackage{hyperref}
\hypersetup{
    colorlinks=true, %set true if you want colored links
    linktoc=all,     %set to all if you want both sections and subsections linked
    linkcolor=blue,  %choose some color if you want links to stand out
}
\usepackage{graphicx}
\graphicspath{ {images/} }


\begin{document}

    \title{Ingeniería de Software I - T - Análisis y Diseño de Sistemas\\
    ISFPP\\
    \large{Cátedera:\\
    Profesor: Lic. Marta Saenz López\\
    JTP: Lic. Sebastián Schanz\\
    Ayudante de Segunda: Guillermo Urrutia}}
    \author{KRMPOTIC, Lucas\\
    MURILLO, Alexis\\
    SERRUYA ALOISI, Luciano Sebastián\\
    SOTO, Kevin\\
    TOLEDO MARGALEF, Pablo Adrián}
    \date{}
    \maketitle
    \pagebreak

    \pagestyle{fancy}
    \fancyhf{}
    \lhead{ISFPP}
    \chead{IS I - T - AyDS}
    \rhead{\includegraphics[scale=0.15]{images/logoUnpsjb.png}}
    \lfoot{\thepage}
    \cfoot []{Krmpotic - Murillo - Serruya Aloisi - Soto - Toledo Margalef} 

    %indice
    \tableofcontents


    \clearpage


    %COMIENZO INTERLINEADO
    \begin{spacing}{1.2}


        \newcommand\OT{\textit{Orden de Trabajo}}
        \newcommand\OTs{\textit{Órdenes de Trabajo}}


    \section{Entrevista}
    \begin{enumerate}
        \item ¿Qué actividades realiza la empresa?
        \item ¿Existen distintas jerarquías dentro la organización? ¿Manejan todos la misma información?
        \item ¿Cómo se realizan las actividades? ¿Cuáles son los estados que se atraviesan?
        \item Dentro de esas actividades, ¿Qué información relevante a cada una requiere ser almacenada?
        \item ¿Cómo se recepcionan los pedidos de servicios de los clientes?
        \item ¿Hay alguna forma de priorización de los trabajos?
        \item ¿Cómo se realiza la asignacion de trabajo a cada empleado? 
        \item ¿Los empleados se especializan en alguna cuestión particular? ¿Hay empleados especializados en reparaciones y otros en otros servicios? 
        \item ¿Existe uso de información producida por una actividad para la realización de otra? Siendo así, qué actividades, hacia cuáles y qué información.
        \item Con respecto a clientes ¿Se da una diferencia en el registro del mismo cuando se trata de un cliente particular o una empresa?
        \item ¿Cómo es el flujo de información entre la organización y un cliente afectado a un servicio?
        \item ¿De qué manera se realiza el seguimiento o registro de lo realizado en un servicio?
        \item ¿Cómo se realiza la cotización de un servicio realizado?
        \item ¿Cómo funciona el cobro por ventas o servicios y cómo se registra éste?
        \item ¿Existen diferencias con respecto al cobro cuando se trata de un cliente particular o una empresa?
        \item ¿Existe alguna forma de seguimiento de aquellas actividad que requieren de más de un día (más de lo habitual) para su conclusión?
        \item ¿Con qué cantidad de proveedores se trabaja? ¿Cómo se realizan las compras? ¿Trabaja con distintos tipos de proveedores?
        \item ¿Brindan servicio técnico en calidad de agente oficial? ¿Se lleva registro? ¿Cambia con respecto al servicio técnico habitual?
        \item ¿Cómo se registra el pago a proveedores, el encargo de mercadería, y su recepción?
        \item ¿Se mantiene el mismo stock para ventas que para reparaciones?
        \item ¿Se lleva registro de cada equipo en particular? ¿Cómo se tratan los equipos reincidentes?
        \item ¿En qué momento se da por finalizado un servicio?
        \item ¿Cómo se maneja el vencimiento de las órdenes y la garantía que las órdenes tienen luego de cerradas?
    \end{enumerate}

    \pagebreak









    \section{Relevamiento de la Organización}
    A.T. Informática es una empresa residente en la ciudad de Puerto Madryn y se dedica principalmente al soporte y mantenimiento tanto de software como de hardware.

    Brindan servicios de instalación de cámaras y redes, asistencia on-site (soporte a domicilio o en el lugar), asistencia por teléfono, asistencia remota, reparación de PC tanto domésticas como para empresas, mantenimiento a impresoras láser, de matriz de punto, fiscales, y plotters; también realizan reparaciones en tablets.  Los objetivos de la empresa son los de brindar un servicio de mantenimiento de calidad a la comunidad y además maximizar ganancias.\\

    \pagebreak
    \subsection{Roles dentro de la Organización}
    Actualmente dentro de la organización existen los siguientes roles:
    \begin{itemize}
        \item Técnico Fiscal: Realiza las reparaciones de impresoras fiscales. Dichos trabajos provienen de los clientes corporativos (ver sección 2.2).
        \item Técnico de Taller: Encargado de realizar las reparaciones dentro del taller (ver sección 2.5.1). 
        \item Técnico On-site: Encargado de realizar los trabajos en los domicilios de los clientes, ya sean reparaciones o instalaciones (ver sección 2.5.2).
        \item Jefe de Taller: Encargado de realizar la asignación de trabajo a cada técnico. Decide cuándo abrir una \OT{} (ver sección 2.5), a qué técnico asignarla y cuándo cerrarla.
        \item Operario Contable: Encargado de realizar la facturación de una \OT{} cerrada (ver sección 2.6), también realiza pedidos de productos a proveedores (ver sección 2.4).
        \item Cajero: Atiende a los clientes que deseen realizar compras en el local (ver sección 2.5.4).
        \item Gerente: Máximo cargo en la organización. Encargado de las decisiones estratégicas. Decide qué clientes reciben servicio y cuáles no, a cuáles proveedores comprarle mercadería, toma todas las decisiones en relación a los recursos humanos, y cualquier otra resolución estratégica.  
    \end{itemize}
    

    Debido al tamaño de la organización, un empleado puede desempeñar varios roles a la vez, como por ejemplo técnico de taller y técnico on-site.  

    En cuanto a la información manejada por cada miembro de la organización, hay una diferencia entre la que manejan los técnicos y el operario contable. Mientras que los técnicos manejan los datos pertinentes al trabajo a realizar, tales como el problema a solucionar o el domicilio a dónde se debe realizar una instalación, el operario contable debe conocer los valores a facturar por cada actividad precisada en una \OT{} y algún medio de comunicación para transmitirle el presupuesto (ver sección 2.5.1) al cliente. El Jefe de Taller también se puede encargar de notificar sobre cotizaciones de servicios a los clientes que los han requerido.

    %Cuando un empleado deja de trabajar para la organización, el Gerente lo da de baja en el sistema de modo que ya no pueda figurar como técnico asociado a una \OT{}. No obstante no se pierde registro de sus actividades en \OTs{} anteriores. 

    A cada empleado se le asigna el trabajo dependiendo de varias cosas:
    \begin{itemize}
        \item El nivel de experiencia en la reparación de equipos.
        \item El rol que cumpla (a un empleado de taller no se le va a encargar que realice una instalación).
        \item El trabajo pendiente que tenga en el momento y la demora que tenga para atender el trabajo nuevo que se le pueda asignar. 
    \end{itemize}

    Hay empleados especializados, en general, por tener más experiencia en ciertos tipos de equipos (por ejemplo: impresoras láser) o en ciertos tipos de instalaciones (por ejemplo: redes inalámbricas); por lo tanto, se les suele asignar esos tipos de trabajo específicos.

    \pagebreak
    \subsection{Clientes}

    Entre los distintos clientes que atiende la organización se puede realizar la siguiente clasificación:
    \begin{itemize}
        \item Clientes particulares: personas que requieren algún servicio o reparación por cuenta propia.
        \item Clientes comerciales: aquellos que son parte de un negocio u organización que no excede las 5 ó 10 computadoras.
        \item Clientes empresariales: superan las 10 computadoras y generan un volumen de trabajo mayor a los clientes comerciales. Suelen contar con distintos sectores, complejizando la comunicación con la organización.
        \item Clientes corporativos: son clientes que tercerizan trabajos para la organización. Estos trabajos incluyen la atención a servidores, reparación de impresoras fiscales, instalación y mantenimiento de redes. Dichos trabajos tercerizados pueden ser realizados a clientes particulares, comerciales o empresariales.
    \end{itemize}

    Al momento de crear un nuevo Cliente el Jefe de Taller registra los siguientes datos:
    \begin{itemize}
        \item Nombre o Razón Social
        \item DNI/CUIT/CUIL
        \item Dirección de facturación
        \item Teléfono
        \item Correo electrónico
        \item Contacto/s
        	\begin{itemize}
				\item Nombre	
		        \item Teléfono/s
		        \begin{itemize}
			        \item Interno	        
		        \end{itemize}
		        \item Correo eléctronico (más de uno en clientes comerciales o empresariales)\\        	
        	\end{itemize}
    \end{itemize}

        NOTA: Si bien es posible registrar contactos en clientes particulares, en el caso de los clientes comerciales o empresariales es común que exista más de un interlocutor. Esto tiene que ver con que distintos sectores de la organización cliente pueden requerir servicios. En esos casos se registra un nombre, un teléfono o interno, y una dirección de correo electrónico del tal interlocutor. \\

    Los datos de un cliente pueden ser modificados solamente por el Jefe de Taller cuando el cliente avisa a la organización que ha cambiado, por ejemplo, de teléfono de contacto o de dirección de correo electrónico.

    Si por alguna razón se debiera eliminar un cliente, el Jefe de Taller o el Gerente pueden darlo de baja en el sistema, y no se podrán realizar trabajos a su nombre. La baja de un cliente se podrá realizar sólo para aquellos clientes que no registren deudas ni notas de crédito a su favor.
    
    Generalmente los clientes se comunican con la organización en caso de requerir un servicio o para chequear el estado de algún trabajo. Si el cliente tiene un equipo en taller puede llamar para verificar si tiene arreglo o no; suele darse que los clientes se comuniquen cuando las reparaciones tardan más de lo habitual. Dicha consulta por parte del cliente se deja asentada en la orden de trabajo asociada al requerimiento de ese servicio. (Ver Sección 2.5.1)\\
    \subsubsection{Cuenta Corriente}
        Los clientes poseen cuenta corriente dentro de la organización donde se detallan las facturas y notas de crédito emitidas a su nombre y los pagos que realizó.\\
        Para ello se registra:
        \begin{itemize}
            \item Fecha del movimiento
            \item Concepto
            \item Si se debitó o se acreditó
            \item Monto
            \item Saldo actual del cliente
        \end{itemize}

        La cuenta corriente de un cliente sólo puede ser vista por personal contable de la organización. Su estado (saldo a favor o en contra) es tenido en cuenta al momento de decidir si prestarle o no servicio. Si el gerente considera que el cliente adeuda un saldo significativo puede decidir negarle el servicio.

    \pagebreak
    \subsection{Productos}
        Para las reparaciones, mantenimiento de equipos e instalaciones, la organización dispone de un stock de productos (insumos) necesarios para el desarrollo de esas actividades (ver seccion 2.5).
    
        Los productos se clasifican en \textit{partes} (engranajes de impresoras, mecanismos de tiqueadoras, memorias fiscales, cables de datos, cables de alimentación, etc.) y \textit{componentes} (discos duros, memorias, placas madres, lecto-grabadoras de CD-DVD, teclados, mouses, etc.).\\

    La organización cuenta con una \textit{lista de productos} en la que, de cada artículo, se detalla: 
    \begin{itemize}
        \item Nombre
        \item Descripción
    	\item Stock
    	\item Stock mínimo
        \item Marca
        \item Código interno de referencia \footnote{Número generado automáticamente por el sistema.}
        \item Distintos precios según el margen de ganancia deseado (por reparación de taller, por venta directa, por utilización en visita)
        \item Proveedor/es (teniendo como primera opción al más usual). 
		\begin{itemize}
	        \item Código propio del proveedor \footnote{Algunos proveedores requieren que los pedidos que se les hacen contengan el código del producto, es decir el código con el cual el proveedor serializa ese producto en \textbf{su} catálogo.}
    	    \item Si tiene o no garantía, y en caso de que la tenga de cuánto tiempo.
    	    \item Precio (en general el último precio pagado por el producto).
		\end{itemize}
    \end{itemize}

    La información propia de un producto puede ser vista por cualquier integrante de la organización, pero sólo puede ser modificada por el Gerente. También es el único que puede agregar o eliminar un producto si se considera necesario, caso este último en el cual ya no podrá formar parte de un pedido a proveedor (ver sección 2.4), de un presupuesto, ni de detalles de trabajos a realizar.
    
    Una gran mayoría de los artículos tienen sus precios dolarizados, por lo tanto el Gerente mantiene la lista de precios actualizada en base a la valuación del dólar. Con respecto a los productos con precios no dolarizados, el Gerente realiza una actualización mensual de sus precios.

    \subsection{Proveedores}
    Los productos que maneja la organización son provistos por diferentes empresas. El stock se actualiza cuando es necesario (on-demand). Sin embargo, se intenta mantener una cantidad básica de cada uno de los elementos.\\

    Al momento de agregar un proveedor el Gerente lo registra con la siguiente información.
        \begin{itemize}
            \item Razón Social
            \item CUIT
            \item Dirección
            \item Localidad
            \item Teléfono (puede ser más de uno)
            \item E-mail
            \item Observaciones varias como el horario de atención o interno de contacto con distintos integrantes de la organización proveedora
        \end{itemize}

    En cualquier momento el Gerente puede modificar los datos del proveedor para reflejar algún cambio de teléfono, e-mail, dirección que se produzca.

    De la misma forma, si se considera necesario, el Gerente puede dar de baja un proveedor. 
    
    Si algún empleado considera que no hay suficiente stock de algún producto, se lo comunica al Gerente, que es quien dará la orden (o no) al Operario Contable para que realice un pedido. Todos los pedidos son realizados por el Operario Contable a través de correo electrónico, o bien en sitios de venta online. Asimismo, debe registrar a qué proveedor se solicitó qué artículos, a qué precio \footnote{Establecido por el proveedor} y en qué cantidad, fecha en la que se hizo y fecha probable de llegada. Este registro, al igual que las recepciones y los pagos, actualmente lo mantiene actualizado el Operario Contable en una hoja de cálculos.

    Si un producto recibido ya forma parte de la lista de productos, se actualiza el stock. Recibido un producto por primera vez, el Gerente lo agrega a la lista de productos con la información pertinente y la cantidad que ingresó. 
        
    Cuando el Operario Contable realiza un pago a un proveedor, lo registra con fecha, monto, medio de pago y, en caso de ser por transferencia o pago eléctrónico, el número de transacción y el banco. Si se paga con cheque, se deja asentado el número de cheque y el banco que emite el cheque.\\

    %Los clientes coporativos proveen los repuestos especializados para sus trabajos encomendados. Éstos forman parte del stock, pero sólo se los utiliza en los trabajos de dichos clientes. Se los registra de igual manera que una parte o componente habitual, pero se especifica de qué cliente proviene; no se los puede utilizar en pedidos a proveedores convencionales ni ser usados en trabajos que no sean del cliente corporativo del que provienen.\\

    La mayoría de los proveedores envían el remito y la factura junto con el pedido. El Gerente de la organización especificó que el remito y la factura siempre coinciden en sus detalles y también con lo enviado realmente por el proveedor. 
    
    La recepción de un pedido la realiza el Operario Contable, verificando que lo que llegó (detalle de remito) coincida con lo registrado en la planilla (ver Anexo 1, Figura 5). De ser así, el pedido se marca como ``satisfecho'' y se procede a actualizar la lista de productos acorde a lo recibido.

    Cuando la cantidad de productos que llegaron es menor a la cantidad solicitada (el proveedor facturó menos productos que los pedidos, y en el remito figuran menos productos que los solicitados), el Operario Contable actualiza el stock con los artículos disponibles y se comunica con el proveedor para avisarle de la situación. Si el proveedor puede completar el pedido, el Operario Contable marca el pedido como ``pendiente'' detallando qué productos faltaron. El proveedor se encarga de enviar los productos que hagan falta, y una vez que llegan, el Operario Contable marca el pedido como ``satisfecho''.  

    %TODO falta hablar sobre productos que lleguen rotos 

    Si el proveedor no tiene los productos faltantes en existencia y no puede completar el pedido, el Operario Contable lo marca como ``satisfecho incompleto'' con el detalle de los productos faltantes. El Operario Contable  mantiene un listado de estos productos pendientes para que al momento de realizar nuevos pedidos a otros proveedores se los tenga en cuenta. Cuando llegan los productos faltantes, el Operario Contable los remueve de la lista de pendientes.

%    En el caso de que lleguen productos de más, el Operario Contable notifica al Gerente y decide si abonar los artículos de más al proveedor y la organización se los queda, o si se los rechaza.\\

    Si algún integrante de la organización encuentra un producto roto o en mal estado, se lo debe notificar al Gerente para que éste actualice el stock de forma manual.

    
    %ACA FALTA HABLAR DE LA DOCUMENTACION - ESCANEAR REMITO PROVEEDOR Y HOJA DE CALCULO 
    
    \pagebreak
    %\subsection{Órdenes de Trabajo}
    %La información de todas las actividades que realiza la empresa se registra bajo un mismo tipo de Orden de Trabajo (de ahora en más OT).
    %e almacena:

    %ACA FALTA HABLAR DE LA DOCUMENTACION - ESCANEAR REMITO PROVEEDOR Y HOJA DE CALCULO 
    
    \pagebreak
    %\subsection{Órdenes de Trabajo}
    %La información de todas las actividades que realiza la empresa se registra bajo un mismo tipo de Orden de Trabajo (de ahora en más OT).
    %e almacena:
    %begin{itemize}
    %   \item Fecha de rececepción
    %   \item Cliente
    %   \item Pedido del cliente. Sea una falla o lo requerido en una instalación.
    %   \item Tipo de Equipo. Marca, modelo y número de serie (en caso de ser una reparación).
    %   \item Técnico que recibió. 
    %   \item Técnico asignado para el trabajo.
    %   \item Vencimiento, manualmente asignado.
    %end{itemize}

    %En la orden de trabajo, además de toda la información requerida al momento de crearla, se registra con fecha y nombre de técnico las Tipo tarea que se han ido realizando.
    %Por ejemplo:\\ 
    %\textit{
    %12-03 (Pablo) El equipo no encendía. Se lo abrió y se encontró que la memoria estaba quemada. Se colocó una nueva de 4Gb. Código: 1234.
    %\\

    %El seguimiento se registra en el campo de observaciones de la orden de trabajo, indicando los avances, problemas y comentarios referentes a lo realizado.\\

    %Si el técnico asociado a la OT considera que el trabajo necesita de un repuesto, se lo detalla en la orden. Dicho repuesto es reservado del stock (la cantidad del artículo disminuye virtualmente). Los artículos reservados todavía no dejan de existir en el stock, pero no se los puede considerar para otros trabajos. 

    %Llegado el caso de que un repuesto reservado del stock no se utilice en el trabajo, se detalla en la orden, se cancela la reserva del stock, y no se factura.

    %El Gerente o el Jefe de Taller puede i/orar dicha reserva y reordenar las asignaciones de repuestos a los presupuestos manualmente, haciendo que ciertas OT se retrasen o aceleren su paso.

    %Si no hubiera la cantidad necesaria para satisfacer la demanda del trabajo se puede cambiar la fecha de vencimiento de la orden, contemplando el tiempo que tardaría conseguir el repuesto. Esta demora en el cumplimento del trabajo se le comunica al cliente, que puede aceptar o no que se continue. Si no acepta se cancela la reparación o la visita.

    %Toda OT es creada y cerrada manualmente.\\

    %El vencimiento incluido en una OT se utiliza como \textbf{sugerencia de prioridad}. Se tiene en cuenta la fecha de vencimiento, a qué cliente pertenece y qué equipo es al momento de realizar los trabajos. 

    %Un Tipo servicio puede darse por finalizado por dos razones:
    %\begin{itemize}
     %  \item Porque se ha completado el trabajo y se cierra la orden.
      % \item Porque no se puede completar, lo cual constituye una orden fallida.
   % \end{itemize}


    \pagebreak


    \subsection{Servicios Ofrecidos}
    La organización cuenta con un \textit{tarifario}, en el cual se establecen los precios de las distintas tareas requeridas para realizar los diferentes servicios brindados para los tipos de equipo con los que trabaja la empresa (están tipificados, pudiendo ser PC, notebook, Tablet, impresora láser, etc.) y, además un precio de \textit{Revisión, Diagnóstico, y Presupuesto} (de ahora en más \textit{RDyP}) por cada uno. También se especifica para cada tipo de equipo las tareas que se pueden realizar y sus precios.

    \subsubsection{Reparación de Equipos}
    El proceso de reparación comienza cuando se acerca un cliente a la oficina de la empresa con algún equipo a reparar. El técnico que atiende al cliente verifica si ya está registrado; si no es así, se le toman los datos y se lo registra. Si el cliente ya estaba registrado y es moroso se debe considerar prestar o no el servicio \footnote{Decisión tomada por el Gerente}.
    
   El técnico abre una \OT{} y se asigna a sí mismo como el que atendió la solicitud y como encargado de la \OT{} creada (de cualquier manera, el Jefe de Taller puede cambiar el encargado mientras no se comience a trabajar en la misma). La \OT{} se encuentra en estado \textit{Asignada} \footnote{Para ver los distintos estados de una \OT{}, ver Anexo 2}. 

    El técnico toma el equipo y verifica si se encuentra registrado. Un cliente puede tener asociado varios equipos, contando cada uno con su historial de reparaciones. Si el equipo no se encuentra registrado, lo da de alta indicando tipo de equipo, marca y modelo, y le asigna un número de serie único (generado automáticamente). El técnico asienta, entonces, los datos del equipo en la \OT{}.

    Los datos pertenecientes a un equipo pueden ser modificados por cualquier empleado en cualquier momento. No obstante, sólo el Jefe de Taller puede dar de baja un equipo, de modo que ya no puedan abrirse \OTs{} para éste.
   
    En la \OT{} también va a registrar:
    \begin{itemize}
        \item La fecha de solicitud de servicio
        \item El cliente solicitante
        \item Una descripción que dio el cliente del problema
        \item Vencimiento (plazo tentativo de finalización del trabajo; generalmente dos semanas a partir de la fecha de creación de la \OT{})
    \end{itemize}

    El cliente abona en efectivo y por adelantado el importe de la tarifa de \textit{RDyP} según el tipo de equipo que se trate. El técnico que lo esté atendiendo le entrega el comprobante de la \OT{} (ver Anexo 1, Figura 2) y un recibo por el pago de la \textit{RDyP} (ver Anexo 1, Figura 3). 


    La \OT{} queda en espera de ser revisada por su técnico encargado para luego ser cotizada (si es que se puede reparar).
    Cuando un técnico revisa el equipo de una \OT{}, indica las tareas que cree necesarias para realizar la reparación y también los repuestos que se requerirán para completar cada tarea, en caso de ser necesario. En este momento, la \OT{} se encuentra en estado \textit{Revisada}. Las tareas también tienen estados, y al anotarse en la \OT{} inicialmente, se encuentran en estado \textit{No empezada}.
    
    Los repuestos indicados por el técnico son reservados del stock (la cantidad del artículo disminuye virtualmente). Los artículos reservados todavía no dejan de existir en el stock, pero no se los puede considerar para otros trabajos.     
    
    Realizada esta revisión, el Técnico engargado confecciona un presupuesto sumando los valores de las tareas a realizar (según lo especificado en el tarifario) y los valores de los posibles repuestos a utilizar. En caso de que se esté trabajando con un tipo de equipo que no esté presente en el tarifario, el Jefe de Taller decide el precio a cotizar por el equipo. La \OT{} se pasa a estado \textit{Cotizada}.
    
    El técnico se contacta con el cliente para informar el diagnóstico del equipo, si va a ser posible una reparación, y en caso de que lo sea, el presupuesto del trabajo. Una vez comunicado el presupuesto al cliente, la \OT{} se pasa a estado \textit{Notificada de cotización}. En caso de que el equipo no se pueda reparar, el técnico le informa al cliente que no será posible llevar a cabo la reparación y queda a disposición del cliente retirar el equipo del local (la \OT{} pasa a estado \textit{No se repara}).
 %TODO Definir bien la diferencia en nombre de estados para cuando no se repara porque no puedo, porque el cliente no quiere y porque me di cuenta a mitad de camino que no va 
    Si el cliente acepta el presupuesto, la \OT{} queda en espera de comienzo de trabajo; se cambia a estado \textit{Presupuesto aceptado}. En caso contrario se le notifica al cliente que debe pasar a retirar el equipo del local (se pasa al estado \textit{No se repara}), y la \OT{} se cierra, imposibilitando agregar nuevas tareas.

    El orden en que se van a ir trabajando las \OTs{} depende de varios factores. Uno de ellos es el \textbf{tipo de cliente}; el Jefe de Taller le da mayor prioridad a las \OTs{} de clientes empresariales o comerciales que a las de clientes particulares.

    Dentro de un mismo cliente el Jefe de Taller prioriza aquellos equipos que atañen al funcionamiento del cliente. Por ejemplo: una máquina de mostrador que factura va a tener mayor prioridad que una máquina de oficina interna.

    También se tiene en cuenta el \textbf{tipo de equipo}, siendo de mayor prioridad un equipo que imposibilite trabajar al cliente. Por ejemplo: la computadora personal del cliente tiene mayor prioridad que unos parlantes.
    De todas maneras, el cliente puede indicar en forma particular cuál de sus equipos atender primero.

    Otro factor importante para la priorización de trabajos es el \textbf{vencimiento} de la \OT{}\footnote{La organización mantiene un listado con las órdenes listadas por vencimiento.}. 
    
    Una vez que el técnico asociado comenzó a trabajar en una de sus \OT{}, pasa al estado \textit{En trámite}. Cuando el técnico utiliza los artículos reservados, deja constancia en el campo de observaciones de la \OT{}.  Llegado el caso de que un repuesto reservado del stock no se utilice en el trabajo, se detalla en las observaciones de la \OT{}, se cancela la reserva del stock, y no se factura. Llegado el caso de que los repuestos que se estipularon necesarios para realizar el trabajo no se encuentren en existencia en ese momento, se le notifica al Gerente sobre la situación y la \OT{} pasa a estado \textit{Espera de repuestos} (se realiza un pedido según lo descripto en Sección 2.4).

    A lo largo de la realización del trabajo, el técnico encargado se vale del campo de observaciones que tiene la \OT{} para registrar la evolución de las tareas realizadas, a modo de bitácora. A su vez, puede requerir ayuda de otros técnicos; su participación también se registra en el campo de observaciones (ver Anexo 1, Figura 12). Mientras se realiza una tarea, la misma pasa a estado \textit{En curso}. Al completarse con éxito, pasa al estado \textit{Realizada}.
    
    Si, durante la reparación del equipo, el técnico considera que se deben utilizar otros repuestos o realizar tareas adicionales, se confecciona un nuevo presupuesto sumando al anterior los valores de lo que se necesite. En esta situación, la \OT{} vuelve primero al estado \textit{Cotizada}. Generado el nuevo presupuesto y notificado al cliente sobre el mismo, se pasa al estado \textit{Notificada de cotización}. Puede suceder que el cliente no acepte este segundo presupuesto, por lo tanto se procede a completar las tareas indicadas en el presupuesto anterior (la \OT{} vuelve al estado de \textit{En trámite}, con las tareas y repuestos del primer presupuesto). Si es aceptado el nuevo presupuesto, puede suceder que los repuestos previstos por el técnico no estén en existencia, pasando la \OT{} al estado \textit{Espera de repuestos} (se actúa cómo se comentó previamente).
    
    En cualquier momento de la reparación, el cliente puede solicitar la cancelación del servicio. En ese caso, la \OT{} se cambia de estado a \textit{Pendiente de facturación}, facturándose solamente las tareas \textit{realizadas} hasta el momento de la cancelación. Si existen tareas \textit{en curso}, se deben finalizar antes de cerrar la \OT{}.
    
    Otro caso posible durante la realización del trabajo es que el técnico llegue a la conclusión de que la reparación del equipo va más allá de las capacidades de la organización, por lo tanto se le notifica al cliente de la situación, y la \OT{} se pasa a estado \textit{No se repara}, cerrándose. En este caso, se factura lo que se pudo terminar con éxito. Las tareas \textit{en curso} que no se hayan podido realizar pasan al estado \textit{Cancelada}.\\

    Completadas exitosamente las tareas de la \OT{}, la misma pasa al estado \textit{Pendiente de facturación}. Según la situación, puede suceder:
    \begin{itemize}
        \item El cliente se acerca a la oficina de la organización a retirar el equipo. Se confecciona la factura (ver sección 2.6). La \OT{} se cambia al estado \textit{Entregada}.
        \item El equipo requiere de una instalación especial. Se confecciona la factura (ver sección 2.6) y se abre una nueva \OT{} para tratar el caso de la visita (ver sección 2.5.2).
    \end{itemize}

    Para cualquiera de los dos casos descriptos previamente, el equipo será entregado al cliente una vez cubierta la factura en su totalidad.

    Luego de completadas las tareas de la \OT{}, la reparación tiene una garantía de 30 días. Si dentro de ese plazo el equipo da señales de que la reparación no fue realizada con éxito, el equipo se puede reingresar sin cargo; se abre una nueva \OT{}, referenciando la \OT{} anterior con su número de identificación (ver Anexo 1, Figura 12). En caso de que la nueva \OT{} requiera un repuesto, sí se va a facturar el repuesto (en caso de que, efectivamente se utilice), pero no la mano de obra. \\
    
    Si pasados los 60 días de la finalización del trabajo, el Cliente no retira su equipo, se lo considera SCRAP o abandonado, el Jefe de Taller lo da de baja en el sistema; el equipo ya no puede formar parte de ninguna nueva \OT{}. De cualquier manera, no se pierde su historial en \OTs{} anteriores.

    Estas bajas se pueden deshacer, para subsanar errores en el uso del sistema.

    \subsubsection{Visitas a clientes}

    Existen tareas que realiza la organización que no pueden realizarse en el taller. Es por eso que se suelen dar las visitas o trabajos On-Site. Esta modalidad se utiliza en el caso de tener que realizar instalaciones de redes, cámaras o equipos con configuraciones especiales, así como soporte de software o reparaciones en equipo cuyo tamaño no permite su acarreo al taller.\\

    Comienza cuando el cliente se comunica con la empresa solicitando una visita. Se verifica si el cliente está en el sistema, en caso de no estarlo se lo ingresa.

    El técnico que atendió al cliente abre una nueva \OT{}, correspondiente a la visita a realizar. Para este servicio, la \OT{} atraviesa los mismos estados que los detallados en la sección 2.5.1.
    Una vez abierta, el técnico detalla la tareas a realizar (instalación, relevamiento , revisión); en la mayoría de los casos los clientes ya saben el servicio que desean recibir, aunque se los puede asesorar.

    Puede suceder que el técnico deba pasar por donde se va a realizar el trabajo para relevar el estado de las instalaciones y poder hacer un cálculo de insumos a utilizar; dicha visita no forma parte de la cotización. La visita que se realiza al lugar para llevar a cabo el trabajo sí se cotiza y cuenta como una tarea dentro de la Orden.\\

    Cuando ya se decidió lo que se va a hacer, se realiza un presupuesto que se le entrega al cliente (confeccionado de igual manera que una reparación de equipo). Si se lo acepta se procede a realizar el trabajo. En caso contrario, se debe abonar sólo la tarifa de \textit{RDyP}. En este último caso la \OT{} pasa al estado \textit{Fallida}.\\

    Si durante la realización de la visita, el técnico no puede concluir con el trabajo que está realizando (ya sea porque el cliente no desea continuarlo, porque no tiene las herramientas o los medios necesarios para completarlo, o por algún imprevisto), la \OT{} pasa a estado \textit{Fallida}; se debe acordar con el cliente si se realizará otra o no. En caso de que se acuerde por hacer otra visita, se tratará de una nueva \OT{}.
    
    Si el técnico debe retirar un equipo para ingresarlo al taller, crea su respectiva \OT{} bajo el servicio de Reparación de Equipo. El técnico también asienta en el campo de observación de la nueva \OT{} el número de identificación de la \OT{} de la visita para saber de dónde surgió esa reparación. La \OT{} perteneciente a la visita pasa al estado ``Fallida''.\\

    Una vez realizada la instalación o el servicio on-site (\OT{} en estado \textit{Completada}), se cierra la \OT{} y se pasa a estado \textit{Pendiente de Facturación}.\\

    La facturación correspondiente a una visita se realiza de igual manera que una reparación de equipo. En una \OT{} en estado \textit{Fallida}, se facturan solamente las tareas que se completaron.

    \subsubsection{Trabajos tercerizados}
    Si bien no se lleva adelante ningún servicio en calidad de agente oficial, la organización cuenta en su cartera de clientes con diferentes empresas que sí realizan trabajos de reparación e instalación como agente oficial, pero que, al no contar con sucursales en la zona, tercerizan dichos trabajos.

    Su registro no cambia con respecto a los trabajos habituales, pero se detalla en la orden que proviene de una tercerización.
    En general, están sujetos a contratos que especifican condiciones de tiempo de respuesta.\\

    Las solicitudes llegan vía e-mail detallando la tarea y el lugar a acudir; se detalla también el plazo de vencimiento.
    Una vez realizada la tarea en tiempo y forma, se debe notificar a la empresa tercerizadora que se la completó exitosamente, primero por teléfono (llamando a la mesa de ayuda, indicando el número de orden y fecha y hora de cierre), y luego vía e-mail se detallan tareas realizadas y repuestos utilizados.
    En caso de que no se hayan cumplido los plazos de vencimiento, se realiza el mismo procedimiento de notificación pero la empresa tercerizadora aplica una penalización al momento de facturar el trabajo.

    Los repuestos especializados utilizados son provistos por la empresa tercerizadora, manteniendo un stock básico. Luego de cada trabajo que consuma repuestos, la empresa realiza envíos de reposición de los artículos utilizados para mantener las cantidades actualizadas. 
    Internamente, estos artículo son tratados de igual manera que las partes y componentes del stock (a diferencia que sólo se utilizan para los trabajos tercerizados y no están disponibles para la venta al público).
    Por lo general, no se usan repuestos más allá de los provistos por la empresa, pero en caso de usar otros, se los detalla en el e-mail (al momento de la facturación, son tomados en cuenta y reintegrados).\\

    Si no se puede completar el trabajo dentro de los plazos establecidos, se avisa a la empresa de la situación, pidiendo una prórroga de tiempo. Esta situación se puede dar por falta de repuestos, o por la complejidad del problema.

    A principio de mes la empresa tercerizadora realiza el pago de los trabajo realizados durante el mes anterior, entregando un comprobante que indica las tareas facturadas. La forma de pago fue acordada cuando se firmó el contrato con la empresa tercerizadora.


    \subsubsection{Venta al Público}
    La organización también realiza venta al público de partes, componentes y accesorios. Si bien no es su actividad principal, igualmente realizan ventas a particulares. Se utiliza el mismo stock que en las reparaciones (luego de una venta, se actualiza el stock). 
    Las ventas se dan solamente en el local de la organización, sin contar con ningún medio de venta on-line ni a distancia. Al igual que las otras actividades que realiza la empresa, sólo se podrán realizar ventas a clientes registrados.

    Se lleva registro de las ventas realizadas en una planilla de cálculo, para luego ser entregado al operario contable. Se detallan los siguientes datos:
    \begin{itemize}
        \item Fecha de emisión
        \item Cliente
        \item Empleado que realizó en la venta
        \item Número de factura
        \item Monto
        \item Artículos vendidos
    \end{itemize}

    Cuando se realiza una venta, el cliente se acerca al local de la organización, dónde están expuestos los productos a vender. Si se encuentra uno o varios de su agrado, los lleva al mostrador donde es atendido por el Operario Contable. 
    El cliente puede preguntar por algún producto que no se encuentre expuesto; en caso de que haya en existencia, se le pregunta qué cantidad desea, se lo busca en el stock y se lo ofrece si es que cuenta con la cantidad pedida. Caso contrario se le informa que no hay disponible. El cliente decide si ver algún artículo distinto, o retirarse del local.

    El Operario Contable procede a hacer la suma de los importes de los productos, buscándolos por código en el sistema y calcula el total de la venta. Se le informa el total de la venta; si el cliente lo acepta, abona el importe, se le emite una factura a su nombre, la cual queda asentada en el registro. Se modifica el stock de los artículos vendidos. 
    Si el monto no es aceptado, no se concreta la venta (no se emite factura ni se modifica el stock); el cliente puede modificar su pedido, o no comprar nada.
    Para las ventas al público, la empresa sólo acepta pago en efectivo y notas de crédito.

    Cuando se desea cancelar una venta ya concretada, el Operario Contable emite una nota de crédito a nombre del cliente indicando:
    \begin{itemize}
        \item Fecha de emisión
        \item Fecha de vencimiento (30 días hábiles de la fecha de emición)
        \item Monto
        \item Artículos devueltos y cantidad
        \item Número de factura que registra la venta de los artículos devueltos
    \end{itemize}
    De cualquiera manera, una venta se puede cancelar dentro de los 30 días hábiles de emitida la factura y si los artículos a devolver están en garantía. Si la venta es cancelada, el stock se actualiza con los artículos que volvieron a ingresar (si no fueron devueltos por fallos).

    La nota de crédito puede ser utilizada sólo por el cliente al cual se le extendió dentro del plazo del vencimiento, tanto en compra de artículos como en servicios o giros a su cuenta corriente.

    Si el cliente desea realizar un cambio, los artículos tienen que estar en garantía. Si se realiza el cambio con éxito, se debe actualizar el stock con los artículos que se dieron en garantía (no con los devueltos).

    \subsection{Facturación}

    %Falta incorporar cómo se cotizan y facturan  los traslados para las visitas

    La facturación es realizada por el Operario Contable, en base a las tareas detalladas en las OTs{} cerradas (en estado ``pendiente de facturación''), y los repuestos utilizados. El Operario Contable debe calcular el importe total de cada factura en función de los valores de las tareas realizadas y los repuestos utilizados. Todos estos valores se encuentran especificados en el tarifario y la lista de productos.
    
    Los precios de las tareas que correspondan a una reparación dentro del taller se encuentran establecidas en el tarifario de la organización. Para las visitas, la tarifa del traslado de los técnicos se considera una tarea más, y como tal se encuentra también especificada en el tarifario.

    La factura es emitida a nombre del cliente que figura en la o las \OTs{} relacionadas. Si se trata de un cliente comercial o empresarial, el Operario Contable le envía a algún contacto asociado correspondiente (ver sección 2.2) vía e-mail todas las facturas de trabajos realizados en el mes anterior (es decir, la factura se entrega a mes vencido). Es común que se facturen varias \OTs{} por factura para clientes empresariales.
    
    En el caso de los clientes particulares, se confecciona una factura por trabajo realizado (es decir, una factura contiene sólo una \OT{}). El Operario Contable se comunica con el cliente para avisarle que su factura ya está confeccionada y que deberá acercarse al local de la organización para abonarla.
   
    Cada factura se registra en la Cuenta Corriente del cliente titular. \\

    \subsubsection*{Notas de Crédito}
    Existe la posibilidad de que el cliente solicite a la organización responder ante determinados reclamos sobre uno o más trabajos realizados, si sucede que no está conforme o hay algún inconveniente con el equipo. En estos casos, y a criterio del Gerente, el Operario Contable podrá confeccionar una Nota de Crédito a nombre del cliente, en respuesta a su reclamo. Este documento refleja la devolución de un monto determinado, adicionando la siguiente información: \\
    \begin{itemize}
        \item Fecha de emisión
        \item Monto
        \item Número de \OT{}
        \item Número de factura asociada a la \OT{}
    \end{itemize}

	En casos donde la nota de crédito responda a la devolución de algún producto a la organización, este reingreso deberá reacomodar el stock de dicho producto.
	Las notas de crédito también pueden ser utilizadas para otorgar descuentos a los clientes. Nuevamente, los descuentos se aplican a determinados clientes, a criterio del Gerente.
	Las notas de crédito son entregadas a los clientes de la misma manera que las facturas.\\
	
	\subsubsection{Cobros}
     Actualmente la organización acepta pagos en efectivo, cheques, transferencias bancarias y notas de crédito. Al recibirse un pago, el Operario Contable deja constancia del mismo en la Cuenta Corriente del cliente. Un pago puede saldar una o más facturas. No existen diferencias entre el registro de cobro a empresas y particulares.\\



    \clearpage
    \section{Casos de uso}
    \subsection{Listado de Casos de uso}
    \begin{multicols}{2}
    \begin{enumerate}	
        \subsubsection*{Clientes}
            \item Agregar Cliente
            \item Modificar Cliente
            \item Eliminar Cliente
            \item Listar Clientes
            \item Consultar saldo del Cliente
            \item Eliminar equipo
            \item Listar equipos
            \item Retirar equipo
        \subsubsection*{Proveedores}
            \item Agregar Proveedor
            \item Modificar Proveedor
            \item Eliminar Proveedor
            \item Listar Proveedores	
        \subsubsection*{Técnicos}
            \item Agregar Técnico
            \item Modificar Técnico
            \item Eliminar Técnico
            \item Listar Técnicos	
            \item Asignar tarea a Técnico
            \item Remover tarea de Técnico
            \item Listar las tareas que puede realizar un Técnico
        \subsubsection*{Rubro}
            \item Agregar Rubro
            \item Modificar Rubro 
            \item Eliminar Rubro
            \item Listar Rubros
            \item Agregar \textit{RDyP}
            \item Modificar precio de \textit{RDyP}
            \item Eliminar \textit{RDyP}
            \item Agregar Tarea
            \item Modificar Tarea
            \item Eliminar Tarea
            \item Listar Tareas
            \item Crear Tarifa
            \item Modificar precio de Tarifa
            \item Eliminar Tarifa
        \subsubsection*{Productos}
            \item Agregar producto
            \item Modificar producto
            \item Eliminar producto
            \item Listar productos
            \item Agregar Proveedor a un producto
            \item Eliminar Proveedor de un producto
            \item Ver proveedores de un producto
            \item Actualizar stock
        \subsubsection*{Tipo servicios}
            \item Agregar Tipo servicio
            \item Modificar Tipo servicio
            \item Eliminar Tipo servicio
            \item Listar tarifario de un tipo de servicio
            \item Listar tipos de servicios
        \subsubsection*{Órdenes de trabajo}
			\item Atender solicitud de servicio            
            \item Comenzar revisión
            \item Finalizar revisión
            \item Reasignar tarea a técnico
            \item Confirmar presupuesto
            \item Cancelar \OT{}
            \item Comenzar tarea 
            \item Finalizar tarea
            \item Cancelar tarea
            \item Registrar consulta de cliente
            \item Listar \OTs{}
            \item Cambiar encargado de \OT{}
            \item Agregar observación a una tarea
            \item Eliminar observación en una tarea
            \item Listar tareas de una \OT{}
            \item Consultar detalle de una tarea de una \OT{}
        \subsubsection*{Facturas, Cobros y Ventas}
            \item Generar factura 
            \item Listar facturas
            \item Registrar pago de factura
            \item Generar Nota de Crédito
        \subsubsection*{Pedidos a proveedores}
            \item Registrar pedido a proveedor
            \item Modificar pedido 
            \item Eliminar pedido a proveedor
            \item Listar pedidos
            \item Registrar recepción de pedido
            \item Registrar pago a proveedor
    \end{enumerate}
    \end{multicols}

    \clearpage

    \subsection{Casos de uso breves}


    \begin{enumerate}



        \subsubsection{Clientes}



        \item 	\textbf{Caso de uso}: Agregar Cliente\\
                \textbf{Actores}: Persona, Empleado\\
                \textbf{Tipo}: Primario\\
                \textbf{Descripción}: Una persona solicita un Tipo servicio. El empleado le toma los datos y se crea un nuevo cliente con sus datos en el sistema.

        \item 	\textbf{Caso de uso}: Modificar Cliente\\
                \textbf{Actores}: Cliente, Empleado\\
                \textbf{Tipo}: Primario\\
                \textbf{Descripción}: Un cliente dado de alta indica al empleado cuáles de sus datos desea cambiar. El empleado registra los cambios en su registro.

        \item 	\textbf{Caso de uso}: Eliminar Cliente\\
                \textbf{Actores}: Cliente, Empleado\\
                \textbf{Tipo}: Primario\\
                \textbf{Descripción}: Se da de baja un cliente (no se pueden registrar nuevas \OTs{} a su nombre ni registrar nuevos movimientos en su Cuenta Corriente).

        \item 	\textbf{Caso de uso}: Listar Clientes\\
                \textbf{Actores}: Empleado\\
                \textbf{Tipo}: Secundario\\
                \textbf{Descripción}: Se muestran los clientes dados de alta con sus datos

        \item 	\textbf{Caso de uso}: Agregar movimiento de Cuenta Corriente\\
                \textbf{Actores}: Operario Contable\\
                \textbf{Tipo}: Primario\\
                \textbf{Descripción}: Se ingresa un nuevo movimiento en la Cuenta Corriente de un cliente indicando fecha, monto, si es de crédito o débito, y el saldo actual del cliente.

        \item   \textbf{Caso de uso}: Eliminar movimiento de Cuenta Corriente\\
                \textbf{Actores}: Operario Contable\\
                \textbf{Tipo}: Primario\\
                \textbf{Descripción}: Se elimina movimiento de la Cuenta Corriente de un cliente

        \item   \textbf{Caso de uso}: Ver resumen de Cuenta Corriente\\
                \textbf{Actores}: Operario Contable\\
                \textbf{Tipo}: Secundario\\
                \textbf{Descripción}: Se muestran los distintos movimientos de una Cuenta Corriente, podiendo ordenar por fecha, monto, si son de crédito o débito



        \subsubsection{Proveedores}



        \item 	\textbf{Nombre del Caso de uso}: Agregar Proveedor\\
                \textbf{Actores}: Proveedor, Empleado\\
                \textbf{Tipo}: Primario\\
                \textbf{Descripción}: El empleado registra los datos del proveedor y su información de contacto. Se crea un proveedor nuevo en el sistema.
        
        \item 	\textbf{Nombre del Caso de uso}: Modificar Proveedor\\
                \textbf{Actores}: Proveedor, Empleado\\
                \textbf{Tipo}: Primario\\
                \textbf{Descripción}: Se modifica el registro de un proveedor dado de alta con sus datos nuevos
        
        \item 	\textbf{Nombre del Caso de uso}: Eliminar Proveedor\\
                \textbf{Actores}: Empleado\\
                \textbf{Tipo}: Primario\\
                \textbf{Descripción}: Se da de baja un proveedor (no se le pueden realizar más pedidos)
        
        \item 	\textbf{Nombre del Caso de uso}: Listar Proveedores\\
                \textbf{Actores}: Empleado\\
                \textbf{Tipo}: Secundario\\
                \textbf{Descripción}: Se muestran los proveedores dados de alta con sus datos. Se pueden ordenar por orden alfabético, por ciudad de origen



        \subsubsection{Empleados}



        \item 	\textbf{Nombre del Caso de uso}: Agregar Empleado\\
                \textbf{Actores}: Persona, Gerente\\
                \textbf{Tipo}: Primario\\
                \textbf{Descripción}: Se registra un nuevo empleado con sus datos asociados.
        
        \item 	\textbf{Nombre del Caso de uso}: Modificar Empleado\\
                \textbf{Actores}: Jefe de Taller\\
                \textbf{Tipo}: Primario\\
                \textbf{Descripción}: Se actualiza el registro del empleado dado de alta con sus nuevos datos
        
        \item 	\textbf{Nombre del Caso de uso}: Eliminar Empleado\\
                \textbf{Actores}: Gerente\\
                \textbf{Tipo}: Primario\\
                \textbf{Descripción}: Se da de baja un empleado (no se lo puede asociar en nuevas \OTs{})
        
        \item 	\textbf{Nombre del Caso de uso}: Listar Empleados\\
                \textbf{Actores}: Gerente\\
                \textbf{Tipo}: Secundario\\
                \textbf{Descripción}: Se muestra un listado de todos los empleados dados de alta con sus datos. Se puede ordenar el listado para mostrarlos por orden alfabético, por los ítems de OT en los que aparecen.
        
        \item 	\textbf{Nombre del Caso de uso}: Crear Rol de Empleado\\
                \textbf{Actores}: Gerente\\
                \textbf{Tipo}: Primario\\
                \textbf{Descripción}: Se crea un nuevo tipo de empleado asignable a un empleado
        
        \item 	\textbf{Nombre del Caso de uso}: Eliminar Rol de Empleado\\
                \textbf{Actores}: Gerente\\
                \textbf{Tipo}: Primario\\
                \textbf{Descripción}: Se da de baja un rol de empleado (no se puede asignar nuevamente a algún empleado)
        
        \item 	\textbf{Nombre del Caso de uso}: Listar Roles de Empleado\\
                \textbf{Actores}: Gerente\\
                \textbf{Tipo}: Secundario\\
                \textbf{Descripción}: Se muestra un listado de todos los roles de empleados con su información asociada. Se los puede filtrar por las distintas actividades que tengan asociadas.
        


        \subsubsection{Equipos}



        \item 	\textbf{Nombre del Caso de uso}: Agregar tipo de Equipo\\
                \textbf{Actores}: Jefe de taller\\
                \textbf{Tipo}: Primario\\
                \textbf{Descripción}: Se realiza una alta de la tipificacion de tipo equipo que podrá ingresar al taller 
        
        \item 	\textbf{Nombre del Caso de uso}: Modificar tipo de Equipo\\
                \textbf{Actores}: Jefe de taller\\
                \textbf{Tipo}: Primario\\
                \textbf{Descripción}: Se realiza una modificación a un tipo de equipo dado de alta.
        
        \item 	\textbf{Nombre del Caso de uso}: Eliminar tipo de Equipo\\
                \textbf{Actores}: Jefe de taller\\
                \textbf{Tipo}: Primario\\
                \textbf{Descripción}: Se da de baja un tipo de equipo (no se pueden ingresar un nuevo equipo de ese tipo)
        
        \item 	\textbf{Nombre del Caso de uso}: Listar tipos de Equipo\\
                \textbf{Actores}: Jefe de taller\\
                \textbf{Tipo}: Secundario\\
                \textbf{Descripción}: Se listan todos los tipos de equipo dados de alta
        
        \item 	\textbf{Nombre del Caso de uso}: Agregar Equipo\\
                \textbf{Actores}: Empleado\\
                \textbf{Tipo}: Primario\\
                \textbf{Descripción}: Se agrega un nuevo equipo con un tipo de equipo dado de alta y cliente dado de alta asociado
        
        \item 	\textbf{Nombre del Caso de uso}: Modificar Equipo\\
                \textbf{Actores}: Empleado\\
                \textbf{Tipo}: Primario\\
                \textbf{Descripción}: Se modifican los datos asociados de un equipo dado de alta
        
        \item 	\textbf{Nombre del Caso de uso}: Eliminar Equipo\\
                \textbf{Actores}: Empleado\\
                \textbf{Tipo}: Primario\\
                \textbf{Descripción}: Se da de baja un equipo (no se le puede asociar en nuevos trabajos)
        
        \item 	\textbf{Nombre del Caso de uso}: Listar equipos\\
                \textbf{Actores}: Empleado\\
                \textbf{Tipo}: Secundario\\
                \textbf{Descripción}: Se listan todos los equipos dados de alta con su información correspondiente. Se pueden ordenar según el dueño, según la última orden de trabajo en la que aparecieron, según si son SCRAP.



        \subsubsection{Stock}



        \item 	\textbf{Nombre del Caso de Uso}: Agregar artículo de Stock\\
                \textbf{Actores}: Jefe de Taller\\
                \textbf{Tipo}: Primario\\
                \textbf{Descripcion}: Se realiza un alta a un artículo del stock

        \item 	\textbf{Nombre del Caso de Uso}: Modificar articulo de Stock\\
                \textbf{Actores}: Jefe de Taller\\
                \textbf{Tipo}: Primario\\
                \textbf{Descripcion}: Se modifica la información de un artículo dado de alta

        \item 	\textbf{Nombre del Caso de Uso}: Eliminar articulo de Stock\\
                \textbf{Actores}: Jefe de Taller\\
                \textbf{Tipo}: Primario\\
                \textbf{Descripcion}: Se realiza una baja de un articulo del stock (no se podrá referenciar nuevamente)

        \item 	\textbf{Nombre del Caso de Uso}: Actualizar stock\\
                \textbf{Actores}: Jefe de Taller\\
                \textbf{Tipo}: Primario\\
                \textbf{Descripcion}: Se acutualizan las cantidades de los distintos productos seleccionados del stock

        \item 	\textbf{Nombre del Caso de Uso}: Listado Stock\\
                \textbf{Actores}: Secundario\\
                \textbf{Tipo}: Empleado\\
                \textbf{Descripcion}: Se listan todos los artículos dados de alta. Se puede ordenar el listado para que los muestre según el proveedor que tengan asociado, según las cantidades que tenga cada artículo. 

        \item 	\textbf{Nombre del Caso de Uso}: Ver detalle de artículo\\
                \textbf{Actores}: Secundario\\
                \textbf{Tipo}: Empleado\\
                \textbf{Descripcion}: Se listan todos los datos pertenecientes a un artículo dado de alta, con los distintos proveedores asociados, sus precios y garantía que posea (si que es tiene).

        \item 	\textbf{Nombre del Caso de Uso}: Agregar marca de artículo\\
                \textbf{Actores}: Jefe de Taller\\
                \textbf{Tipo}: Primario\\
                \textbf{Descripcion}: Se crea una nueva marca de artículo

        \item 	\textbf{Nombre del Caso de Uso}: Eliminar marca de artículo\\
                \textbf{Actores}: Jefe de Taller\\
                \textbf{Tipo}: Primario\\
                \textbf{Descripcion}: Se elimina una marca de artículo (no se podrá vincular para nuevos artículos)

        \item 	\textbf{Nombre del Caso de Uso}: Listar marcas de artículo\\
                \textbf{Actores}: Jefe de Taller\\
                \textbf{Tipo}: Secundario\\
                \textbf{Descripcion}: Se listan todas las marcas de artículos registradas	



        \subsubsection{Actividades}



        \item 	\textbf{Nombre del Caso de Uso}: Agregar nueva actividad\\
                \textbf{Actores}: Primario\\
                \textbf{Tipo}: Jefe de Taller\\
                \textbf{Descripcion}: Se registra una nueva actividad que realiza la empresa (figurarán en el ítem)

        \item 	\textbf{Nombre del Caso de Uso}: Modificar actividad\\
                \textbf{Actores}: Primario\\
                \textbf{Tipo}: Jefe de Taller\\
                \textbf{Descripcion}: Se modifica una actividad dada de alta

        \item 	\textbf{Nombre del Caso de Uso}: Eliminar actividad\\
                \textbf{Actores}: Primario\\
                \textbf{Tipo}: Jefe de Taller\\
                \textbf{Descripcion}: Se da de baja una actividad existente (no se podrá referenciar más en nueva OT)

        \item 	\textbf{Nombre del Caso de Uso}: Listar actividades\\
                \textbf{Actores}: Secundario\\
                \textbf{Tipo}: Empleado\\
                \textbf{Descripcion}: Se listan actividades dadas de alta. Se podrán ordenar las actividades por roles que tengan asociadas



        \subsubsection{Órdenes de trabajo}



        \item 	\textbf{Nombre del Caso de uso}: Abrir nueva Orden de Trabajo\\
                \textbf{Actores}: Jefe de taller\\
                \textbf{Tipo}: Primario\\
                \textbf{Descripción}: El cliente pide un Tipo servicio a la organización; el Jefe de Taller abre una nueva Orden de Trabajo para manejar el Tipo servicio a realizar

        \item 	\textbf{Nombre del Caso de uso}: Listar OT\\
                \textbf{Actores}: Empleado\\
                \textbf{Tipo}: Primario\\
                \textbf{Descripción}: se muestra un listado con las \OTs{}. El listado se puede ordenar según cliente que tengan asociada, fecha de creación.

        \item 	\textbf{Nombre del Caso de uso}: Listar ítems\\
                \textbf{Actores}: Empleado\\
                \textbf{Tipo}: Primario\\
                \textbf{Descripción}: se muestra un listado con los ítems creado. El listado se puede mostrar según orden a la que pertenezcan, empleado vinculado, por actividad que tenga asociada, por fecha de vencimiento, por repuestos reservados, por equipo que tengan, por el estado en que se encuentren.

        \item 	\textbf{Nombre del Caso de uso}: Modificar OT\\
                \textbf{Actores}: Jefe de taller\\
                \textbf{Tipo}: Primario\\
                \textbf{Descripción}: se realiza una modificación sobre la información de una OT existente.

        \item 	\textbf{Nombre del Caso de uso}: Modificar ítem\\
                \textbf{Actores}: Jefe de taller\\
                \textbf{Tipo}: Primario\\
                \textbf{Descripción}: se realiza una modificación de la información de un ítem en una OT existente. 

        \item 	\textbf{Nombre del Caso de uso}: Eliminar OT\\
                \textbf{Actores}: Jefe de taller\\
                \textbf{Tipo}: Primario\\
                \textbf{Descripción}: Se elimina una OT.

        \item 	\textbf{Nombre del Caso de uso}: Agregar nuevo estado de ítem\\
                \textbf{Actores}: Jefe de taller\\
                \textbf{Tipo}: Primario\\
                \textbf{Descripción}: se crea un nuevo estado que puede tomar un ítem.

        \item 	\textbf{Nombre del Caso de uso}: Eliminar estados de un ítem\\
                \textbf{Actores}: Jefe de taller\\
                \textbf{Tipo}: Primario\\
                \textbf{Descripción}: se elimina un estado de ítem.
        
        \item 	\textbf{Nombre del Caso de uso}: Listar estados de ítem\\
                \textbf{Actores}: Jefe de taller\\
                \textbf{Tipo}: Secundario\\
                \textbf{Descripción}: se listan todos los estados de ítem dados de alta.

        \item 	\textbf{Nombre del Caso de uso}: Ver detalle de Orden de Trabajo\\
                \textbf{Actores}: Empleado\\
                \textbf{Tipo}: Primario\\
                \textbf{Descripción}: se muestra un listado de ítems sobre una orden de trabajo particular.

        \item 	\textbf{Nombre del Caso de uso}: Ver detalle de un ítem\\
                \textbf{Actores}: Empleado\\
                \textbf{Tipo}: Primario\\
                \textbf{Descripción}: el empleado solicita un listado de observaciones y artículos reservados para un item sobre una orden de trabajo.



        \subsubsection{Facturas, Cobros y Ventas}



        \item 	\textbf{Nombre del Caso de Uso}: Generar Factura\\
                \textbf{Actores}: Operario Contable, Cliente\\
                \textbf{Tipo}: Primario\\
                \textbf{Descripción}: El Operario Contable solicita al sistema generar una factura con una OT cerrada o con artículos vendidos

        \item 	\textbf{Nombre del Caso de Uso}: Listar facturas\\
                \textbf{Actores}: Operario Contable\\
                \textbf{Tipo}: Primario\\
                \textbf{Descripción}: El Operario Contable solicita al sistema mostrar las facturas existentes, pudiéndolas ordenar por monto, por fecha de emisión, por cliente al cual se emitió, por si están pagadas totalmente, pagadas parcialmente, o sin pagar.

        \item 	\textbf{Nombre del Caso de Uso}: Registrar Pago de Factura\\
                \textbf{Actores}: Operario Contable, Cliente\\
                \textbf{Tipo}: Primario\\
                \textbf{Descripción}: El cliente realiza el pago de una factura
        
        \item 	\textbf{Nombre del Caso de Uso}: Realizar venta\\
                \textbf{Actores}: Operario Contable, Cliente\\
                \textbf{Tipo}: Primario\\
                \textbf{Descripción}: El Operario Contable solicita al sistema actualizar el stock de los artículos involucrados en una venta, y emitir una factura a nombre del cliente dado de alta

        \item 	\textbf{Nombre del Caso de Uso}: Generar Nota de Crédito\\
                \textbf{Actores}: Operario Contable, Cliente\\
                \textbf{Tipo}: Primario\\
                \textbf{Descripción}: El Operario Contable genera un Nota de Crédito anulando uno o más renglones de una o más facturas a nombre de un cliente. La Nota es emitida y entregada al cliente.



        \subsubsection{Pedidos a proveedores}



        \item 	\textbf{Nombre del Caso de Uso}: Crear pedido a proveedor\\
                \textbf{Actores}: Operario Contable\\
                \textbf{Tipo}: Primario\\
                \textbf{Descripción}: El Operario Contable arma el pedido de los productos de stock dados de alta y lo almacena en el sistema.

        \item 	\textbf{Nombre del Caso de Uso}: Modificar pedido a Proveedor\\
                \textbf{Actores}: Operario Contable\\
                \textbf{Tipo}: Primario\\
                \textbf{Descripción}: El Operario Contable modifica un pedido a proveedor guardado en el sistema antes de que éste sea cerrado para envío.

        \item 	\textbf{Nombre del Caso de Uso}: Eliminar pedido a proveedor\\
                \textbf{Actores}: Operario Contable\\
                \textbf{Tipo}: Primario\\
                \textbf{Descripción}: El Operario Contable elimina del sistema un pedido a un proveedor que no ha sido enviado.

        %Ver que pasa si se quiere cancelar un pedido que ya fue cerrado para su envio

        \item 	\textbf{Nombre del Caso de Uso}: Listar pedidos \\
                \textbf{Actores}: Operario Contable\\
                \textbf{Tipo}: Secundario\\
                \textbf{Descripción}: El Operario Contable solicita el listado de los pedidos registrados. El listado se podrá desplegar de manera que se muestren primero los pedidos enviados y los no enviados, los pedidos recibidos y por su estado.

        \item 	\textbf{Nombre del Caso de Uso}: Registrar recepción de pedido\\
                \textbf{Actores}: Operario Contable\\
                \textbf{Tipo}: Primario\\
                \textbf{Descripción}: El Operario Contable registra que un pedido guardado en el sistema se recibió y en qué estado. 

        \item 	\textbf{Nombre del Caso de Uso}: Registrar pago a proveedor\\
                \textbf{Actores}: Operario Contable\\
                \textbf{Tipo}: Primario\\
                \textbf{Descripción}: Se registra un pago a un proveedor dado de alta con un pedido realizado.
    \end{enumerate}
    \clearpage














    

    \subsection{Casos de Uso expandidos}

    %DEFINIMOS UN CONTADOR
    \newcounter{step}
    \newcommand\inc{\stepcounter{step}\textbf{\thestep. }}

    %para resetear el contador del curso normal
    \newcommand\resetinc{\setcounter{step}{0}}

    %raya para separar CUs
    \newcommand\raya{\noindent\rule{169mm}{0.8mm}\\}

    %CUANDO SE TERMINA UN CU, PONER ESTE
    %\finCU{}
    \newcommand\finCU{\resetinc{}\raya{}}


	\begin{longtable}{ |p{8cm}|p{8cm}| }
		\hline
		\multicolumn{2}{|p{16cm}|}{\textbf{Caso de uso}: Actualizar stock}\\
		\multicolumn{2}{|p{16cm}|}{\textbf{Tipo}: Primario Esencial}\\
		\multicolumn{2}{|p{16cm}|}{\textbf{Actores}: Gerente}\\
		\multicolumn{2}{|p{16cm}|}{\textbf{Descripción}: el Gerente modifica el stock de un producto manualmente}\\
		\multicolumn{2}{|p{16cm}|}{\textbf{Precondiciones}: -}\\
        \multicolumn{2}{|p{16cm}|}{\textbf{Postcondiciones}: Stock de un producto modificado}\\
		\hline
		\multicolumn{2}{|c|}{\textbf{Curso normal de los eventos}}\\
		\hline
		\textbf{Acción de los actores} & \textbf{Respuesta del sistema}\\
		\hline
			\inc Este CU comienza cuando el Gerente desea modificar manualmente el stock de un producto& \\
			\hline
			\inc  El Gerente ingresa el código de referencia interna del producto& \\
			\hline
		    & \inc Busca el producto \\
			\hline
			& \inc Muestra los datos del producto\\
			\hline


			\inc El Gerente ingresa una nueva cantidad de stock para el producto & \\
			\hline
			& \inc Registra la nueva cantidad en el producto\\
			\hline
			\inc  Repetir pasos 2 al 6 hasta que no desee actualizar más stocks de productos& \\
			\hline
			\inc Fin CU. & \\


		\hline
		\multicolumn{2}{|c|}{\textbf{Cursos alternos}}\\
		\hline
		\multicolumn{2}{|p{16cm}|}{\textbf{4. }El sistema informa que el producto no existe. Ir al paso 7}\\
		\hline
	\end{longtable}


    \finCU{}

    \begin{longtable}{ |p{8cm}|p{8cm}| }
        \hline
        \multicolumn{2}{|p{16cm}|}{\textbf{Caso de uso}: Agregar Técnico}\\
        \multicolumn{2}{|p{16cm}|}{\textbf{Tipo}: Primario Esencial}\\
        \multicolumn{2}{|p{16cm}|}{\textbf{Actores}: Gerente, Técnico}\\
        \multicolumn{2}{|p{16cm}|}{\textbf{Descripción}: El Gerente agrega un nuevo Técnico al sistema, y si lo desea, le indica al sistema las tareas que está capacitado para realizar.}\\
        \multicolumn{2}{|p{16cm}|}{\textbf{Precondiciones}: - }\\
        \multicolumn{2}{|p{16cm}|}{\textbf{Postcondiciones}: Técnico creado}\\
        \hline
        \multicolumn{2}{|c|}{\textbf{Curso normal de los eventos}}\\
        \hline
        \textbf{Acción de los actores} & \textbf{Respuesta del sistema}\\
        \hline
            \inc Este CU comienza cuando el Gerente desea agregar un nuevo Técnico.& \\
            \hline
            \inc  El Gerente le solicita el numero de documento al nuevo Técnico, el Técnico se lo da. El Gerente le solicita al Sistema agregar el Técnico ingresando su número de documento.& \\
            \hline
            & \inc  Crea un nuevo Técnico.\\
            \hline
            & \inc  Muestra los datos del Técnico.\\
            \hline


            \inc El Gerente le solicita el resto de sus datos: nombre y apellido completos, número de teléfono, domicilio y correo electrónico. Los ingresa en el sistema y le solicita que los guarde.& \\
            \hline
            & \inc Registra los datos del Técnico.\\
            \hline
            \inc El Gerente ingresa un Tipo de Tarea al sistema indicando su nombre y le solicita al sistema que lo busque.& \\
            \hline
            & \inc Busca el Tipo de Tarea. \\
            \hline


            & \inc Muestra los datos del Tipo de Tarea.\\
            \hline
            \inc El Gerente ingresa el nombre del Tipo de Equipo para el Tipo de Tarea & \\
            \hline
            & \inc Busca el Tipo de Equipo.\\
            \hline
            & \inc Muestra los datos del Tipo de Equipo\\
            \hline


            \inc El Gerente le solicita al sistema agregar al nuevo Técnico ese Tipo de Tarea para ese Tipo de Equipo.& \\
            \hline
            & \inc Agrega al nuevo Técnico ese tipo de Tarea para ese Tipo de Equipo\\
            \hline
            \inc Repetir pasos 7 a 14 hasta que el Gerente no desee agregar más Tareas al Técnico.& \\
            \hline
            \inc Fin CU. & \\
        \hline
        \multicolumn{2}{|c|}{\textbf{Cursos alternos}}\\
        \hline
        \multicolumn{2}{|p{16cm}|}{\textbf{4. }El Sistema informa que el Técnico ya existe. Fin CU.}\\
        \hline
        \multicolumn{2}{|p{16cm}|}{\textbf{9. }El Sistema informa que el Tipo de Tarea no existe. Ir a paso 15.}\\
        \hline
        \multicolumn{2}{|p{16cm}|}{\textbf{12a. }El Sistema informa que el Tipo de Equipo no existe. Ir a paso 15.}\\
        \hline	
        \multicolumn{2}{|p{16cm}|}{\textbf{12b. }El Sistema informa que el Tipo de Tarea no se puede realizar para el Tipo de Equipo ingresado. Ir al paso 15.}\\
        \hline	
    \end{longtable}

    \finCU{}


	\begin{longtable}{ |p{8cm}|p{8cm}| }
		\hline
		\multicolumn{2}{|p{16cm}|}{\textbf{Caso de uso}: Agregar tipo de equipo}\\
		\multicolumn{2}{|p{16cm}|}{\textbf{Tipo}: Primario Esencial}\\
		\multicolumn{2}{|p{16cm}|}{\textbf{Actores}: Jefe de Taller}\\
		\multicolumn{2}{|p{16cm}|}{\textbf{Descripción}: El Jefe de Taller crea un nuevo tipo de equipo, indicando los tipos de tarea que se le podrán realizar, en qué tipo de servicio podrá ser incluído, y creando tarifas con el tipo de equipo creado}\\
		\multicolumn{2}{|p{16cm}|}{\textbf{Precondiciones}: -}\\
		\multicolumn{2}{|p{16cm}|}{\textbf{Postcondiciones}: Tipo de equipo creado}\\
		\hline
		\multicolumn{2}{|c|}{\textbf{Curso normal de los eventos}}\\
		\hline
		\textbf{Acción de los actores} & \textbf{Respuesta del sistema}\\
		\hline
			\inc Este CU comienza cuando el Jefe de Taller desea agregar un nuevo tipo de equipo & \\
			\hline
            \inc El Jefe de Taller ingresa un nombre y una descripción del tipo de equipo a dar de alta y le solicita al sistema que lo cree & \\
			\hline
			& \inc Crea el tipo de equipo \\
			\hline
			& \inc Muestra los datos del tipo de equipo creado\\
			\hline


			\inc El Jefe de Taller ingresa el nombre de un tipo de tarea para indicar que ese tipo de tarea se podrá realizar para el tipo de equipo & \\
			\hline
			& \inc Busca el tipo de tarea \\
			\hline
            & \inc Vincula el tipo de equipo con el tipo de tarea \\
			\hline
            & \inc Muestra los nuevos del tipo de equipo\\
			\hline


            \inc Repetir pasos 5 a 8 hasta que el Jefe de Taller no desee indicar más tipos de tarea para el tipo de equipo&\\
			\hline
            %creamos tarifas
			%\inc El Jefe de Taller ingresa el nombre de un tipo de tarea para agregar a una tarifa & \\
			%\hline
			%& \inc Busca el tipo de tarea \\
			%\hline
            %& \inc Muestra los datos del tipo de tarea\\
			%\hline


            %\inc El Jefe de Taller ingresa el nombre de un tipo de servicio para agregar a la tarifa&\\
			%\hline
            %& \inc Busca el tipo de servicio \\
			%\hline
            %& \inc Muestra los datos del tipo de servicio\\
			%\hline
            %\inc El Jefe de Taller ingresa un precio para la tarifa &\\
			%\hline


            %& \inc Crea la tarifa con el tipo de equipo, tipo de tarea, tipo de serivicio y el precio ingresados\\
			%\hline
            %& \inc Muestra los datos de la tarifa creada\\
			%\hline
            %\inc Repetir pasos 10 al 18 hasta que el Jefe de Taller no desee crear más tarifas para el tipo de equipo creado& \\
			%\hline
			\inc Fin CU. & \\
		\hline
		\multicolumn{2}{|c|}{\textbf{Cursos alternos}}\\
		\hline
		\multicolumn{2}{|p{16cm}|}{\textbf{4. }El sistema informa que el tipo de equipo ingresado ya existe. Fin CU.}\\
		\hline
		\multicolumn{2}{|p{16cm}|}{\textbf{7. }El sistema informa que el tipo de tarea ingresado no existe. Ir al paso 9}\\
		\hline	
		\multicolumn{2}{|p{16cm}|}{\textbf{8. }El sistema informa que el tipo de equipo y el tipo de tarea ya están vinculados. Ir al paso 9}\\
		\hline	
		%\multicolumn{2}{|p{16cm}|}{\textbf{12a. }El sistema informa que el tipo de tarea ingresado no existe. Ir al paso 19}\\
		%\hline	
		%\multicolumn{2}{|p{16cm}|}{\textbf{12b. }El sistema informa que el tipo de tarea ingresado no se puede realizar para el tipo de equipo. Ir al paso 19}\\
		%\hline	
		%\multicolumn{2}{|p{16cm}|}{\textbf{15a. }El sistema informa que el tipo de servicio ingresado no existe. Ir al paso 19}\\
		%\hline	
		%\multicolumn{2}{|p{16cm}|}{\textbf{15b. }El sistema informa que el tipo de tarea ingresado no se puede realizar bajo el tipo de servicio ingresado. Ir al paso 19}\\
		%\hline	
		%\multicolumn{2}{|p{16cm}|}{\textbf{18. }El sistema informa que ya existe una tarifa con el tipo de equipo, tipo de equipo y tipo de servicio ingresados. Ir al paso 19}\\
		%\hline	
	\end{longtable}
    
    \finCU{}


	\begin{longtable}{ |p{8cm}|p{8cm}| }
		\hline
		\multicolumn{2}{|p{16cm}|}{\textbf{Caso de uso}: Agregar tipo de servicio}\\
		\multicolumn{2}{|p{16cm}|}{\textbf{Tipo}: Primario Esencial}\\
		\multicolumn{2}{|p{16cm}|}{\textbf{Actores}: Jefe de Taller}\\
		\multicolumn{2}{|p{16cm}|}{\textbf{Descripción}: El Jefe de Taller crea un nuevo tipo de servicio, indicando los tipos de tarea que abarcará, y creando tarifas con el tipo de servicio creado}\\
		\multicolumn{2}{|p{16cm}|}{\textbf{Precondiciones}: -}\\
		\multicolumn{2}{|p{16cm}|}{\textbf{Postcondiciones}: Tipo de servicio creado}\\
		\hline
		\multicolumn{2}{|c|}{\textbf{Curso normal de los eventos}}\\
		\hline
		\textbf{Acción de los actores} & \textbf{Respuesta del sistema}\\
		\hline
			\inc Este CU comienza cuando el Jefe de Taller desea agregar un nuevo tipo de servicio & \\
			\hline
            \inc El Jefe de Taller ingresa un nombre y una descripción del tipo de servicio a dar de alta y le solicita al sistema que lo cree & \\
			\hline
			& \inc Crea el tipo de servicio \\
			\hline
			& \inc Muestra los datos del tipo de servicio creado\\
			\hline


			\inc El Jefe de Taller ingresa el nombre de un tipo de tarea para indicar que ese tipo de tarea se podrá realizar bajo el tipo de servicio & \\
			\hline
			& \inc Busca el tipo de tarea \\
			\hline
            & \inc Vincula el tipo de servicio con el tipo de tarea \\
			\hline
            & \inc Muestra los datos del tipo de servicio\\
			\hline


            \inc Repetir pasos 5 a 8 hasta que el Jefe de Taller no desee indicar más tipos de tarea para el tipo de servicio&\\
			\hline
            %creamos tarifas. PASO 10
			%\inc El Jefe de Taller ingresa el nombre de un tipo de tarea para agregar a una tarifa & \\
			%\hline
			%& \inc Busca el tipo de tarea \\
			%\hline
            %& \inc Muestra los datos del tipo de tarea\\
			%\hline


            %\inc El Jefe de Taller ingresa el nombre de un tipo de equipo para agregar a la tarifa&\\
			%\hline
            %& \inc Busca el tipo de equipo \\
			%\hline
            %& \inc Muestra los datos del tipo de equipo\\
			%\hline
            %\inc El Jefe de Taller ingresa un precio para la tarifa &\\
			%\hline


            %& \inc Crea la tarifa con el tipo de servicio, tipo de tarea, tipo de equipo y el precio ingresados\\
			%\hline
            %& \inc Muestra los datos de la tarifa creada\\
			%\hline
            %\inc Repetir pasos 10 al 18 hasta que el Jefe de Taller no desee crear más tarifas para el tipo de servicio creado& \\
			%\hline
			\inc Fin CU. & \\
		\hline
		\multicolumn{2}{|c|}{\textbf{Cursos alternos}}\\
		\hline
		\multicolumn{2}{|p{16cm}|}{\textbf{4. }El sistema informa que el tipo de servicio ingresado ya existe. Fin CU.}\\
		\hline
		\multicolumn{2}{|p{16cm}|}{\textbf{7. }El sistema informa que el tipo de tarea ingresado no existe. Ir al paso 9}\\
		\hline	
		\multicolumn{2}{|p{16cm}|}{\textbf{8. }El sistema informa que el tipo de servicio y el tipo de tarea ya están vinculados. Ir al paso 9}\\
		\hline	
		%\multicolumn{2}{|p{16cm}|}{\textbf{12a. }El sistema informa que el tipo de tarea ingresado no existe. Ir al paso 19}\\
		%\hline	
		%\multicolumn{2}{|p{16cm}|}{\textbf{12b. }El sistema informa que el tipo de tarea ingresado no se puede realizar bajo el tipo de servicio. Ir al paso 19}\\
		%\hline	
		%\multicolumn{2}{|p{16cm}|}{\textbf{15a. }El sistema informa que el tipo de equipo ingresado no existe. Ir al paso 19}\\
		%\hline	
		%\multicolumn{2}{|p{16cm}|}{\textbf{15b. }El sistema informa que el tipo de tarea ingresado no se puede realizar para el tipo de equipo ingresado. Ir al paso 19}\\
		%\hline	
		%\multicolumn{2}{|p{16cm}|}{\textbf{18. }El sistema informa que ya existe una tarifa con el tipo de servicio, tipo de servicio y tipo de servicio ingresados. Ir al paso 19}\\
		%\hline	
	\end{longtable}

    \finCU{}


	\begin{longtable}{ |p{8cm}|p{8cm}| }
		\hline
		\multicolumn{2}{|p{16cm}|}{\textbf{Caso de uso}: Agregar tipo de tarea}\\
		\multicolumn{2}{|p{16cm}|}{\textbf{Tipo}: Primario Esencial}\\
		\multicolumn{2}{|p{16cm}|}{\textbf{Actores}: Jefe de Taller}\\
		\multicolumn{2}{|p{16cm}|}{\textbf{Descripción}: El Jefe de Taller crea un nuevo tipo de tarea, indicando los tipos de equipos para los cuales se podrá realizar, en qué tipo de servicio podrá ser incluído, y creando tarifas con el tipo de tarea}\\
		\multicolumn{2}{|p{16cm}|}{\textbf{Precondiciones}: -}\\
		\multicolumn{2}{|p{16cm}|}{\textbf{Postcondiciones}: Tipo de tarea creado}\\
		\hline
		\multicolumn{2}{|c|}{\textbf{Curso normal de los eventos}}\\
		\hline
		\textbf{Acción de los actores} & \textbf{Respuesta del sistema}\\
		\hline
			\inc Este CU comienza cuando el Jefe de Taller desea agregar un nuevo tipo de tarea & \\
			\hline
            \inc El Jefe de Taller ingresa un nombre y una descripción del tipo de tarea a dar de alta y le solicita al sistema que lo cree & \\
			\hline
			& \inc Crea el tipo de tarea \\
			\hline
			& \inc Muestra los datos del tipo de tarea creado\\
			\hline


			\inc El Jefe de Taller ingresa el nombre de un tipo de equipo para indicar que el tipo de tarea se podrá realizar para ese tipo de equipo & \\
			\hline
			& \inc Busca el tipo de equipo \\
			\hline
            & \inc Vincula el tipo de equipo con el tipo de tarea \\
			\hline
            & \inc Muestra los datos del tipo de tarea\\
			\hline


            \inc Repetir pasos 5 a 8 hasta que el Jefe de Taller no desee indicar más tipos de equipo para el tipo de tarea&\\
			\hline
			\inc El Jefe de Taller ingresa el nombre de un tipo de servicio para indicar que el tipo de tarea se podrá realizar dentro de ese tipo de servicio & \\
			\hline
			& \inc Busca el tipo de servicio \\
			\hline
            & \inc Vincula el tipo de servicio con el tipo de tarea \\
			\hline


            & \inc Muestra los datos del tipo de tarea\\
			\hline
            \inc Repetir pasos 9 a 13 hasta que el Jefe de Taller no desee indicar más tipos de servicio para el tipo de tarea&\\
			\hline
            %creamos tarifas PASO 15
			\inc El Jefe de Taller ingresa el nombre de un tipo de equipo para agregar a una tarifa & \\
			\hline
			& \inc Busca el tipo de equipo \\
			\hline


            & \inc Muestra los datos del tipo de equipo\\
			\hline
            \inc El Jefe de Taller ingresa el nombre de un tipo de servicio para agregar a la tarifa&\\
			\hline
            & \inc Busca el tipo de servicio \\
			\hline
            & \inc Muestra los datos del tipo de servicio\\
			\hline


            \inc El Jefe de Taller ingresa un precio para la tarifa &\\
			\hline
            & \inc Crea la tarifa con el tipo de tarea, tipo de equipo, tipo de serivicio y el precio ingresados\\
			\hline
            & \inc Muestra los datos de la tarifa creada\\
			\hline
            \inc Repetir pasos 15 al 23 hasta que el Jefe de Taller no desee crear más tarifas para el tipo de tarea creado& \\
			\hline


			\inc Fin CU. & \\
		\hline
		\multicolumn{2}{|c|}{\textbf{Cursos alternos}}\\
		\hline
		\multicolumn{2}{|p{16cm}|}{\textbf{4. }El sistema informa que el tipo de tarea ingresado ya existe. Fin CU.}\\
		\hline
		\multicolumn{2}{|p{16cm}|}{\textbf{7. }El sistema informa que el tipo de equipo ingresado no existe. Ir al paso 9}\\
		\hline	
		\multicolumn{2}{|p{16cm}|}{\textbf{8. }El sistema informa que el tipo de equipo y el tipo de tarea ya están vinculados. Ir al paso 9}\\
		\hline	
		\multicolumn{2}{|p{16cm}|}{\textbf{12. }El sistema informa que el tipo de servicio ingresado no existe. Ir al paso 14}\\
		\hline	
		\multicolumn{2}{|p{16cm}|}{\textbf{13. }El sistema informa que el tipo de servicio y el tipo de tarea ya están vinculados. Ir al paso 14}\\
		\hline	
		\multicolumn{2}{|p{16cm}|}{\textbf{18a. }El sistema informa que el tipo de equipo ingresado no existe. Ir al paso 24}\\
		\hline	
		\multicolumn{2}{|p{16cm}|}{\textbf{18b. }El sistema informa que el tipo de tarea no se puede realizar para el tipo de equipo ingresado. Ir al paso 24}\\
		\hline	
		\multicolumn{2}{|p{16cm}|}{\textbf{20a. }El sistema informa que el tipo de servicio ingresado no existe. Ir al paso 24}\\
		\hline	
		\multicolumn{2}{|p{16cm}|}{\textbf{20b. }El sistema informa que el tipo de tarea no se puede realizar bajo el tipo de servicio ingresado. Ir al paso 24}\\
		\hline	
		\multicolumn{2}{|p{16cm}|}{\textbf{23. }El sistema informa que ya existe una tarifa con el tipo de tarea, tipo de equipo y tipo de servicio ingresados. Ir al paso 24}\\
		\hline	
	\end{longtable}

    \finCU{}


	\begin{longtable}{ |p{8cm}|p{8cm}| }
		\hline
		\multicolumn{2}{|p{16cm}|}{\textbf{Caso de uso}: Crear tarifa}\\
		\multicolumn{2}{|p{16cm}|}{\textbf{Tipo}: Primario Esencial}\\
		\multicolumn{2}{|p{16cm}|}{\textbf{Actores}: Jefe de Taller}\\
		\multicolumn{2}{|p{16cm}|}{\textbf{Descripción}: El Jefe de Taller crea una nueva tarifa indicando qué tipo de incluirá, para qué tipo de equipo, bajo qué tipo de servicio, y un precio.}\\
		\multicolumn{2}{|p{16cm}|}{\textbf{Precondiciones}: -}\\
		\multicolumn{2}{|p{16cm}|}{\textbf{Postcondiciones}: Tarifa creada}\\
		\hline
		\multicolumn{2}{|c|}{\textbf{Curso normal de los eventos}}\\
		\hline
		\textbf{Acción de los actores} & \textbf{Respuesta del sistema}\\
		\hline

			\inc Este CU comienza cuando el Jefe de Taller desea crear una tarifa& \\
			\hline
            \inc El Jefe de Taller ingresa el nombre de un tipo de equipo para agregar a la tarifa & \\
			\hline
            & \inc Busca el tipo de equipo \\
			\hline
			& \inc Muestra los datos del tipo de equipo \\
			\hline


			\inc El Jefe de Taller ingresa el nombre un tipo de tarea para agregar a la tarifa & \\
			\hline
			& \inc Busca el tipo de tarea \\
			\hline
			& \inc Muestra los datos del tipo de tarea \\
			\hline
            \inc El Jefe de Taller ingresa el nombre de un tipo de servicio para vincular con la tarifa &\\
			\hline


            & \inc Busca el tipo de servicio \\
			\hline
			& \inc Muestra los datos del tipo de servicio \\
			\hline
            \inc El Jefe de Taller ingresa un precio para la tarifa&\\
			\hline
            & \inc Crea la tarifa con el tipo de tarea, tipo de equipo, tipo de serivicio y el precio ingresados\\
			\hline


            & \inc Muestra los datos de la tarifa creada\\
			\hline
            \inc Repetir pasos 5 al 13 hasta que el Jefe de Taller no desee crear más tarifas para el tipo de equipo ingresado & \\
			\hline
			\inc Fin CU. & \\

        \hline
		\multicolumn{2}{|c|}{\textbf{Cursos alternos}}\\
		\hline
		\multicolumn{2}{|p{16cm}|}{\textbf{4. }El sistema informa que el tipo de equipo ingresado no existe. Fin CU.}\\
		\hline
		\multicolumn{2}{|p{16cm}|}{\textbf{7a. }El sistema informa que el tipo de tarea ingresado no existe. Ir al paso 14}\\
		\hline	
		\multicolumn{2}{|p{16cm}|}{\textbf{7b. }El sistema informa que el tipo de tarea ingresado no se puede realizar para el tipo de equipo ingresado. Ir al paso 14}\\
		\hline	
		\multicolumn{2}{|p{16cm}|}{\textbf{10a. }El sistema informa que el tipo de servicio ingresado no existe. Ir al paso 14}\\
		\hline	
		\multicolumn{2}{|p{16cm}|}{\textbf{10b. }El sistema informa que el tipo de tarea ingresado no se puede realizar bajo el tipo de servicio ingresado. Ir al paso 14}\\
		\hline	
        \multicolumn{2}{|p{16cm}|}{\textbf{13. }El sistema informa que ya existe una tarifa con el tipo de tarea, tipo de equipo, y tipo de servicio ingresados. Ir al paso 14}\\
		\hline	
	\end{longtable}


    \finCU{}


	\begin{longtable}{ |p{8cm}|p{8cm}| }
		\hline
		\multicolumn{2}{|p{16cm}|}{\textbf{Caso de uso}: Modificar tarifa}\\
		\multicolumn{2}{|p{16cm}|}{\textbf{Tipo}: Primario Esencial}\\
		\multicolumn{2}{|p{16cm}|}{\textbf{Actores}: Jefe de Taller}\\
		\multicolumn{2}{|p{16cm}|}{\textbf{Descripción}: El Jefe de Taller ingresa el tipo de equipo, tipo de tarea, y tipo de servicio de una tarifa para modificar su precio.}\\
		\multicolumn{2}{|p{16cm}|}{\textbf{Precondiciones}: -}\\
		\multicolumn{2}{|p{16cm}|}{\textbf{Postcondiciones}: Tarifa modificada}\\
		\hline
		\multicolumn{2}{|c|}{\textbf{Curso normal de los eventos}}\\
		\hline
		\textbf{Acción de los actores} & \textbf{Respuesta del sistema}\\
		\hline

			\inc Este CU comienza cuando el Jefe de Taller desea modificar una tarifa& \\
			\hline
            \inc El Jefe de Taller ingresa el nombre del tipo de equipo de la tarifa & \\
			\hline
            & \inc Busca el tipo de equipo \\
			\hline
			& \inc Muestra los datos del tipo de equipo \\
			\hline


			\inc El Jefe de Taller ingresa el nombre del tipo de tarea de la tarifa & \\
			\hline
			& \inc Busca el tipo de tarea \\
			\hline
			& \inc Muestra los datos del tipo de tarea \\
			\hline
            \inc El Jefe de Taller ingresa el nombre del tipo de servicio de la tarifa &\\
			\hline


            & \inc Busca el tipo de servicio \\
			\hline
			& \inc Muestra los datos del tipo de servicio \\
			\hline
            \inc El Jefe de Taller ingresa un nuevo precio para la tarifa&\\
			\hline
            & \inc Busca la tarifa \\
			\hline


            & \inc Modifica la tarifa \\
			\hline
            \inc Repetir pasos 5 al 13 hasta que el Jefe de Taller no desee modificar más tarifas para el tipo de equipo ingresado & \\
			\hline
			\inc Fin CU. & \\

        \hline
		\multicolumn{2}{|c|}{\textbf{Cursos alternos}}\\
		\hline
		\multicolumn{2}{|p{16cm}|}{\textbf{4. }El sistema informa que el tipo de equipo ingresado no existe. Fin CU.}\\
		\hline
		\multicolumn{2}{|p{16cm}|}{\textbf{7. }El sistema informa que el tipo de tarea ingresado no existe. Ir al paso 14}\\
		\hline	
		\multicolumn{2}{|p{16cm}|}{\textbf{10. }El sistema informa que el tipo de servicio ingresado no existe. Ir al paso 14}\\
		\hline	
		\multicolumn{2}{|p{16cm}|}{\textbf{13. }El sistema informa que la tarifa no existe. Ir al paso 14}\\
		\hline	
	\end{longtable}


    \finCU{}


	\begin{longtable}{ |p{8cm}|p{8cm}| }
		\hline
		\multicolumn{2}{|p{16cm}|}{\textbf{Caso de uso}: Eliminar tarifa}\\
		\multicolumn{2}{|p{16cm}|}{\textbf{Tipo}: Secundario Esencial}\\
		\multicolumn{2}{|p{16cm}|}{\textbf{Actores}: Jefe de Taller}\\
		\multicolumn{2}{|p{16cm}|}{\textbf{Descripción}: El Jefe de Taller ingresa un tipo de equipo, tipo de tarea, y tipo de servicio que constituyan una tarifa. El sistema da de baja la tarifa}\\
		\multicolumn{2}{|p{16cm}|}{\textbf{Precondiciones}: -}\\
		\multicolumn{2}{|p{16cm}|}{\textbf{Postcondiciones}: Tarifa eliminada}\\
		\hline
		\multicolumn{2}{|c|}{\textbf{Curso normal de los eventos}}\\
		\hline
		\textbf{Acción de los actores} & \textbf{Respuesta del sistema}\\
		\hline

			\inc Este CU comienza cuando el Jefe de Taller desea eliminar una tarifa& \\
			\hline
            \inc El Jefe de Taller ingresa el nombre del tipo de equipo de la tarifa & \\
			\hline
            & \inc Busca el tipo de equipo \\
			\hline
			& \inc Muestra los datos del tipo de equipo \\
			\hline


			\inc El Jefe de Taller ingresa el nombre del tipo de tarea de la tarifa & \\
			\hline
			& \inc Busca el tipo de tarea \\
			\hline
			& \inc Muestra los datos del tipo de tarea \\
			\hline
            \inc El Jefe de Taller ingresa el nombre del tipo de servicio de la tarifa &\\
			\hline


            & \inc Busca el tipo de servicio \\
			\hline
			& \inc Muestra los datos del tipo de servicio \\
			\hline
            & \inc Busca la tarifa \\
			\hline
            & \inc Elimina la tarifa \\


			\hline
            \inc Repetir pasos 5 al 12 hasta que el Jefe de Taller no desee eliminar más tarifas para el tipo de equipo ingresado & \\
			\hline
			\inc Fin CU. & \\

        \hline
		\multicolumn{2}{|c|}{\textbf{Cursos alternos}}\\
		\hline
		\multicolumn{2}{|p{16cm}|}{\textbf{4. }El sistema informa que el tipo de equipo ingresado no existe. Fin CU.}\\
		\hline
		\multicolumn{2}{|p{16cm}|}{\textbf{7. }El sistema informa que el tipo de tarea ingresado no existe. Ir al paso 13}\\
		\hline	
		\multicolumn{2}{|p{16cm}|}{\textbf{10. }El sistema informa que el tipo de servicio ingresado no existe. Ir al paso 13}\\
		\hline	
		\multicolumn{2}{|p{16cm}|}{\textbf{12a. }El sistema informa que la tarifa no existe. Ir al paso 13}\\
		\hline	
        \multicolumn{2}{|p{16cm}|}{\textbf{12b. }El sistema informa que la tarifa está presente en un presupuesto de una \OT{} sin finalizar. Ir al paso 13}\\
		\hline	
	\end{longtable}


    \finCU{}


	\begin{longtable}{ |p{8cm}|p{8cm}| }
		\hline
		\multicolumn{2}{|p{16cm}|}{\textbf{Caso de uso}: Asignar Tarea a Técnico}\\
		\multicolumn{2}{|p{16cm}|}{\textbf{Tipo}: Primario Esencial}\\
		\multicolumn{2}{|p{16cm}|}{\textbf{Actores}: Jefe de Taller}\\
		\multicolumn{2}{|p{16cm}|}{\textbf{Descripción}: El Jefe de Taller registra en el sistema que un técnico puede realizar ciertas tareas.}\\
		\multicolumn{2}{|p{16cm}|}{\textbf{Precondiciones}: - }\\
		\multicolumn{2}{|p{16cm}|}{\textbf{Postcondiciones}: Tarea asignada a un Técnico.}\\
		\hline
		\multicolumn{2}{|c|}{\textbf{Curso normal de los eventos}}\\
		\hline
		\textbf{Acción de los actores} & \textbf{Respuesta del sistema}\\
		\hline
			\inc Este Caso de Uso comienza cuando el Jefe de Taller desea agregar una Tarea a un Técnico.& \\
			\hline
			\inc  El Jefe de Taller le solicita al sistema buscar al Técnico ingresando su nombre.& \\
			\hline
			& \inc Busca al Técnico. \\
			\hline
            & \inc Muestra los datos personales del Técnico. \\
			\hline


			\inc El Jefe de Taller le solicita al sistema buscar un Tipo de Tarea ingresando su nombre. & \\
			\hline
			& \inc Busca el Tipo de Tarea. \\
			\hline
			& \inc Muestra los datos del Tipo de Tarea. \\
			\hline
			\inc  El Jefe de Taller le solicita al sistema buscar un Tipo de Equipo ingresando su nombre. & \\
			\hline


			& \inc Busca el Tipo de Equipo.\\
			\hline
			& \inc Muestra los datos del Tipo de Equipo. \\
			\hline
			\inc  El Jefe de Taller le solicita al sistema agregar ese Tipo de Tarea para ese Tipo de Equipo al Técnico. & \\
			\hline
			& \inc Agrega el Tipo de Tarea para el Tipo de Equipo al Técnico. \\
			\hline


			\inc  Repetir pasos 5 a 12 hasta que el Jefe de Taller no desee agregar más tareas al Técnico. & \\
			\hline
			\inc Fin CU. & \\
		\hline
		\multicolumn{2}{|c|}{\textbf{Cursos alternos}}\\
		\hline
		\multicolumn{2}{|p{16cm}|}{\textbf{4. }El Sistema informa que el Técnico no existe. Fin CU.}\\
		\hline
		\multicolumn{2}{|p{16cm}|}{\textbf{7. }El Sistema informa que el Tipo de Tarea no existe. Ir al paso 13}\\
		\hline
		\multicolumn{2}{|p{16cm}|}{\textbf{10a. }El Sistema informa que el Tipo de Equipo no existe. Ir al paso 13}\\
		\hline	
		\multicolumn{2}{|p{16cm}|}{\textbf{10b. }El Sistema informa que el Tipo de Tarea ingresado no se puede realizar para el Tipo de Equipo ingresado. Ir al paso 13.}\\
		\hline	
	\end{longtable}

    \finCU{}


	\begin{longtable}{ |p{8cm}|p{8cm}| }
		\hline
		\multicolumn{2}{|p{16cm}|}{\textbf{Caso de uso}: Remover Tarea de un Técnico}\\
		\multicolumn{2}{|p{16cm}|}{\textbf{Tipo}: Primario Esencial}\\
		\multicolumn{2}{|p{16cm}|}{\textbf{Actores}: Jefe de Taller}\\
		\multicolumn{2}{|p{16cm}|}{\textbf{Descripción}: El Jefe de Taller le solicita al sistema remover uno o más Tipos de Tarea para un Tipo de Equipo de un Técnico en particular.}\\
		\multicolumn{2}{|p{16cm}|}{\textbf{Precondiciones}: - }\\
		\multicolumn{2}{|p{16cm}|}{\textbf{Postcondiciones}: Tipo de Tarea removida de un técnico.}\\
		\hline
		\multicolumn{2}{|c|}{\textbf{Curso normal de los eventos}}\\
		\hline
		\textbf{Acción de los actores} & \textbf{Respuesta del sistema}\\
		\hline
            \inc Este Caso de Uso comienza el Jefe de Taller decide remover de un técnico un Tipo de Tarea de los Tipos de Tarea está capacitado para realizar.& \\
			\hline
			\inc El Jefe de Taller se solicita al sistema buscar un Técnico ingresando su nombre. & \\
			\hline
			& \inc Busca el Técnico.\\
			\hline
			& \inc Muestra los datos personales del Técnico.\\
			\hline


			\inc El Jefe de Taller le solicita al sistema buscar un Tipo de Tarea ingresando su nombre & \\
			\hline
			& \inc Busca el Tipo de Tarea. \\
			\hline
			& \inc Muestra los datos del Tipo de Tarea. \\
			\hline
			\inc El Jefe de Taller le solicita al sistema buscar un Tipo de Equipo, ingresando su nombre.& \\
			\hline


			& \inc Busca el Tipo de Equipo.\\
			\hline
			& \inc Muestra los datos del Tipo de Equipo.\\
			\hline
			\inc El Jefe de Taller le solicita al sistema que remueva del Técnico el Tipo de Tarea para el Tipo de Equipo ingresados. & \\
			\hline
			& \inc Elimina el Tipo de Tarea para el Tipo de Equipo de las Tareas que puede realizar el Técnico.\\
			\hline


			& \inc Muestra los datos del técnico.\\
			\hline
			\inc Repetir pasos 5 a 13 hasta que el Jefe de Taller no desee remover más Tipos de Tareas asignados a un Técnico.&\\
			\hline
			\inc Fin CU.& \\
		\hline
		\multicolumn{2}{|c|}{\textbf{Cursos alternos}}\\
		\hline
		\multicolumn{2}{|p{16cm}|}{\textbf{4. }El sistema informa que el técnico no existe. Fin CU.}\\
		\hline
		\multicolumn{2}{|p{16cm}|}{\textbf{7. }El sistema informa que el Tipo de Tarea ingresado no existe. Ir al paso 14.}\\
		\hline
		\multicolumn{2}{|p{16cm}|}{\textbf{10a. }El sistema informa que el Tipo de Equipo ingresado no existe. Ir al paso 14. }\\
		\hline
		\multicolumn{2}{|p{16cm}|}{\textbf{10b. }El sistema informa que el Tipo de Tarea ingresado no se puede realizar para el Tipo de Equipo ingresado. Ir al paso 14. }\\
		\hline
		\multicolumn{2}{|p{16cm}|}{\textbf{13. }El sistema informa que el técnico no está capacitado para realizar el Tipo de Tarea ingresado para el Tipo de Equipo ingresado, por lo tanto no puede realizar la eliminación. Ir al paso 14.}\\
		\hline
	\end{longtable}


    \finCU{}



%\begin{longtable}{ |p{8cm}|p{8cm}| }
%    \hline
%    \multicolumn{2}{|p{16cm}|}{\textbf{Caso de uso}: Registrar Pago diferido de Factura}\\
%    \multicolumn{2}{|p{16cm}|}{\textbf{Tipo}: Primario}\\
%    \multicolumn{2}{|p{16cm}|}{\textbf{Actores}: Operario Contable, Cliente}\\
%    \multicolumn{2}{|p{16cm}|}{\textbf{Descripción}: El cliente realiza el pago de una factura}\\
%    \multicolumn{2}{|p{16cm}|}{\textbf{Precondiciones}: -}\\
%    \multicolumn{2}{|p{16cm}|}{\textbf{Postcondiciones}: Una o más facturas pagas}\\
%    \hline
%    \multicolumn{2}{|c|}{\textbf{Curso normal de los eventos}}\\
%    \hline
%    \textbf{Acción de los actores} & \textbf{Respuesta del sistema}\\
%    \hline
%        \inc Este CU comienza cuando el cliente se acerca al local para pagar una factura.& \\
%        \hline
%        \inc El Operario Contable le pregunta al cliente qué factura desea pagar. El cliente le contesta el número de factura. & \\
%        \hline
%        \inc El Operario Contable le solicita al sistema que busque la factura. & \\
%        \hline
%        & \inc El sistema busca la factura. \\
%        \hline
%        & \inc El sistema muestra los datos de la factura. \\
%        \hline
%        & \inc El sistema busca el cliente que figura en la factura.\\
%        \hline
%        \inc El Operario le informa al cliente el importe indicado en el total de la factura. El cliente decide:
%            \begin{enumerate}[label=(\alph*)]
%                \item Pagar en efectivo: ir a sección \textit{Pago en Efectivo}.
%                \item Pagar con Nota de Crédito: ir a sección \textit{Pago con Nota de Crédito}.
%                \item Pagar con Transferencia Bancaria: ir a sección \textit{Pago con Transferencia Bancaria}.
%                \item Pagar con Cheque: ir a sección \textit{Pago con Cheque}.
%                \item No paga. Ir al paso 8.
%            \end{enumerate}
%        & \\
%        \hline
%        & \inc El sistema muestra el subtotal de lo pagado y el total de lo adeudado.\\
%        \hline
%        & \inc Repetir pasos 5 al 6 hasta que se registre el cobro total del importe a pagar.\\
%        \hline
%        \inc El Operario Contable registra el pago realizado en la Cuenta Corriente del cliente de la factura.&\\
%        \hline
%        & \inc El sistema calcula y almacena el saldo actual del cliente que figura en la factura.\\
%        \hline
%        \inc El Operario Contable le extiende al cliente un comprobante de pago & \\
%        \hline
%        & \inc Si la factura fue pagada en su totalidad, el sistema la marca como pagada. Si no, el sistema la marca como pagada parcialmente.\\
%        \hline
%        \inc Fin CU. & \\
%    \hline
%    \multicolumn{2}{|c|}{\textbf{Cursos alternos}}\\
%    \hline
%    \multicolumn{2}{|p{16cm}|}{\textbf{5a. } El sistema informa que la factura no existe, fin CU.}\\
%    \hline	
%    \multicolumn{2}{|p{16cm}|}{\textbf{5b. } El sistema informa que la factura ya fue pagada, fin CU.}\\
%    \hline	
%    \multicolumn{2}{|p{16cm}|}{\textbf{9. } El sistema le informa al Operario Contable que el cliente dispone de saldo a favor en su cuenta corriente. El Operario Contable genera una Nota de Crédito a nombre del Cliente con dicho saldo. Se la entrega al Cliente. Continuar con el paso 14 del Curso Normal.}\\
%    \hline	
%\end{longtable}
%
%%LO RESETEAMOS A 0
%\resetinc
%
%\begin{longtable}{ |p{8cm}|p{8cm}| }
%    \hline
%    
%    \multicolumn{2}{|p{16cm}|}{\textbf{Sección}: Pago en efectivo}\\
%    \hline
%    \multicolumn{2}{|c|}{\textbf{Curso normal de los eventos}}\\
%    \hline
%    \textbf{Acción de los actores} & \textbf{Respuesta del sistema}\\
%    \hline
%        \inc  El cliente da un pago en efectivo& \\
%        \hline
%        \inc  El Operario Contable le solicita al sistema el registro del cobro del importe a pagar ingresando el monto del efectivo entregado por el cliente& \\
%        \hline
%        & \inc  El sistema registra el pago en efectivo\\
%        \hline
%        & \inc  El sistema calcula y muestra el vuelto\\
%        \hline
%        \inc  El Operario Contable deposita el efectivo recibido, extrae la diferencia y se la entrega al cliente (de ser necesario)&\\
%        \hline
%        \inc Fin Sección. & \\
%    \hline
%    \multicolumn{2}{|c|}{\textbf{Cursos alternos}}\\
%    \hline
%    \multicolumn{2}{|p{16cm}|}{\textbf{1. } El cliente no tiene suficiente efectivo como para pagar el importe. Ir a paso 5 del curso principal de CU.}\\
%    \hline
%\end{longtable}
%
%%LO RESETEAMOS A 0
%\resetinc
%
%\begin{longtable}{ |p{8cm}|p{8cm}| }
%    \hline
%    \multicolumn{2}{|p{16cm}|}{\textbf{Sección}: Pago con Nota de Crédito}\\
%    \hline
%    \multicolumn{2}{|c|}{\textbf{Curso normal de los eventos}}\\
%    \hline
%    \textbf{Acción de los actores} & \textbf{Respuesta del sistema}\\
%    \hline
%        \inc  El cliente le entrega al Operario Contable la Nota de Crédito con la que desea abonar.& \\
%        \hline
%        \inc  El Operario Contable contable busca en el sistema la Nota de Crédito, ingresando su número.& \\
%        \hline
%        & \inc  El sistema busca la Nota de Crédito.\\
%        \hline
%        & \inc  El sistema muestra los datos de la Nota de Crédito.\\
%        \hline
%        \inc  El Operario Contable registra en el sistema el uso de la Nota de Credito como cobro del importe a pagar.&\\
%        \hline
%        & \inc El Sistema registra el pago con el monto de la Nota de Crédito.\\
%        \hline
%        & \inc El sistema marca la nota de crédito como ya usada.\\
%        \hline
%        \inc Fin Sección.&\\
%    \hline
%    \multicolumn{2}{|c|}{\textbf{Cursos alternos}}\\
%    \hline
%    \multicolumn{2}{|p{16cm}|}{\textbf{4a. } La Nota de Crédito no existe. El Operario se lo informa al cliente y le da la posibilidad de elegir otra forma de pago, el cliente decide:
%        
%        \begin{itemize}
%            \item Si desea usar otra Nota de Crédito, ir a paso 1.
%            \item Si desea utilizar otra forma de pago, ir a paso 5 del curso principal de CU.
%        \end{itemize}}\\
%    \hline
%    \multicolumn{2}{|p{16cm}|}{\textbf{4b. } La Nota de Crédito ya fue usada. El Operario se lo informa al cliente y le da la posibilidad de elegir otra forma de pago, el cliente decide:
%    
%        \begin{itemize}
%            \item Si desea usar otra Nota de Crédito, ir a paso 1.
%            \item Si desea utilizar otra forma de pago, ir a paso 5 del curso principal de CU.
%        \end{itemize}}\\
%    \hline
%\end{longtable}
%
%%LO RESETEAMOS A 0
%\resetinc
%
%\begin{longtable}{ |p{8cm}|p{8cm}| }
%    \hline
%    \multicolumn{2}{|p{16cm}|}{\textbf{Sección}: Pago con Cheque}\\
%    \hline
%    \multicolumn{2}{|c|}{\textbf{Curso normal de los eventos}}\\
%    \hline
%    \textbf{Acción de los actores} & \textbf{Respuesta del sistema}\\
%        \hline
%        \inc  El Cliente le entrega al Operario Contable el cheque con el que desea abonar & \\
%        \hline
%        \inc  El Operario Contable solicita al sistema el registro del cobro del importe a pagar con el monto que figura en el cheque &  \\
%        \hline
%        & \inc El sistema registra el pago con monto del cheque. \\
%        \hline
%        \inc Fin Sección. & \\
%        \hline
%    \multicolumn{2}{|c|}{\textbf{Cursos alternos}}\\
%    \hline
%\end{longtable}
%
%%LO RESETEAMOS A 0
%\resetinc
%
%\begin{longtable}{ |p{8cm}|p{8cm}| }
%    \hline
%    \multicolumn{2}{|p{16cm}|}{\textbf{Sección}: Pago con Transferencia Bancaria}\\
%    \hline
%    \multicolumn{2}{|c|}{\textbf{Curso normal de los eventos}}\\
%    \hline
%    \textbf{Acción de los actores} & \textbf{Respuesta del sistema}\\
%        \hline
%        \inc El cliente le entrega el comprobante de la Transferencia Bancaria al Operario Contable& \\
%        \hline
%        \inc El Operario contable busca en el registro de la organización la transferencia &\\
%        \hline
%        \inc El Operario Contable confirma la acreditación de la transferencia & \\
%        \hline
%        \inc El Operario Contable solicita al sistema que registre el cobro del importe a pagar con el monto que figura en la Transferencia& \\
%        \hline
%        & \inc  El sistema registra el pago con monto de la transferencia\\
%        \hline
%        \inc Fin Sección &\\
%        \hline
%    \multicolumn{2}{|c|}{\textbf{Cursos alternos}}\\
%    \hline
%    \multicolumn{2}{|p{16cm}|}{\textbf{3. } La transferencia no se ha acreditado. El Operario Contable le infoma la situación al Cliente y le da la posibilidad de elegir otra forma de pago. El cliente decide:
%    \begin{enumerate}[label=(\alph*)]
%        \item Indicar otra transferencia. Ir a paso 1.
%        \item Elegir otra forma de pago. Ir al paso 5 del Curso Normal de CU. 
%    \end{enumerate}}\\
%    \hline
%\end{longtable}





\end{spacing}

\pagebreak
\section{Anexo 1}

    \begin{figure}[h]
    \includegraphics[scale=0.3]{images/atinformatica-facturaB.jpg}
    \caption{Factura}
    \end{figure}

    \begin{figure}[h]
    \includegraphics[scale=0.5]{images/atinformatica-orden_de_trabajo.jpg}
    \caption{Orden de trabajo}
    \end{figure}

    \begin{figure}[h]
    \includegraphics[scale=0.4]{images/atinformatica-recibo.jpg}
    \caption{Recibo}
    \end{figure}

    \begin{figure}[h]
    \includegraphics[scale=0.4]{images/atinformatica-remito.jpg}
    \caption{Remito}
    \end{figure}

    \begin{figure}[h]
    \includegraphics[scale=0.8]{images/atinformatica-remito1.jpg}
    \caption{Remito (proveedor)}
    \end{figure}

    %\begin{figure}[h]
    %\includegraphics[scale=0.6]{images/atinformatica-remito2.jpg}
    %\caption{Remito (proveedor)}
    %\end{figure}

    \begin{figure}[h]
    \includegraphics[scale=0.275]{images/atinformatica-sticker.jpg}
    \caption{Sticker}
    \end{figure}

   %\begin{figure}[h]
   %\includegraphics[scale=0.5]{images/atinformatica-tarifario-pixel.jpg}
   %\caption{Tarifario}
   %\end{figure}

    \begin{figure}[h]
    \includegraphics[scale=0.5]{images/cliente_ingreso.jpg}
    \caption{Factura}
    \end{figure}

    \begin{figure}[h]
    \includegraphics[scale=0.5]{images/equipo_e_ingreso.jpg}
    \caption{Listado de equipos e ingreso}
    \end{figure}

    \begin{figure}[h]
    \includegraphics[scale=0.5]{images/estados_orden.jpg}
    \caption{Estados de una orden}
    \end{figure}

    \begin{figure}[h]
    \includegraphics[scale=0.5]{images/listado_clientes.jpg}
    \caption{Listado de clientes}
    \end{figure}

    \begin{figure}[h]
    \includegraphics[scale=0.5]{images/listado_ordenes.jpg}
    \caption{Listado de órdenes}
    \end{figure}

    \begin{figure}[h]
    \includegraphics[scale=0.5]{images/orden1.png}
    \caption{Orden de trabajo}
    \end{figure}

\clearpage

\section{Anexo 2 - Estados de las Órdenes de Trabajo}

    \begin{enumerate}
        \item \textbf{Asignada}: estado inicial de la orden. Cuenta con un técnico asociado.
        \item \textbf{Revisada}: el técnico asociado realizó un primer diagnóstico del equipo ingresado (se lo anexa a la OT; incluye el problema que describió el cliente y lo que descubrió el técnico).
        \item \textbf{Cotizada}: equipo ya diagnosticado. Presupuesto elaborado.
        \item \textbf{Notificada de cotización}: el cliente ya fue notificado del presupuesto. En caso de que el cliente no acepte, la orden pasa al estado \textbf{No se repara} o bien a \textbf{Fallida} (en caso de tratarse de una visita). 
        \item \textbf{Presupuesto aceptado}: el cliente aceptó el presupuesto indicado para realizar el trabajo.
        \item \textbf{Espera de repuestos}: se pidieron repuestos necesarios para realizar el trabajo. No se puede reanudar hasta su arribo.
        \item \textbf{En trámite}: se está realizando el trabajo; durante esta etapa, varios técnicos pueden participar en el trabajo y realizar observaciones en la orden. En caso de que el trabajo requiera algún Tipo servicio más, la OT vuelve a pasar por los estados \textbf{Cotizada}, \textbf{Notificada de cotización}, y en caso de aceptarse el presupuesto, \textbf{Presupuesto aceptado}. En caso de no aceptarse, se pasa a \textbf{No se repara} o a \textbf{Fallida} (si es una visita).
        \item \textbf{Pendiente de facturación}: trabajo terminado. 
        \item \textbf{Notificada de reparación}: se le avisa al cliente que su equipo está listo para ser retirado.
        \item \textbf{Cumplido}: se realizó la visita exitosamente. La orden se encuentra pendiente de facturación.
        \item \textbf{No se repara}: no se completó la reparación porque el cliente no aceptó el presupuesto.
        \item \textbf{Fallida}: no se pudo completar el trabajo on-site.
        \item \textbf{Entregada}: se entregó el equipo reparado al cliente. El trabajo fue facturado.
        \item \textbf{SCRAP}: el cliente no retiró el equipo pasados los 60 (sesenta) días hábiles. Se encuentra a disposición de la organización.
    \end{enumerate}

\end{document}
