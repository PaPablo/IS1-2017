\documentclass[12pt]{extarticle}
\usepackage[utf8]{inputenc}
\usepackage{caption}
\usepackage{subcaption}
\usepackage[spanish]{babel}
\usepackage{multicol}
\usepackage{fancyhdr}
\usepackage{longtable}
\usepackage{enumitem}

\usepackage{url}
\usepackage[a4paper]{geometry}
\usepackage{float}
\usepackage{setspace}
\usepackage{color}   %May be necessary if you want to color links
\usepackage{hyperref}
\hypersetup{
    colorlinks=true, %set true if you want colored links
    linktoc=all,     %set to all if you want both sections and subsections linked
    linkcolor=blue,  %choose some color if you want links to stand out
}
\usepackage{graphicx}
\graphicspath{ {images/} }


\begin{document}

    \subsection{Casos de Uso expandidos}

    %DEFINIMOS UN CONTADOR
    \newcounter{step}
    \newcommand\inc{\stepcounter{step}\textbf{\thestep. }}
    \newcommand\resetinc{\setcounter{step}{0}}
    
    \newcommand\raya{\noindent\rule{169mm}{0.8mm}\\}

\begin{longtable}{ |p{8cm}|p{8cm}| }
    \hline
    \multicolumn{2}{|p{16cm}|}{\textbf{Caso de uso}: Atender solicitud de servicio}\\
    \multicolumn{2}{|p{16cm}|}{\textbf{Tipo}: Primario Esencial}\\
    \multicolumn{2}{|p{16cm}|}{\textbf{Actores}: Técnico, Cliente}\\
    \multicolumn{2}{|p{16cm}|}{\textbf{Descripción}: Un Cliente se comunica con la empresa para solicitar un servicio de reparación. El Técnico crea una \OT{} para atender las solicitud.}\\
    \multicolumn{2}{|p{16cm}|}{\textbf{Precondiciones}: - }\\
    \multicolumn{2}{|p{16cm}|}{\textbf{Postcondiciones}: \OT{} creada\\
    \hline
    \multicolumn{2}{|c|}{\textbf{Curso normal de los eventos}}\\
    \hline
    \textbf{Acción de los actores} & \textbf{Respuesta del sistema}\\
    \hline
        \inc Este CU comienza cuando un Cliente se contacta con la empresa para requerir un servicio.&\\
        \hline
        \inc El Técnico le solicita el nombre al cliente y este se lo da. El técnico le solicita al Sistema buscar al cliente ingresando el nombre.& \\
        \hline
        & \inc Busca el cliente.\\
        \hline
        & \inc Verifica que el cliente no sea moroso.\\
        \hline

        & \inc Muestra los datos personales del Cliente.\\
        \hline
        \inc El Técnico le solicita al Sistema crear una \OT{} a nombre del cliente y se designa a sí mismo como encargado.& \\
        \hline
        & \inc Crea la \OT{}.\\
        \hline
        \inc El técnico ingresa el número de serie del equipo para incorporarlo a la \OT{}.& \\
        \hline
        
        & \inc Busca el equipo. \\
        \hline
        & \inc Incorpora el equipo a la \OT{}.\\
        \hline
        \inc El Técnico le solicita al cliente una descripción de problema del equipo. El Cliente se la da. El técnico ingresa en el sistema la descripción y le solicita al Sistema que la incorpore a la \OT{}& \\
        \hline
        & \inc Incorpora la descripción del problema a la \OT{}\\
        \hline

        \inc El técnico le solicita el Sistema calcular el monto de RDyP para el tipo de equipo del equipo ingresado.& \\
        \hline
        & \inc Calcula el importe.\\
        \hline
        & \inc Muestra el importe.\\
        \hline
        \inc El Técnico le comunica al Cliente el monto a pagar por la RDyP.&\\
        \hline

        \inc El Cliente abona en efectivo el importe de RDyP.& \\
        \hline
        \inc El Técnico le solicita al sistema el registro del cobro del importe a pagar ingresando el monto del efectivo entregado por el cliente& \\
        \hline
        & \inc Registra el pago del importe.\\
        \hline
        & \inc Calcula y muestra el vuelto.\\
        \hline

        \inc El Técnico deposita el efectivo recibido, extrae la diferencia y se la entrega al cliente (de ser necesario)&\\
        \hline
        \inc El Técnico le solicita al Sistema imprimir la factura por el importe abonado y el comprobante de la \OT{}& \\
        \hline
        & \inc Imprime la factura y el comprobante de la \OT{}.\\
        \hline
        \inc El Técnico le entrega al Cliente la factura junto con el comprobante de la \OT{}.&\\

        \hline
        \inc Fin CU. & \\
    \hline
    \multicolumn{2}{|c|}{\textbf{Cursos alternos}}\\
    \hline
    \multicolumn{2}{|p{16cm}|}{\textbf{4. } El Sistema informa que el Cliente no existe. Fin CU.}\\
    \hline
    \multicolumn{2}{|p{16cm}|}{\textbf{5. } El Sistema informa que el cliente es moroso. El Técnico consulta con el gerente si prestarle o no servicio al Cliente. Si se decide que sí, ir a paso x. Si no, fin CU.}\\
    \hline
    \multicolumn{2}{|p{16cm}|}{\textbf{8. } El Técnico ve que es la primera vez que ingresa el equipo al taller, includes "Agregar Equipo". Ir a paso x.}\\
    \hline	
    \multicolumn{2}{|p{16cm}|}{\textbf{13. } El Sistema informa que el Tipo de Equipo del Equipo no existe. El Técnico consulta un valor de RDyP con el Jefe de Taller para ese Tipo de Equipo. El Jefe de Taller se lo comunica al técnico. El Técnico lo detalla en la \OT{} y se lo comunica al Cliente. Ir a paso x.}\\
    \hline	
    \multicolumn{2}{|p{16cm}|}{\textbf{17. } El Cliente decide no abonar el importe de RDyP. Fin CU.}\\
    \hline	
\end{longtable}

\end{document}
