\documentclass[12pt]{extarticle}
\usepackage[utf8]{inputenc}
\usepackage{caption}
\usepackage{subcaption}
\usepackage[spanish]{babel}
\usepackage{multicol}
\usepackage{listings}
\usepackage{fancyhdr}

\usepackage{longtable}
\usepackage{enumitem}

\usepackage{url}
\usepackage[a4paper]{geometry}
\usepackage{float}
\usepackage{setspace}
\usepackage{color}   %May be necessary if you want to color links
\usepackage{hyperref}
\hypersetup{
    colorlinks=true, %set true if you want colored links
    linktoc=all,     %set to all if you want both sections and subsections linked
    linkcolor=blue,  %choose some color if you want links to stand out
}
\usepackage{graphicx}
\graphicspath{ {images/} }


\begin{document}
%/******************************************************/
	%/*PLANTILLA PARA CU EXPANDIDO**************************/
	%/*IR ELIMINANDO LOS NÚMERO DE PASO QUE NO CORRESPONDAN*/
	%/******************************************************/
	\begin{longtable}{ |p{8cm}|p{8cm}| }
		\hline
		\multicolumn{2}{|p{16cm}|}{\textbf{Caso de uso}: Registrar Pago de Factura}\\
		\multicolumn{2}{|p{16cm}|}{\textbf{Tipo}: Primario}\\
		\multicolumn{2}{|p{16cm}|}{\textbf{Actores}: Operario Contable, Cliente}\\
		\multicolumn{2}{|p{16cm}|}{\textbf{Descripción}: El cliente realiza el pago de una factura}\\
		\multicolumn{2}{|p{16cm}|}{\textbf{Precondiciones}: -}\\
		\multicolumn{2}{|p{16cm}|}{\textbf{Postcondiciones}: Una o más facturas pagas}\\
		\hline
		\multicolumn{2}{|c|}{\textbf{Curso normal de los eventos}}\\
		\hline
		\textbf{Acción de los actores} & \textbf{Respuesta del sistema}\\
		\hline
			\inc Este CU comienza cuando el cliente se acerca al local para pagar una factura.& \\
            \hline
			\inc El Operario Contable le pregunta al cliente qué factura desea pagar. El cliente le contesta cuál. & \\
            \hline
			\inc El Operario Contable le solicita al sistema que busque la factura. & \\
            \hline
			& \inc El sistema busca la factura. \\
            \hline
			& \inc El sistema muestra los datos de la factura. \\
            \hline
			\inc El Operario le informa al cliente el importe indicado en el total de la factura. El cliente decide:
                \begin{enumerate}[label=(\alph*)]
                    \item Pagar en efectivo: ir a sección \textit{Pago en Efectivo}.
                    \item Pagar con Nota de Crédito: ir a sección \textit{Pago con Nota de Crédito}.
                    \item Pagar con Transferencia Bancaria: ir a sección \textit{Pago con Transferencia Bancaria}.
                    \item Pagar con Cheque: ir a sección \textit{Pago con Cheque}.
                    \item No paga. Ir al paso 8.
                \end{enumerate}
            & \\
            \hline
            & \inc El sistema muestra el subtotal de lo pagado y el total de lo adeudado.\\
            \hline
            & \inc Repetir pasos 5 al 6 hasta que se registre el cobro total del importe a pagar.\\
            \hline
			\inc El Operario Contable solicita al sistema buscar el Cliente que figura en la factura.& \\
            \hline
            & \inc El sistema busca el cliente.\\
            \hline
            & \inc El sistema muestra los datos del cliente.\\
            \hline
            \inc El Operario Contable registra el pago realizado en la Cuenta Corriente del cliente de la factura.&\\
            \hline
            & \inc El sistema calcula y almacena el saldo actual del cliente que figura en la factura.\\
            \hline
			\inc El Operario Contable le extiende al cliente un comprobante de pago & \\
            \hline
            & \inc Si la factura fue pagada en su totalidad, el sistema la marca como pagada. Si no, el sistema la marca como pagada parcialmente.\\
            \hline
			\inc Fin CU. & \\
		\hline
		\multicolumn{2}{|c|}{\textbf{Cursos alternos}}\\
		\hline
		\multicolumn{2}{|p{16cm}|}{\textbf{4a. } El sistema informa que la factura no existe, fin CU.}\\
		\hline	
        \multicolumn{2}{|p{16cm}|}{\textbf{4b. } El sistema informa que la factura ya fue pagada, fin CU.}\\
		\hline	
        \multicolumn{2}{|p{16cm}|}{\textbf{12. } El sistema le informa al Operario Contable que el cliente dispone de saldo a favor en su cuenta corriente. El Operario Contable genera una Nota de Crédito a nombre del Cliente con dicho saldo. Se la entrega al Cliente. Continuar con el Curso Normal.}\\
		\hline	
	\end{longtable}


    \begin{longtable}{ |p{8cm}|p{8cm}| }
        \hline
        
        \multicolumn{2}{|p{16cm}|}{\textbf{Sección}: Pago en efectivo}\\
        \hline
        \multicolumn{2}{|c|}{\textbf{Curso normal de los eventos}}\\
        \hline
        \textbf{Acción de los actores} & \textbf{Respuesta del sistema}\\
        \hline
            \inc  El cliente da un pago en efectivo& \\
            \hline
            \inc  El Operario Contable le solicita al sistema el registro del cobro del importe a pagar ingresando el monto del efectivo entregado por el cliente& \\
            \hline
            & \inc  El sistema registra el pago en efectivo\\
            \hline
            & \inc  El sistema calcula y muestra el vuelto\\
            \hline
            \inc  El Operario Contable deposita el efectivo recibido, extrae la diferencia y se la entrega al cliente (de ser necesario)&\\
            \hline
            \inc Fin Sección. & \\
        \hline
        \multicolumn{2}{|c|}{\textbf{Cursos alternos}}\\
        \hline
        \multicolumn{2}{|p{16cm}|}{\textbf{1. } El cliente no tiene suficiente efectivo como para pagar el importe. Ir a paso 5 del curso principal de CU.}\\
        \hline
    \end{longtable}
    
    \begin{longtable}{ |p{8cm}|p{8cm}| }
        \hline
        \multicolumn{2}{|p{16cm}|}{\textbf{Sección}: Pago con Nota de Crédito}\\
        \hline
        \multicolumn{2}{|c|}{\textbf{Curso normal de los eventos}}\\
        \hline
        \textbf{Acción de los actores} & \textbf{Respuesta del sistema}\\
        \hline
            \inc  El cliente le entrega al Operario Contable la Nota de Crédito con la que desea abonar.& \\
            \hline
            \inc  El Operario Contable contable busca en el sistema la Nota de Crédito, ingresando su número.& \\
            \hline
            & \inc  El sistema busca la Nota de Crédito.\\
            \hline
            & \inc  El sistema muestra los datos de la Nota de Crédito.\\
            \hline
            \inc  El Operario Contable registra en el sistema el uso de la Nota de Credito como cobro del importe a pagar.&\\
            \hline
            & \inc El Sistema registra el pago con el monto de la Nota de Crédito.\\
            \hline
            & \inc El sistema marca la nota de crédito como ya usada.\\
            \hline
            \inc Fin Sección.&\\
        \hline
        \multicolumn{2}{|c|}{\textbf{Cursos alternos}}\\
        \hline
        \multicolumn{2}{|p{16cm}|}{\textbf{4a. } La Nota de Crédito no existe. El Operario se lo informa al cliente y le da la posibilidad de elegir otra forma de pago, el cliente decide:
            
            \begin{itemize}
                \item Si desea usar otra Nota de Crédito, ir a paso 1.
                \item Si desea utilizar otra forma de pago, ir a paso 5 del curso principal de CU.
            \end{itemize}}\\
        \hline
        \multicolumn{2}{|p{16cm}|}{\textbf{4b. } La Nota de Crédito ya fue usada. El Operario se lo informa al cliente y le da la posibilidad de elegir otra forma de pago, el cliente decide:
        
            \begin{itemize}
                \item Si desea usar otra Nota de Crédito, ir a paso 1.
                \item Si desea utilizar otra forma de pago, ir a paso 5 del curso principal de CU.
            \end{itemize}}\\
        \hline
    \end{longtable}
    \begin{longtable}{ |p{8cm}|p{8cm}| }
        \hline
        \multicolumn{2}{|p{16cm}|}{\textbf{Sección}: Pago con Cheque}\\
        \hline
        \multicolumn{2}{|c|}{\textbf{Curso normal de los eventos}}\\
        \hline
        \textbf{Acción de los actores} & \textbf{Respuesta del sistema}\\
            \hline
            \inc  El Cliente le entrega al Operario Contable el cheque con el que desea abonar & \\
            \hline
            \inc  El Operario Contable solicita al sistema el registro del cobro del importe a pagar con el monto que figura en el cheque &  \\
            \hline
            & \inc El sistema registra el pago con monto del cheque. \\
            \hline
            \inc Fin Sección. & \\
            \hline
        \multicolumn{2}{|c|}{\textbf{Cursos alternos}}\\
        \hline
    \end{longtable}
    \begin{longtable}{ |p{8cm}|p{8cm}| }
        \hline
        \multicolumn{2}{|p{16cm}|}{\textbf{Sección}: Pago con Transferencia Bancaria}\\
        \hline
        \multicolumn{2}{|c|}{\textbf{Curso normal de los eventos}}\\
        \hline
        \textbf{Acción de los actores} & \textbf{Respuesta del sistema}\\
            \hline
            \inc El cliente le entrega el comprobante de la Transferencia Bancaria al Operario Contable& \\
            \hline
            \inc El Operario contable busca en el registro de la organización la transferencia &\\
            \hline
            \inc El Operario Contable confirma la acreditación de la transferencia & \\
            \hline
            \inc El Operario Contable solicita al sistema que registre el cobro del importe a pagar con el monto que figura en la Transferencia& \\
            \hline
            & \inc  El sistema registra el pago con monto de la transferencia\\
            \hline
            \inc Fin Sección &\\
            \hline
        \multicolumn{2}{|c|}{\textbf{Cursos alternos}}\\
        \hline
        \multicolumn{2}{|p{16cm}|}{\textbf{3. } La transferencia no se ha acreditado. El Operario Contable le infoma la situación al Cliente y le da la posibilidad de elegir otra forma de pago. El cliente decide:
        \begin{enumerate}[label=(\alph*)]
            \item Indicar otra transferencia. Ir a paso 1.
            \item Elegir otra forma de pago. Ir al paso 5 del Curso Normal de CU. 
        \end{enumerate}}\\
        \hline
    \end{longtable}
\end{document}
