
	%DEFINIMOS UN CONTADOR
    %\newcounter{step}
    %\newcommand\inc{\stepcounter{step}\textbf{\thestep. }}


%/******************************************************/
	%/*PLANTILLA PARA CU EXPANDIDO**************************/
	%/*IR ELIMINANDO LOS \inc QUE NO CORRESPONDAN*/
    %/*para copiar la tabla, hay que copiar desde el
    %\begin{longtable} hasta el \setcounter que esta al final*/
	%/*el \inc es el contador que está definido en el informe
	%/******************************************************/
	\begin{longtable}{ |p{8cm}|p{8cm}| }
		\hline
		\multicolumn{2}{|p{16cm}|}{\textbf{Caso de uso}: Confirmar Presupuesto}\\
		\multicolumn{2}{|p{16cm}|}{\textbf{Tipo}: Primario Esencial}\\
		\multicolumn{2}{|p{16cm}|}{\textbf{Actores}: Técnico, Cliente}\\
        \multicolumn{2}{|p{16cm}|}{\textbf{Descripción}: El Técnico le comunica al Cliente el Presupuesto de su \OT{} y el Cliente confirma las tareas que se realizarán.}\\
		\multicolumn{2}{|p{16cm}|}{\textbf{Precondiciones}: - }\\
        \multicolumn{2}{|p{16cm}|}{\textbf{Postcondiciones}: Presupuesto confirmado y Tareas creadas en la \OT{}}\\
		\hline
		\multicolumn{2}{|c|}{\textbf{Curso normal de los eventos}}\\
		\hline
		\textbf{Acción de los actores} & \textbf{Respuesta del sistema}\\
		\hline
            \inc Este CU comienza cuando el Técnico se comunica con un Cliente para confirmar el presupuesto de su \OT{}& \\
			\hline
            \inc El Técnico ingresa el número de \OT{} y le solicita al Sistema que busque su presupuesto.& \\
			\hline
            & \inc Busca la \OT{}.\\
			\hline
			& \inc Busca el presupuesto.\\
			\hline


			& \inc Muestra los datos del presupuesto. \\
			\hline
			\inc El Técnico le informa al Cliente una tarea a realizar, los repuestos necesarios, y su precio.& \\
			\hline
			\inc El Cliente confirma la tarea&  \\
			\hline
            \inc El Técnico confirma el detalle del presupuesto & \\
			\hline


            & \inc Crea un detalle en la \OT{} vinculándolo con el detalle del presupuesto. \\
			\hline
            & \inc Asigna el técnico encargado de la \OT{} como técnico del detalle de \OT{}. \\
			\hline
            & \inc Repetir pasos 6 al 10 hasta que no haya más detalles de presupuestos por confirmar\\
			\hline
            & \inc Muestra datos de la \OT{}\\
			\hline


			\inc Fin CU. & \\
		\hline
		\multicolumn{2}{|c|}{\textbf{Cursos alternos}}\\
		\hline
        \multicolumn{2}{|p{16cm}|}{\textbf{4. }El sistema informa que la \OT{} no existe. Fin CU.}\\
		\hline
        \multicolumn{2}{|p{16cm}|}{\textbf{5a. }El sistema informa que la \OT{} no fue presupuestada. Fin CU.}\\
		\hline
        \multicolumn{2}{|p{16cm}|}{\textbf{5b. }El sistema informa que el presupuesto de la \OT{} ya fue confirmado. Fin CU.}\\
		\hline
		\multicolumn{2}{|p{16cm}|}{\textbf{8. }El Cliente no confirma la tarea. El Técnico cancela el detalle del presupuesto. Ir al paso 11}\\
		\hline
        \multicolumn{2}{|p{16cm}|}{\textbf{12. }El Cliente no confirmó ninguna de las tarea presupuestadas. El sistema marca la \OT{} como \textit{No se repara}. Fin CU.}\\
		\hline
	\end{longtable}


    %LO RESETEAMOS A 0
    \setcounter{step}{0}

    \noindent\rule{169mm}{0.8mm}\\
    %LA LINEA SEPARA CU, NO SECCIONES
