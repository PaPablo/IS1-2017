
	%DEFINIMOS UN CONTADOR
    %\newcounter{step}
    %\newcommand\inc{\stepcounter{step}\textbf{\thestep. }}


%/******************************************************/
	%/*PLANTILLA PARA CU EXPANDIDO**************************/
	%/*IR ELIMINANDO LOS \inc QUE NO CORRESPONDAN*/
    %/*para copiar la tabla, hay que copiar desde el
    %\begin{longtable} hasta el \setcounter que esta al final*/
	%/*el \inc es el contador que está definido en el informe
	%/******************************************************/
	\begin{longtable}{ |p{8cm}|p{8cm}| }
		\hline
		\multicolumn{2}{|p{16cm}|}{\textbf{Caso de uso}: }\\
		\multicolumn{2}{|p{16cm}|}{\textbf{Tipo}: }\\
		\multicolumn{2}{|p{16cm}|}{\textbf{Actores}: }\\
		\multicolumn{2}{|p{16cm}|}{\textbf{Descripción}: }\\
		\multicolumn{2}{|p{16cm}|}{\textbf{Precondiciones}: }\\
		\multicolumn{2}{|p{16cm}|}{\textbf{Postcondiciones}: }\\
		\hline
		\multicolumn{2}{|c|}{\textbf{Curso normal de los eventos}}\\
		\hline
		\textbf{Acción de los actores} & \textbf{Respuesta del sistema}\\
		\hline
			\inc Este CU comienza cuando & \\
			\hline
			\inc  & \inc  \\
			\hline
			\inc  & \inc  \\
			\hline
			\inc  & \inc  \\
			\hline
			\inc  & \inc  \\
			\hline
			\inc  & \inc  \\
			\hline
			\inc  & \inc  \\
			\hline
			\inc  & \inc  \\
			\hline
			\inc  & \inc  \\
			\hline
			\inc  & \inc   \\
			\hline
			\inc  & \inc   \\
			\hline
			\inc  & \inc   \\
			\hline
			\inc Fin CU. & \\
		\hline
		\multicolumn{2}{|c|}{\textbf{Cursos alternos}}\\
		\hline
		\multicolumn{2}{|p{16cm}|}{\textbf{1. }Bla bla }\\
		\hline
		\multicolumn{2}{|p{16cm}|}{\textbf{1. }Bla bla }\\
		\hline
		\multicolumn{2}{|p{16cm}|}{\textbf{1. }Bla bla }\\
		\hline	
	\end{longtable}


    %LO RESETEAMOS A 0
    \setcounter{step}{0}

    \noindent\rule{169mm}{0.8mm}\\
    %LA LINEA SEPARA CU, NO SECCIONES
