\documentclass[12pt]{extarticle}
\usepackage[utf8]{inputenc}
\usepackage{caption}
\usepackage{subcaption}
\usepackage[spanish]{babel}
\usepackage{multicol}
\usepackage{fancyhdr}
\usepackage{longtable}
\usepackage{enumitem}

\usepackage{url}
\usepackage[a4paper]{geometry}
\usepackage{float}
\usepackage{setspace}
\usepackage{color}   %May be necessary if you want to color links
\usepackage{hyperref}
\hypersetup{
    colorlinks=true, %set true if you want colored links
    linktoc=all,     %set to all if you want both sections and subsections linked
    linkcolor=blue,  %choose some color if you want links to stand out
}
\usepackage{graphicx}
\graphicspath{ {images/} }


\begin{document}

    \subsection{Casos de Uso expandidos}

    %DEFINIMOS UN CONTADOR
    \newcounter{step}
    \newcommand\inc{\stepcounter{step}\textbf{\thestep. }}
    \newcommand\resetinc{\setcounter{step}{0}}
    
    \newcommand\raya{\noindent\rule{169mm}{0.8mm}\\}

\begin{longtable}{ |p{8cm}|p{8cm}| }
    \hline
    \multicolumn{2}{|p{16cm}|}{\textbf{Caso de uso}: Agregar Empleado}\\
    \multicolumn{2}{|p{16cm}|}{\textbf{Tipo}: Primario Esencial}\\
    \multicolumn{2}{|p{16cm}|}{\textbf{Actores}: Encargado [Gerente | Jefe de Taller], Empleado}\\
    \multicolumn{2}{|p{16cm}|}{\textbf{Descripción}: El Gerente o Jefe de Taller agrega un nuevo empleado al sistema, si el Jefe de Taller lo desea, le puede agregar tareas que pueda realizar.}\\
    \multicolumn{2}{|p{16cm}|}{\textbf{Precondiciones}: - }\\
    \multicolumn{2}{|p{16cm}|}{\textbf{Postcondiciones}: - }\\
    \hline
    \multicolumn{2}{|c|}{\textbf{Curso normal de los eventos}}\\
    \hline
    \textbf{Acción de los actores} & \textbf{Respuesta del sistema}\\
    \hline
        \inc Este CU comienza cuando el Encargado desea agregar un nuevo empleado.& \\
        \hline
        \inc  El Encargado le solicita el numero de documento al nuevo empleado, el empleado se lo da. Le solicita al Sistema agregar el empleado ingresando su número de documento.& \\
        \hline
        & \inc  Busca al empleado.\\
        \hline
        & \inc  Crea un nuevo empleado.\\
        \hline
        \inc El Encargado le solicita el resto de sus datos: Nombre y Apellido completos, número de teléfono, domicilio y correo electrónico. Los ingresa en el sistema y le solicita que los guarde.& \\
        \hline
        & \inc Registra los datos del empleado.\\
        \hline
        \inc El Encargado ingresa un Tipo de Tarea al sistema indicando su nombre y le solicita al sistema que la busque.& \\
        \hline
        & \inc Busca la Tarea. \\
        \hline
        & \inc  Muestra los datos de la Tipo de Tarea.\\
        \hline
        \inc El Encargado ingresa el nombre del Tipo de Equipo para el Tipo de Tarea & \\
        \hline
        & \inc Busca el Tipo de Equipo.\\
        \hline
        & \inc Muestra los datos del Tipo de Equipo\\
        \hline
        \inc El Encargado le solicita al sistema agregar al nuevo Empleado ese Tipo de Tarea para ese Tipo de Equipo.& \\
        \hline
        \inc & Agrega al nuevo empleado ese tipo de Tarea para ese Tipo de Equipo\\
        \hline
        \inc Repetir pasos x a x hasta que el Encargado no desee agregar más Tareas al Empleado.& \\
        \hline
        & \inc Muestra los datos personales del Empleado junto con sus nuevas tareas asociadas (si las hubiese).\\
        \hline
        \inc Fin CU. & \\
    \hline
    \multicolumn{2}{|c|}{\textbf{Cursos alternos}}\\
    \hline
    \multicolumn{2}{|p{16cm}|}{\textbf{4. } El Sistema informa que el Empleado ya existe. Fin CU.}\\
    \hline
    \multicolumn{2}{|p{16cm}|}{\textbf{9a. } El Sistema informa que el Tipo de Tarea no existe. Si el Encargado desea ingresar un nuevo Tipo de Tarea, ir a paso x. Si no, ir a paso x.}\\
    \hline
    \multicolumn{2}{|p{16cm}|}{\textbf{12a. } El Sistema informa que el Tipo de Equipo no existe.}\\
    \hline	
    \multicolumn{2}{|p{16cm}|}{\textbf{12b. } El Sistema informa que el Tipo de Equipo no está disponible para el Tipo de Tarea seleccionada.}\\
    \hline	
\end{longtable}


%LO RESETEAMOS A 0
\resetinc

\noindent\rule{169mm}{0.8mm}\\
%LA LINEA SEPARA CU, NO SECCIONES




\pagebreak

\end{document}
