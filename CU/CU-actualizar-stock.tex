
	%DEFINIMOS UN CONTADOR
    %\newcounter{step}
    %\newcommand\inc{\stepcounter{step}\textbf{\thestep. }}


%/******************************************************/
	%/*PLANTILLA PARA CU EXPANDIDO**************************/
	%/*IR ELIMINANDO LOS \inc QUE NO CORRESPONDAN*/
    %/*para copiar la tabla, hay que copiar desde el
    %\begin{longtable} hasta el \setcounter que esta al final*/
	%/*el \inc es el contador que está definido en el informe
	%/******************************************************/
	\begin{longtable}{ |p{8cm}|p{8cm}| }
		\hline
		\multicolumn{2}{|p{16cm}|}{\textbf{Caso de uso}: Actualizar stock}\\
		\multicolumn{2}{|p{16cm}|}{\textbf{Tipo}: Primario Esencial}\\
		\multicolumn{2}{|p{16cm}|}{\textbf{Actores}: Gerente}\\
        \multicolumn{2}{|p{16cm}|}{\textbf{Descripción}: el Gerente ingresa el cambio de stock de un producto. El sistema registra la actualización del stock}\\
		\multicolumn{2}{|p{16cm}|}{\textbf{Precondiciones}: -}\\
        \multicolumn{2}{|p{16cm}|}{\textbf{Postcondiciones}: Stock de un producto actualizado}\\
		\hline
		\multicolumn{2}{|c|}{\textbf{Curso normal de los eventos}}\\
		\hline
		\textbf{Acción de los actores} & \textbf{Respuesta del sistema}\\
		\hline
			\inc Este CU comienza cuando el Gerente desea modificar manualmente el stock de un producto& \\
			\hline
			\inc  El Gerente ingresa el código de referencia interna del producto& \\
			\hline
			& \inc Busca el producto \\
			\hline
			& \inc Muestra los datos del producto\\
			\hline


			\inc El Gerente ingresa una nueva cantidad para modificar el stock del producto & \\
			\hline
			& \inc Registra la nueva cantidad en el producto\\
			\hline
			& \inc Muestra los datos del producto\\
			\hline
			\inc Fin CU. & \\


		\hline
		\multicolumn{2}{|c|}{\textbf{Cursos alternos}}\\
		\hline
		\multicolumn{2}{|p{16cm}|}{\textbf{4. }El sistema informa que el producto no existe. Fin CU.}\\
		\hline
		\multicolumn{2}{|p{16cm}|}{\textbf{7. }El sistema informa que se decrementó el stock del producto más allá de lo disponible. Fin CU.}\\
		\hline
	\end{longtable}


    %LO RESETEAMOS A 0
    \setcounter{step}{0}

    \noindent\rule{169mm}{0.8mm}\\
    %LA LINEA SEPARA CU, NO SECCIONES
