\documentclass[12pt]{extarticle}
\usepackage[utf8]{inputenc}
\usepackage[spanish]{babel}
\usepackage{multicol}
\usepackage{longtable}
\usepackage{enumitem}

\begin{document}
	%DEFINIMOS UN CONTADOR
    \newcounter{step}
    \newcommand\inc{\stepcounter{step}\textbf{\thestep. }}
    %LO RESETEAMOS A 0
    \newcommand\resetinc{\setcounter{step}{0}}
    \newcommand\raya{\noindent\rule{169mm}{0.8mm}\\}


%/******************************************************/
	%/*PLANTILLA PARA CU EXPANDIDO**************************/
	%/*IR ELIMINANDO LOS \inc QUE NO CORRESPONDAN*/
    %/*para copiar la tabla, hay que copiar desde el
    %\begin{longtable} hasta el \setcounter que esta al final*/
	%/*el \inc es el contador que está definido en el informe
	%/******************************************************/
	\begin{longtable}{ |p{8cm}|p{8cm}| }
		\hline
		\multicolumn{2}{|p{16cm}|}{\textbf{Caso de uso}: Registrar Pedido a Proveedor}\\
		\multicolumn{2}{|p{16cm}|}{\textbf{Tipo}: Primario Esencial}\\
		\multicolumn{2}{|p{16cm}|}{\textbf{Actores}: Operario Contable}\\
		\multicolumn{2}{|p{16cm}|}{\textbf{Descripción}: El Operario contable registra un pedido a un proveedor en el sistema, con el código interno de referencia y la cantidad de los productos pedido.}\\
		\multicolumn{2}{|p{16cm}|}{\textbf{Precondiciones}: -}\\
		\multicolumn{2}{|p{16cm}|}{\textbf{Postcondiciones}: Pedido registrado.}\\
		\hline
		\multicolumn{2}{|c|}{\textbf{Curso normal de los eventos}}\\
		\hline
		\textbf{Acción de los actores} & \textbf{Respuesta del sistema}\\
		\hline

			\inc Este CU comienza cuando el Operario Contable decide registrar un pedido realizado a un proveedor.& \\
			\hline
            \inc El Operario Contable le solicita al sistema buscar un proveedor ingresando su nombre.& \\
			\hline
            & \inc Busca el Proveedor.\\
			\hline
			& \inc Muestra los datos del proveedor\\
			\hline

            \inc El Operario Contable ingresa el código interno de referencia de un producto.&\\
			\hline
            & \inc Busca el producto.\\
			\hline
			& \inc Muestra los datos del producto.\\
			\hline
            \inc El Operario Contable le solicita al sistema agregar el producto e ingresa la cantidad pedida.&\\
			\hline

            & \inc Agrega el producto al pedido.\\
			\hline
            & \inc Muestra los productos incluídos en el pedido\\
			\hline
            \inc Repetir pasos 5 al 12 hasta que el Operario Contable no desee incluir más productos en el pedido& \\
			\hline
			\inc El Operario contable le solicita al sistema crear un nuevo pedido con ese proveedor con los productos seleccionados.& \\
			\hline
            
			& \inc Crea el pedido.\\
			\hline
            & \inc Muestra el pedido completo\\
			\hline
        \hline
		\multicolumn{2}{|c|}{\textbf{Cursos alternos}}\\
		\hline
		\multicolumn{2}{|p{16cm}|}{\textbf{4. }El Sistema informa que el Proveedor no existe. Fin CU.}\\
		\hline
		\multicolumn{2}{|p{16cm}|}{\textbf{9a. }El Sistema informa que el Producto no existe. Ir a paso 13.}\\
		\hline	
		\multicolumn{2}{|p{16cm}|}{\textbf{9b. }El Sistema informa que el Producto no es proveído por el Proveedor. Ir a paso 13.}\\
		\hline	
		\multicolumn{2}{|p{16cm}|}{\textbf{14. }El Sistema informa que el Pedido no contiene ningún producto y elimina el pedido. Ir a paso 15}\\
		\hline	
	\end{longtable}
    \resetinc{}
    \raya{}
\end{document}
