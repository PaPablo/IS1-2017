\documentclass[12pt]{extarticle}
\usepackage[utf8]{inputenc}
\usepackage[spanish]{babel}
\usepackage{multicol}
\usepackage{longtable}
\usepackage{enumitem}

\begin{document}
	%DEFINIMOS UN CONTADOR
    \newcounter{step}
    \newcommand\inc{\stepcounter{step}\textbf{\thestep. }}
    %LO RESETEAMOS A 0
    \newcommand\resetinc{\setcounter{step}{0}}
    \newcommand\raya{\noindent\rule{169mm}{0.8mm}\\}


%/******************************************************/
	%/*PLANTILLA PARA CU EXPANDIDO**************************/
	%/*IR ELIMINANDO LOS \inc QUE NO CORRESPONDAN*/
    %/*para copiar la tabla, hay que copiar desde el
    %\begin{longtable} hasta el \setcounter que esta al final*/
	%/*el \inc es el contador que está definido en el informe
	%/******************************************************/



        %despues de crear la tarifa, vamos a tener que crear las vinculaciones con tipoServicio y tipoEquipo
        %despues vamos a tener que crear tarifas para ese tipo de tarea indicando tipoServicio y tipoEquipo


	\begin{longtable}{ |p{8cm}|p{8cm}| }
		\hline
		\multicolumn{2}{|p{16cm}|}{\textbf{Caso de uso}: Agregar tipo de tarea}\\
		\multicolumn{2}{|p{16cm}|}{\textbf{Tipo}: Primario Esencial}\\
		\multicolumn{2}{|p{16cm}|}{\textbf{Actores}: Jefe de Taller}\\
		\multicolumn{2}{|p{16cm}|}{\textbf{Descripción}: El Jefe de Taller registra un nuevo tipo de tarea, indicando el tipo de equipo para el cual se podrá realizar. El sistema crea el tipo de tarea y lo vincula con el tipo de equipo}\\
		\multicolumn{2}{|p{16cm}|}{\textbf{Precondiciones}: -}\\
		\multicolumn{2}{|p{16cm}|}{\textbf{Postcondiciones}: Tipo de tarea creado}\\
		\hline
		\multicolumn{2}{|c|}{\textbf{Curso normal de los eventos}}\\
		\hline
		\textbf{Acción de los actores} & \textbf{Respuesta del sistema}\\
		\hline
			\inc Este CU comienza cuando el Jefe de Taller desea agregar un nuevo tipo de tarea & \\
			\hline
            \inc El Jefe de Taller ingresa un nombre de tipo de equipo y le solicita que lo busque& \\
			\hline
			& \inc Busca el tipo de equipo\\
			\hline
			& \inc Muestra los datos del tipo de equipo\\
			\hline


			\inc El Jefe de Taller ingresa el nombre y la descripción de un tipo de tarea para agregar al tipo de equipo ingresado& \\
			\hline
			& \inc Crea el tipo de tarea y lo vincula con el tipo de equipo \\
			\hline
            & \inc Muestra los datos del tipo de tarea\\
			\hline
			\inc Fin CU. & \\
		\hline
		\multicolumn{2}{|c|}{\textbf{Cursos alternos}}\\
		\hline
		\multicolumn{2}{|p{16cm}|}{\textbf{4. }El sistema informa que el tipo de equipo ingresado no existe. Fin CU.}\\
		\hline
		\multicolumn{2}{|p{16cm}|}{\textbf{7. }El sistema informa que el tipo de tarea ingresado ya existe. Fin CU}\\
		\hline	
	\end{longtable}

    \resetinc{}
    \raya{}
\end{document}
