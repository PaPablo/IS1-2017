\documentclass[12pt]{extarticle}
\usepackage[utf8]{inputenc}
\usepackage[spanish]{babel}
\usepackage{multicol}
\usepackage{longtable}
\usepackage{enumitem}

\begin{document}
	%DEFINIMOS UN CONTADOR
    \newcounter{step}
    \newcommand\inc{\stepcounter{step}\textbf{\thestep. }}
    %LO RESETEAMOS A 0
    \newcommand\resetinc{\setcounter{step}{0}}
    \newcommand\raya{\noindent\rule{169mm}{0.8mm}\\}


%/******************************************************/
	%/*PLANTILLA PARA CU EXPANDIDO**************************/
	%/*IR ELIMINANDO LOS \inc QUE NO CORRESPONDAN*/
    %/*para copiar la tabla, hay que copiar desde el
    %\begin{longtable} hasta el \setcounter que esta al final*/
	%/*el \inc es el contador que está definido en el informe
	%/******************************************************/
	\begin{longtable}{ |p{8cm}|p{8cm}| }
		\hline
		\multicolumn{2}{|p{16cm}|}{\textbf{Caso de uso}: Finalizar Tarea}\\
		\multicolumn{2}{|p{16cm}|}{\textbf{Tipo}: Primario Esencial}\\
		\multicolumn{2}{|p{16cm}|}{\textbf{Actores}: Técnico}\\
		\multicolumn{2}{|p{16cm}|}{\textbf{Descripción}: El Técnico asignado a una Tarea decide finalizarla, agrega una observación y luego el sistema la marca como finalizada.}\\
		\multicolumn{2}{|p{16cm}|}{\textbf{Precondiciones}: -}\\
		\multicolumn{2}{|p{16cm}|}{\textbf{Postcondiciones}: Tarea finalizada}\\
		\hline
		\multicolumn{2}{|c|}{\textbf{Curso normal de los eventos}}\\
		\hline
		\textbf{Acción de los actores} & \textbf{Respuesta del sistema}\\
		\hline
            \inc Este CU comienza cuando un Técnico decide finalizar una Tarea& \\
            \hline
            \inc El Técnico le solicita al sistema buscar la \OT{} ingresando su número.& \\
            \hline
            & \inc Busca la \OT{}\\
            \hline
            & \inc Muestra los datos de la \OT{}\\
            \hline


            \inc El Técnico le solicita al sistema buscar la Tarea que desea finalizar, ingresando el número de Tarea& \\
            \hline
            & \inc Busca la Tarea\\
            \hline
            & \inc Muestra los datos de la Tarea\\
            \hline
            \inc El Técnico le solicita al Sistema buscar un repuesto asociado a la Tarea para marcarlo como usado, ingresando su código y cantidad& \\
            \hline


            & \inc Busca el repuesto.\\
            \hline
            & \inc Disminuye el stock del repuesto según la cantidad ingresada e incrementa el stock del producto (cancela el remanente reservado) en caso de ser necesario.\\
            \hline
            & \inc Muestra los datos de la Tarea\\
            \hline
            \inc Repetir pasos 8 al 11 mientras el Técnico desee registrar más repuestos utilizados en la Tarea& \\
            \hline
            
            % Mas o menos por acá hay que ver qué hacemos con los repuestos asignados a la OT
            \inc El técnico le solicita al sistema marcar la Tarea como finalizada, agregando además una observación sobre el desarrollo de dicha tarea.&\\
            \hline
            & \inc Marca la tarea como finalizada, registrando su fecha y hora de finalización, y agregando la observación realizada.\\
            \hline
            & \inc Muestra las tareas pendientes de la \OT{}.\\
            \hline
            \inc Fin CU. & \\
        \hline
		\multicolumn{2}{|c|}{\textbf{Cursos alternos}}\\
		\hline
        \multicolumn{2}{|p{16cm}|}{\textbf{4a. }El Sistema informa que le \OT{} no existe. Fin CU.}\\
		\hline
        \multicolumn{2}{|p{16cm}|}{\textbf{4b. }El Sistema informa que la \OT{} no posee Tareas por finalizar. Fin CU.}\\
		\hline
		\multicolumn{2}{|p{16cm}|}{\textbf{7. }El Sistema informa que la Tarea no existe. Fin CU.}\\
		\hline	
        \multicolumn{2}{|p{16cm}|}{\textbf{8a. }La Tarea no incluye repuestos. Ir al paso 13}\\
		\hline	
        \multicolumn{2}{|p{16cm}|}{\textbf{8b. }El técnico no utilizó ninguno de los repuestos reservados. El sistema incrementa el stock del producto (cancela la reserva del producto). Ir al paso 13}\\
		\hline	
		\multicolumn{2}{|p{16cm}|}{\textbf{10a. }El Sistema informa que el repuesto no existe. Ir al paso 12}\\
		\hline	
		\multicolumn{2}{|p{16cm}|}{\textbf{11a. }El Sistema informa que el repuesto no está asignado a la Tarea. Ir al paso 12}\\
		\hline	
		\multicolumn{2}{|p{16cm}|}{\textbf{11b. }El Sistema informa que el respuesto ya fue marcado como utilizado. Ir al paso 12}\\
		\hline	
        \multicolumn{2}{|p{16cm}|}{\textbf{11c. }El Sistema informa que la cantidad de repuesto ingresada excede la cantidad reservada. Ir al paso 12}\\
		\hline	
	\end{longtable}

    \resetinc{}
    \raya{}
\end{document}
