
	%DEFINIMOS UN CONTADOR
    %\newcounter{step}
    %\newcommand\inc{\stepcounter{step}\textbf{\thestep. }}


%/******************************************************/
	%/*PLANTILLA PARA CU EXPANDIDO**************************/
	%/*IR ELIMINANDO LOS \inc QUE NO CORRESPONDAN*/
    %/*para copiar la tabla, hay que copiar desde el
    %\begin{longtable} hasta el \setcounter que esta al final*/
	%/*el \inc es el contador que está definido en el informe
	%/******************************************************/
	\begin{longtable}{ |p{8cm}|p{8cm}| }
		\hline
		\multicolumn{2}{|p{16cm}|}{\textbf{Caso de uso}: Eliminar cliente}\\
		\multicolumn{2}{|p{16cm}|}{\textbf{Tipo}: Esencial, primario}\\
		\multicolumn{2}{|p{16cm}|}{\textbf{Actores}: Jefe de Taller}\\
		\multicolumn{2}{|p{16cm}|}{\textbf{Descripción}: El Jefe de Taller da de baja un cliente, de modo que no se abrirán nuevas órdenes de trabajo a su nombre.}\\
		\multicolumn{2}{|p{16cm}|}{\textbf{Precondiciones}: -}\\
		\multicolumn{2}{|p{16cm}|}{\textbf{Postcondiciones}: Cliente dado de baja}\\
		\hline
		\multicolumn{2}{|c|}{\textbf{Curso normal de los eventos}}\\
		\hline
		\textbf{Acción de los actores} & \textbf{Respuesta del sistema}\\
		\hline
			\inc Este CU comienza cuando el Jefe de Taller decide eliminar un cliente.& \\
			\hline
			\inc  El Jefe de Taller le solicita al sistema buscar al cliente, ingresando su DNI/CUIT/CUIL. & \\
			\hline
			& \inc Busca al cliente. \\
			\hline
			& \inc Muestra los datos del cliente. \\
			\hline
			\inc El Jefe de Taller le solicita al sistema eliminar al cliente.&\\
			\hline
			& \inc Verifica el estado de la cuenta del cliente. \\
			\hline
			& \inc Da de baja al cliente. \\
			\hline
			\inc Fin CU. & \\
		\hline
		\multicolumn{2}{|c|}{\textbf{Cursos alternos}}\\
		\hline
		\multicolumn{2}{|p{16cm}|}{\textbf{4. }El sistema informa que el cliente no existe. Fin CU.}\\
		\hline
		\multicolumn{2}{|p{16cm}|}{\textbf{7. }El sistema informa que el cliente posee deuda. Fin CU.}\\
		\hline	
	\end{longtable}

    %LO RESETEAMOS A 0
    \setcounter{step}{0}

    \noindent\rule{169mm}{0.8mm}\\
    %LA LINEA SEPARA CU, NO SECCIONES

%----------------------------NUEVO CU-------------------------------
	\begin{longtable}{ |p{8cm}|p{8cm}| }
		\hline
		\multicolumn{2}{|p{16cm}|}{\textbf{Caso de uso}: Eliminar proveedor}\\
		\multicolumn{2}{|p{16cm}|}{\textbf{Tipo}: Esencial, primario}\\
		\multicolumn{2}{|p{16cm}|}{\textbf{Actores}: Jefe de Taller}\\
		\multicolumn{2}{|p{16cm}|}{\textbf{Descripción}: El Jefe de Taller da de baja un proveedor, de modo que no pueden hacerse nuevos pedidos a éste.}\\
		\multicolumn{2}{|p{16cm}|}{\textbf{Precondiciones}: -}\\
		\multicolumn{2}{|p{16cm}|}{\textbf{Postcondiciones}: Proveedor dado de baja.}\\
		\hline
		\multicolumn{2}{|c|}{\textbf{Curso normal de los eventos}}\\
		\hline
		\textbf{Acción de los actores} & \textbf{Respuesta del sistema}\\
		\hline
			\inc Este CU comienza cuando el Jefe de Taller decide dar de baja un proveedor.& \\
			\hline
			\inc  El Jefe de Taller le solicita al sistema buscar al proveedor, ingresando su CUIT. & \\
			\hline
			& \inc Busca al proveedor.\\
			\hline
			& \inc Busca los pedidos pendientes de entrega. \\
			\hline
			& \inc Muestra los datos del proveedor. \\
			\hline
			\inc El Jefe de Taller le solicita al sistema eliminar al proveedor.&\\
			\hline
			& \inc Da de baja al proveedor.\\
			\hline
			\inc Fin CU. & \\
		\hline
		\multicolumn{2}{|c|}{\textbf{Cursos alternos}}\\
		\hline
		\multicolumn{2}{|p{16cm}|}{\textbf{4. }El sistema informa que el proveedor no existe. Fin CU.}\\
		\hline
		\multicolumn{2}{|p{16cm}|}{\textbf{5. }El sistema informa que el proveedor tiene pedidos pendientes de entrega. Fin CU.}\\
		\hline	
	\end{longtable}

    %LO RESETEAMOS A 0
    \setcounter{step}{0}

    \noindent\rule{169mm}{0.8mm}\\
    %LA LINEA SEPARA CU, NO SECCIONES

%----------------------------NUEVO CU-------------------------------
	\begin{longtable}{ |p{8cm}|p{8cm}| }
		\hline
		\multicolumn{2}{|p{16cm}|}{\textbf{Caso de uso}: Eliminar empleado}\\
		\multicolumn{2}{|p{16cm}|}{\textbf{Tipo}: Esencial, primario}\\
		\multicolumn{2}{|p{16cm}|}{\textbf{Actores}: Jefe de Taller}\\
		\multicolumn{2}{|p{16cm}|}{\textbf{Descripción}: El Jefe de Taller da de baja un empleado, de modo que no pueda asignársele nuevas tareas, o tomar nuevas órdenes de trabajo.}\\
		\multicolumn{2}{|p{16cm}|}{\textbf{Precondiciones}: -}\\
		\multicolumn{2}{|p{16cm}|}{\textbf{Postcondiciones}: Empleado dado de baja.}\\
		\hline
		\multicolumn{2}{|c|}{\textbf{Curso normal de los eventos}}\\
		\hline
		\textbf{Acción de los actores} & \textbf{Respuesta del sistema}\\
		\hline
			\inc Este CU comienza cuando el Jefe de Taller decide dar de baja un empleado.& \
			\hline
			\inc  El Jefe de Taller le solicita al sistema buscar al empleado, ingresando su DNI. & \\
			\hline
			& \inc Busca al empleado.\\
			\hline
			& \inc Busca las órdenes de trabajo abiertas. \\
			\hline
			& \inc Busca las tareas pendientes. \\
			\hline
			& \inc Muestra los datos del empleado. \\
			\hline
			\inc El Jefe de Taller le solicita al sistema eliminar al empleado.&\\
			\hline
			& \inc Da de baja al empleado.\\
			\hline
			\inc Fin CU. & \\
		\hline
		\multicolumn{2}{|c|}{\textbf{Cursos alternos}}\\
		\hline
		\multicolumn{2}{|p{16cm}|}{\textbf{4. }El sistema informa que el empleado no existe. Fin CU.}\\
		\hline
		\multicolumn{2}{|p{16cm}|}{\textbf{6. }El sistema informa que el empleado posee tareas pendientes. Fin CU.}\\
		\hline
	\end{longtable}

    %LO RESETEAMOS A 0
    \setcounter{step}{0}

    \noindent\rule{169mm}{0.8mm}\\
    %LA LINEA SEPARA CU, NO SECCIONES

%----------------------------NUEVO CU-------------------------------
	\begin{longtable}{ |p{8cm}|p{8cm}| }
		\hline
		\multicolumn{2}{|p{16cm}|}{\textbf{Caso de uso}: Eliminar equipo}\\
		\multicolumn{2}{|p{16cm}|}{\textbf{Tipo}: Esencial, primario}\\
		\multicolumn{2}{|p{16cm}|}{\textbf{Actores}: Jefe de Taller}\\
		\multicolumn{2}{|p{16cm}|}{\textbf{Descripción}: El Jefe de Taller da de baja un equipo, de modo que no se puedan crear nuevas órdenes de trabajo para éste.}\\
		\multicolumn{2}{|p{16cm}|}{\textbf{Precondiciones}: -}\\
		\multicolumn{2}{|p{16cm}|}{\textbf{Postcondiciones}: Equipo dado de baja.}\\
		\hline
		\multicolumn{2}{|c|}{\textbf{Curso normal de los eventos}}\\
		\hline
		\textbf{Acción de los actores} & \textbf{Respuesta del sistema}\\
		\hline
			\inc Este CU comienza cuando el Jefe de Taller decide eliminar un equipo.& \\
			\hline
			\inc El Jefe de Taller le solicita al sistema buscar el equipo, ingresando su número de serie. &\\
			\hline
			& \inc Busca el equipo.\\
			\hline
			& \inc Busca las órdenes de trabajo asociadas.\\
			\hline
			& \inc Muestra los datos del equipo.\\
			\hline
			\inc El Jefe de Taller le solicita al sistema eliminar el equipo.&\\
			\hline
			& \inc Da de baja el equipo.\\
			\hline
			\inc Fin CU. & \\
		\hline
		\multicolumn{2}{|c|}{\textbf{Cursos alternos}}\\
		\hline
		\multicolumn{2}{|p{16cm}|}{\textbf{4. }El sistema informa que el equipo no existe. Fin CU. }\\
		\hline
		\multicolumn{2}{|p{16cm}|}{\textbf{5. }El sistema informa que el equipo tiene órdenes de trabajo abiertas. Fin CU.}\\
		\hline	
	\end{longtable}


    %LO RESETEAMOS A 0
    \setcounter{step}{0}

    \noindent\rule{169mm}{0.8mm}\\
    %LA LINEA SEPARA CU, NO SECCIONES
	%/******************************************************/
	%/*PLANTILLA PARA CU EXPANDIDO**************************/
	%/*IR ELIMINANDO LOS \inc QUE NO CORRESPONDAN*/
    %/*para copiar la tabla, hay que copiar desde el
    %\begin{longtable} hasta el \setcounter que esta al final*/
	%/*el \inc es el contador que está definido en el informe
	%/******************************************************/
	\begin{longtable}{ |p{8cm}|p{8cm}| }
		\hline
		\multicolumn{2}{|p{16cm}|}{\textbf{Caso de uso}: }\\
		\multicolumn{2}{|p{16cm}|}{\textbf{Tipo}: }\\
		\multicolumn{2}{|p{16cm}|}{\textbf{Actores}: }\\
		\multicolumn{2}{|p{16cm}|}{\textbf{Descripción}: }\\
		\multicolumn{2}{|p{16cm}|}{\textbf{Precondiciones}: }\\
		\multicolumn{2}{|p{16cm}|}{\textbf{Postcondiciones}: }\\
		\hline
		\multicolumn{2}{|c|}{\textbf{Curso normal de los eventos}}\\
		\hline
		\textbf{Acción de los actores} & \textbf{Respuesta del sistema}\\
		\hline
			\inc Este CU comienza cuando & \\
			\hline
			\inc  & \inc  \\
			\hline
			\inc  & \inc  \\
			\hline
			\inc  & \inc  \\
			\hline
			\inc  & \inc  \\
			\hline
			\inc  & \inc  \\
			\hline
			\inc  & \inc  \\
			\hline
			\inc  & \inc  \\
			\hline
			\inc  & \inc  \\
			\hline
			\inc  & \inc   \\
			\hline
			\inc  & \inc   \\
			\hline
			\inc  & \inc   \\
			\hline
			\inc Fin CU. & \\
		\hline
		\multicolumn{2}{|c|}{\textbf{Cursos alternos}}\\
		\hline
		\multicolumn{2}{|p{16cm}|}{\textbf{1. }Bla bla }\\
		\hline
		\multicolumn{2}{|p{16cm}|}{\textbf{1. }Bla bla }\\
		\hline
		\multicolumn{2}{|p{16cm}|}{\textbf{1. }Bla bla }\\
		\hline	
	\end{longtable}


    %LO RESETEAMOS A 0
    \setcounter{step}{0}

    \noindent\rule{169mm}{0.8mm}\\
    %LA LINEA SEPARA CU, NO SECCIONES
