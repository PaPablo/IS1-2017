\documentclass[12pt]{extarticle}
\usepackage[utf8]{inputenc}
\usepackage{caption}
\usepackage{subcaption}
\usepackage[spanish]{babel}
\usepackage{multicol}
\usepackage{fancyhdr}
\usepackage{longtable}
\usepackage{enumitem}

\usepackage{url}
\usepackage[a4paper]{geometry}
\usepackage{float}
\usepackage{setspace}
\usepackage{color}   %May be necessary if you want to color links
\usepackage{hyperref}
\hypersetup{
    colorlinks=true, %set true if you want colored links
    linktoc=all,     %set to all if you want both sections and subsections linked
    linkcolor=blue,  %choose some color if you want links to stand out
}
\usepackage{graphicx}
\graphicspath{ {images/} }


\begin{document}


    %DEFINIMOS UN CONTADOR
    \newcounter{step}
    \newcommand\inc{\stepcounter{step}\textbf{\thestep. }}
    \newcommand\resetinc{\setcounter{step}{0}}
    
    \newcommand\raya{\noindent\rule{169mm}{0.8mm}\\}

    \newcommand\OT{\textit{Orden de Trabajo}}
    \newcommand\OTs{\textit{Órdenes de Trabajo}}
    \section{ Circuito de la \OT{} }

\begin{longtable}{ |p{8cm}|p{8cm}| }
    \hline
    \multicolumn{2}{|p{16cm}|}{\textbf{Caso de uso}: \textbf{47}. Atender solicitud de servicio}\\
    \multicolumn{2}{|p{16cm}|}{\textbf{Tipo}: Primario Esencial}\\
    \multicolumn{2}{|p{16cm}|}{\textbf{Actores}: Técnico, Cliente}\\
    \multicolumn{2}{|p{16cm}|}{\textbf{Descripción}: Un Cliente se comunica con la empresa para solicitar un servicio de reparación. El Técnico crea una \OT{} para atender la solicitud.}\\
    \multicolumn{2}{|p{16cm}|}{\textbf{Precondiciones}: -}\\
    \multicolumn{2}{|p{16cm}|}{\textbf{Postcondiciones}: \OT{} creada.}\\
    \hline
    \multicolumn{2}{|c|}{\textbf{Curso normal de los eventos}}\\
    \hline
    \textbf{Acción de los actores} & \textbf{Respuesta del sistema}\\
    \hline
        \inc Este CU comienza cuando un Cliente se contacta con la empresa para requerir un servicio.&\\
        \hline
        \inc El Técnico le solicita el nombre al Cliente y este se lo da. El técnico le solicita al Sistema buscar al Cliente ingresando el nombre.& \\
        \hline
        \inc El técnico le solicita al Sistema buscar al Cliente ingresando su nombre.& \\
        \hline
        & \inc Busca el Cliente.\\
        \hline


        & \inc Verifica que el Cliente no sea moroso.\\
        \hline
        & \inc Muestra los datos personales del Cliente.\\
        \hline
        \inc El Técnico le solicita al Sistema crear una \OT{} a nombre del Cliente y se designa a sí mismo como encargado de la \OT{}.& \\
        \hline
        & \inc Crea la \OT{}.\\
        \hline


        \inc El técnico ingresa el número de serie del equipo y le solicita al sistema que lo incorpore a la \OT{}.& \\
        \hline
        & \inc Busca el equipo. \\
        \hline
        & \inc Incorpora el equipo a la \OT{}.\\
        \hline
        \inc El Técnico le solicita al Cliente una descripción de problema del equipo. El Cliente se la da. El técnico ingresa en el sistema la descripción y le solicita al Sistema que la incorpore a la \OT{}& \\
        \hline


        \inc El técnico ingresa en el sistema la descripción y le solicita al Sistema que la incorpore a la \OT{}& \\
        \hline
        & \inc Incorpora la descripción del problema a la \OT{}\\
        \hline
        \inc El técnico ingresa un tipo de servicio y le indica al sistema que la \OT{} se realizará bajo ese tipo de servicio. &\\
        \hline
        & \inc Busca el tipo de servicio.\\
        \hline


        & \inc Incorpora el tipo de servicio a la \OT{}.\\
        \hline
        \inc El técnico le solicita al Sistema calcular el monto de \textit{RDyP} para el rubro al cual pertenece el equipo.& \\
        \hline
        & \inc Busca el precio de \textit{RDyP}.\\
        \hline
        & \inc Muestra el importe.\\
        \hline


        \inc El Técnico le comunica al Cliente el monto a pagar por la \textit{RDyP}.&\\
        \hline
        \inc El Cliente abona en efectivo el importe de \textit{RDyP}.& \\
        \hline
        \inc El Técnico le solicita al sistema registrar el pago del \textit{RDyP} ingresando el monto entregado por el Cliente& \\
        \hline
        & \inc Registra el pago del importe.\\
        \hline


        & \inc Calcula y muestra el vuelto.\\
        \hline
        \inc El Técnico deposita el efectivo recibido, extrae la diferencia y se la entrega al Cliente (de ser necesario)&\\
        \hline
        \inc El Técnico le solicita al Sistema imprimir la factura por el importe abonado y el comprobante de la \OT{}& \\
        \hline
        & \inc Imprime la factura y el comprobante de la \OT{}.\\
        \hline


        \inc El Técnico le entrega al Cliente la factura junto con el comprobante de la \OT{}.&\\
        \hline
        \inc Fin CU. & \\
    \hline


    \multicolumn{2}{|c|}{\textbf{Cursos alternos}}\\
    \hline
    \multicolumn{2}{|p{16cm}|}{\textbf{4. }El Sistema informa que el Cliente no existe. Fin CU.}\\
    \hline
    \multicolumn{2}{|p{16cm}|}{\textbf{5. }El Sistema informa que el Cliente es moroso. El Técnico consulta con el Gerente si prestarle o no servicio al Cliente. Si se decide que sí, ir a paso 6. Si no, fin CU.}\\
    \hline
    \multicolumn{2}{|p{16cm}|}{\textbf{9. }El Técnico ve que es la primera vez que ingresa el equipo al taller. El técnico ingresa en el sistema el rubro del equipo, marca y modelo. El sistema lo registra y genera automáticamente un número de serie para el equipo. Ir a paso 10.}\\
    \hline	
    \multicolumn{2}{|p{16cm}|}{\textbf{15. }El Sistema informa que el Tipo de servicio ingresado no existe. Fin CU.}\\
    \hline	
    \multicolumn{2}{|p{16cm}|}{\textbf{18. }El Sistema informa que no existe un precio de \textit{RDyP} para el rubro al cual pertenece el equipo. El Técnico consulta un valor de \textit{RDyP} con el Jefe de Taller para ese rubro. El Jefe de Taller se lo comunica al técnico. El Técnico lo detalla en la \OT{} y se lo comunica al Cliente. Ir a paso 16.}\\
    \hline	
    \multicolumn{2}{|p{16cm}|}{\textbf{21. }El Cliente decide no abonar el importe de \textit{RDyP}. Fin CU.}\\
    \hline	
\end{longtable}

\resetinc{}
\raya{}


    \textbf{Caso de uso}: \textbf{48}. Comenzar revisión\\
    \textbf{Tipo}: Primario Esencial\\
    \textbf{Actores}: Técnico\\
    \textbf{Descripción}: El técnico encargado de una \OT{} da comienzo a la \textit{RDyP} de la misma. El sistema registra el comienzo de la revisión\\
    \textbf{Precondiciones}: La \OT{} existe y no tiene presupuestos sin confirmar. El Técnico es el Técnico encargado de la \OT{}
    \textbf{Postcondiciones}: \textit{RDyP} de una \OT{} comenzada\\
    
 \resetinc{}
 \raya{}

	\begin{longtable}{ |p{8cm}|p{8cm}| }
		\hline
        \multicolumn{2}{|p{16cm}|}{\textbf{Caso de uso}: \textbf{49}. Finalizar Revisión}\\
		\multicolumn{2}{|p{16cm}|}{\textbf{Tipo}: Primario Esencial}\\
		\multicolumn{2}{|p{16cm}|}{\textbf{Actores}: Técnico}\\
        \multicolumn{2}{|p{16cm}|}{\textbf{Descripción}: El técnico le solicita al Sistema crear un presupuesto para una \OT{}. El Técnico indica las tareas necesarias para completar el trabajo y los repuestos que hagan falta. El Sistema genera el presupuesto correspondiente.}\\
		\multicolumn{2}{|p{16cm}|}{\textbf{Precondiciones}: -}\\
        \multicolumn{2}{|p{16cm}|}{\textbf{Postcondiciones}: \OT{} revisada y prespuesto creado.}\\
		\hline
		\multicolumn{2}{|c|}{\textbf{Curso normal de los eventos}}\\
		\hline
		\textbf{Acción de los actores} & \textbf{Respuesta del sistema}\\
		\hline
            \inc Este CU comienza cuando el Técnico desea dar por finalizada la revisión de una \OT{}.& \\
			\hline
            \inc El Técnico ingresa el número de \OT{} y le solicita al Sistema que cree un presupuesto para dicha \OT{}.& \\
			\hline
            & \inc Busca la \OT{} \\
			\hline
             & \inc Crea el presupuesto y lo vincula con la \OT{}.\\
			\hline


            \inc El Técnico ingresa una Tarea al Sistema y le solicita que la vincule con un detalle de Presupuesto.& \\
			\hline
            & \inc Busca la Tarea ingresada en el Rubro de la \OT{}.\\
			\hline
            & \inc Busca la Tarifa de la Tarea en el Tipo de Servicio de la \OT{}.\\
			\hline
            & \inc Crea el detalle del Presupuesto y le incluye la Tarifa.\\
			\hline


            & \inc El Técnico ingresa un código de repuesto y su cantidad y le solicita al Sistema agregarlo al detalle de Presupuesto.\\
			\hline
            & \inc Busca el Repuesto.\\
			\hline
            & \inc Agrega el Respuesto al detalle del presupuesto.\\
			\hline
            \inc Repetir pasos 5 al 11 hasta que no desee ingresar más Tareas al Presupuesto&\\
			\hline


            & \inc Imprime el Presupuesto.\\
			\hline
			\inc Fin CU. & \\


		\hline
		\multicolumn{2}{|c|}{\textbf{Cursos alternos}}\\
		\hline
        \multicolumn{2}{|p{16cm}|}{\textbf{4a. }El Sistema informa que la \OT{} no existe. Fin CU.}\\
		\hline
        \multicolumn{2}{|p{16cm}|}{\textbf{4b. }El Sistema informa que el Técnico no es el Técnico encargado de la \OT{}. Fin CU.}\\
		\hline
        \multicolumn{2}{|p{16cm}|}{\textbf{4c. }El Sistema informa que la \OT{} ingresada está cerrada y no se le pueden generar nuevos presupuestos. Fin CU.}\\
		\hline
        \multicolumn{2}{|p{16cm}|}{\textbf{5. }El Técnico no desea crear un presupuesto para la \OT{}. El Sistema marca la \OT{} como \textit{No se repara}. Fin CU.}\\
		\hline
		\multicolumn{2}{|p{16cm}|}{\textbf{7. }El Sistema informa que la Tarea no existe. Ir al paso 12.}\\
		\hline
		\multicolumn{2}{|p{16cm}|}{\textbf{8. }El Sistema informa que no existe una Tarifa para el Tipo de Servicio de la \OT{} y la tarea ingresada. Ir al paso 12.}\\
		\hline
		\multicolumn{2}{|p{16cm}|}{\textbf{9. }El Técnico no agrega Repuestos al detalle del presupuesto. Ir al paso 12.}\\
		\hline
		\multicolumn{2}{|p{16cm}|}{\textbf{11. }El Sistema informa que el Repuesto no existe. Ir al paso 12.}\\
		\hline	
	\end{longtable}

\resetinc{}
\raya{}
	\begin{longtable}{ |p{8cm}|p{8cm}| }
		\hline
        \multicolumn{2}{|p{16cm}|}{\textbf{Caso de uso}: \textbf{53}. Comenzar Tarea}\\
		\multicolumn{2}{|p{16cm}|}{\textbf{Tipo}: Primario Esencial}\\
		\multicolumn{2}{|p{16cm}|}{\textbf{Actores}: Técnico}\\
		\multicolumn{2}{|p{16cm}|}{\textbf{Descripción}: El Técnico asignado a una Tarea decide comenzarla, marca su comienzo en el sistema.}\\
		\multicolumn{2}{|p{16cm}|}{\textbf{Precondiciones}: -}\\
		\multicolumn{2}{|p{16cm}|}{\textbf{Postcondiciones}: Tarea comenzada}\\
		\hline
		\multicolumn{2}{|c|}{\textbf{Curso normal de los eventos}}\\
		\hline
		\textbf{Acción de los actores} & \textbf{Respuesta del sistema}\\
		\hline

			\inc Este CU comienza cuando un Técnico decide comenzar a trabajar en una Tarea& \\
			\hline
            \inc El Técnico le solicita al sistema buscar la \OT{} ingresando su número.& \\
			\hline
            & \inc Busca la \OT{}\\
			\hline
			& \inc Muestra los datos de la \OT{}\\
			\hline


			\inc El Técnico le solicita al sistema buscar la Tarea que desea comenzar, ingresando el número de Tarea& \\
			\hline
			& \inc Busca la Tarea\\
			\hline
			& \inc Muestra los datos de la Tarea\\
			\hline
            \inc El Técnico le solicita al sistema registrar el comienzo de la Tarea.&\\
			\hline


            & \inc Marca la Tarea como comenzada, registrando la fecha y hora actual.\\
			\hline
			& \inc Muestra los datos de la Tarea comenzada.\\
			\hline
			\inc Fin CU. & \\
        \hline
		\multicolumn{2}{|c|}{\textbf{Cursos alternos}}\\
		\hline
        \multicolumn{2}{|p{16cm}|}{\textbf{4a. }El sistema informa que la \OT{} no existe. Fin CU.}\\
		\hline
        \multicolumn{2}{|p{16cm}|}{\textbf{4b. }El sistema informa que la \OT{} no posee Tareas pendientes de realización. Fin CU.}\\
		\hline
        \multicolumn{2}{|p{16cm}|}{\textbf{7a. }El sistema informa que la Tarea no existe. Fin CU.}\\
		\hline	
        \multicolumn{2}{|p{16cm}|}{\textbf{7b. }El sistema informa que el Técnico no es el asignado para la Tarea. Fin CU.}\\
		\hline	
        \multicolumn{2}{|p{16cm}|}{\textbf{7c. }El sistema informa que el Técnico no está capacitado para llevar a cabo la Tarea. Fin CU.}\\
		\hline	
        \multicolumn{2}{|p{16cm}|}{\textbf{7d. }El sistema informa que no se cuenta con el stock necesario de productos a utilizar en la Tarea. Fin CU.}\\
		\hline	
		\multicolumn{2}{|p{16cm}|}{\textbf{10. }El Sistema informa que la Tarea ya ha sido comenzada. Fin CU}\\
		\hline	
	\end{longtable}

    \resetinc{}
    \raya{}
	\begin{longtable}{ |p{8cm}|p{8cm}| }
		\hline
        \multicolumn{2}{|p{16cm}|}{\textbf{Caso de uso}: \textbf{54}. Finalizar Tarea}\\
		\multicolumn{2}{|p{16cm}|}{\textbf{Tipo}: Primario Esencial}\\
		\multicolumn{2}{|p{16cm}|}{\textbf{Actores}: Técnico}\\
		\multicolumn{2}{|p{16cm}|}{\textbf{Descripción}: El Técnico asignado a una Tarea decide finalizarla, agrega una observación y luego el sistema la marca como finalizada.}\\
		\multicolumn{2}{|p{16cm}|}{\textbf{Precondiciones}: -}\\
		\multicolumn{2}{|p{16cm}|}{\textbf{Postcondiciones}: Tarea finalizada}\\
		\hline
		\multicolumn{2}{|c|}{\textbf{Curso normal de los eventos}}\\
		\hline
		\textbf{Acción de los actores} & \textbf{Respuesta del sistema}\\
		\hline
            \inc Este CU comienza cuando un Técnico decide finalizar una Tarea& \\
            \hline
            \inc El Técnico le solicita al sistema buscar la \OT{} ingresando su número.& \\
            \hline
            & \inc Busca la \OT{}\\
            \hline
            & \inc Muestra los datos de la \OT{}\\
            \hline


            \inc El Técnico le solicita al sistema buscar la Tarea que desea finalizar, ingresando el número de Tarea& \\
            \hline
            & \inc Busca la Tarea\\
            \hline
            & \inc Muestra los datos de la Tarea\\
            \hline
            \inc El Técnico le solicita al Sistema buscar un repuesto asociado a la Tarea para marcarlo como usado, ingresando su código y cantidad& \\
            \hline


            & \inc Busca el repuesto.\\
            \hline
            & \inc Disminuye el stock del repuesto según la cantidad ingresada e incrementa el stock del producto (cancela el remanente reservado) en caso de ser necesario.\\
            \hline
            & \inc Muestra los datos de la Tarea\\
            \hline
            \inc Repetir pasos 8 al 11 mientras el Técnico desee registrar más repuestos utilizados en la Tarea& \\
            \hline
            
            % Mas o menos por acá hay que ver qué hacemos con los repuestos asignados a la OT
            \inc El Técnico le solicita al sistema marcar la Tarea como finalizada, agregando además una observación sobre el desarrollo de dicha tarea.&\\
            \hline
            & \inc Marca la tarea como finalizada, registrando su fecha y hora de finalización, y agregando la observación realizada.\\
            \hline
            & \inc Muestra las tareas pendientes de la \OT{}.\\
            \hline
            \inc Fin CU. & \\
        \hline
		\multicolumn{2}{|c|}{\textbf{Cursos alternos}}\\
		\hline
        \multicolumn{2}{|p{16cm}|}{\textbf{4a. }El Sistema informa que le \OT{} no existe. Fin CU.}\\
		\hline
        \multicolumn{2}{|p{16cm}|}{\textbf{4b. }El Sistema informa que la \OT{} no posee Tareas por finalizar. Fin CU.}\\
		\hline
		\multicolumn{2}{|p{16cm}|}{\textbf{7a. }El Sistema informa que la Tarea no existe. Fin CU.}\\
		\hline	
		\multicolumn{2}{|p{16cm}|}{\textbf{7b. }El Sistema informa que el Técnico no es el Técnico encargado de la Tarea. Fin CU.}\\
		\hline	
        \multicolumn{2}{|p{16cm}|}{\textbf{8a. }La Tarea no incluye repuestos. Ir al paso 13}\\
		\hline	
        \multicolumn{2}{|p{16cm}|}{\textbf{8b. }El Técnico no utilizó ninguno de los repuestos reservados. El sistema incrementa el stock de los productos (cancela las respectivas reservas). Ir al paso 13}\\
		\hline	
		\multicolumn{2}{|p{16cm}|}{\textbf{10a. }El Sistema informa que el repuesto no existe. Ir al paso 12}\\
		\hline	
		\multicolumn{2}{|p{16cm}|}{\textbf{11a. }El Sistema informa que el repuesto no está asignado a la Tarea. Ir al paso 12}\\
		\hline	
		\multicolumn{2}{|p{16cm}|}{\textbf{11b. }El Sistema informa que el respuesto ya fue marcado como utilizado. Ir al paso 12}\\
		\hline	
        \multicolumn{2}{|p{16cm}|}{\textbf{11c. }El Sistema informa que la cantidad de repuesto ingresada excede la cantidad reservada. Ir al paso 12}\\
		\hline	
        \multicolumn{2}{|p{16cm}|}{\textbf{15. }La tarea cerrada fue la última por cerrar de la \OT{}. El Sistema marca comom cerrada la \OT{}. Fin CU.}\\
		\hline	
	\end{longtable}

\resetinc{}
\raya{}

	\begin{longtable}{ |p{8cm}|p{8cm}| }
		\hline
        \multicolumn{2}{|p{16cm}|}{\textbf{Caso de uso}: \textbf{63}. Generar factura}\\
		\multicolumn{2}{|p{16cm}|}{\textbf{Tipo}: Primario Esencial}\\
		\multicolumn{2}{|p{16cm}|}{\textbf{Actores}: Operario Contable}\\
        \multicolumn{2}{|p{16cm}|}{\textbf{Descripción}: El Operario Contable le solicita al sistema generar la factura correspondiente para una \OT{}. El sistema crea la factura.}\\
		\multicolumn{2}{|p{16cm}|}{\textbf{Precondiciones}: -}\\
        \multicolumn{2}{|p{16cm}|}{\textbf{Postcondiciones}: Factura creada}\\
		\hline
		\multicolumn{2}{|c|}{\textbf{Curso normal de los eventos}}\\
		\hline
		\textbf{Acción de los actores} & \textbf{Respuesta del sistema}\\
		\hline
            \inc Este CU comienza cuando el Operario Contable desea facturar una \OT{}& \\
			\hline
            \inc El Operario Contable ingresa el número de \OT{} y le solicita al sistema que la facture&\\
			\hline
            & \inc Busca la \OT{}\\
			\hline
            & \inc Crea una nueva factura y la vincula con la \OT{}\\
			\hline


            & \inc Toma un detalle de la \OT{} y busca su detalle del presupuesto correspondiente.  \\
			\hline
            & \inc Toma la tarifa de la tarea correspondiente al detalle del presupuesto y el precio del repuesto asociado y su cantidad utilizada \\
			\hline
            & \inc Calcula el precio final del detalle de la \OT{} multiplicando el precio del repuesto asociado por la cantidad utilizada y lo sumándolo al precio de la tarifa de la tarea.\\
			\hline
            & \inc Crea un nuevo detalle de factura y lo vincula con el detalle de la \OT{} y con la factura\\
			\hline


            & \inc Incorpora al detalle de factura el precio final del detalle de la \OT{}\\
			\hline
            & \inc Repetir pasos 5 al 9 hasta que no haya más detalles de la \OT{} por facturar.\\
			\hline
            & \inc Imprime la factura.\\
			\hline
			\inc Fin CU. & \\
		\hline
		\multicolumn{2}{|c|}{\textbf{Cursos alternos}}\\
		\hline
        \multicolumn{2}{|p{16cm}|}{\textbf{4a. }El sistema informa que la \OT{} ingresada no existe. Fin CU.}\\
		\hline
        \multicolumn{2}{|p{16cm}|}{\textbf{4b. }El sistema informa que la \OT{} ingresada ya fue facturada. Fin CU.}\\
		\hline
        \multicolumn{2}{|p{16cm}|}{\textbf{4c. }El sistema informa que la \OT{} ingresada todavía tiene tareas por realizar. Fin CU.}\\
		\hline
        \multicolumn{2}{|p{16cm}|}{\textbf{6a. }El detalle de la \OT{} fue cancelado. Ir al paso 10.}\\
		\hline
        \multicolumn{2}{|p{16cm}|}{\textbf{6b. }El detalle de la \OT{} ya fue facturado. Ir al paso 10.}\\
		\hline
        \multicolumn{2}{|p{16cm}|}{\textbf{7. }El detalle de la \OT{} no incluye repuestos. El precio final del detalle es el valor de la tarifa de la tarea. Ir al paso 8.}\\
		\hline
	\end{longtable}

\resetinc{}
\raya{}
    
	\begin{longtable}{ |p{8cm}|p{8cm}| }
		\hline
        \multicolumn{2}{|p{16cm}|}{\textbf{Caso de uso}: \textbf{64}. Registrar Pago de Factura}\\
		\multicolumn{2}{|p{16cm}|}{\textbf{Tipo}: Primario}\\
		\multicolumn{2}{|p{16cm}|}{\textbf{Actores}: Operario Contable, Cliente}\\
		\multicolumn{2}{|p{16cm}|}{\textbf{Descripción}: El cliente realiza el pago de una factura}\\
		\multicolumn{2}{|p{16cm}|}{\textbf{Precondiciones}: -}\\
		\multicolumn{2}{|p{16cm}|}{\textbf{Postcondiciones}: Una o más facturas pagas}\\
		\hline
		\multicolumn{2}{|c|}{\textbf{Curso normal de los eventos}}\\
		\hline
		\textbf{Acción de los actores} & \textbf{Respuesta del sistema}\\
		\hline
			\inc Este CU comienza cuando el cliente se acerca al local para pagar una factura.& \\
            \hline
			\inc El Operario Contable le pregunta al cliente qué factura desea pagar. El cliente le contesta cuál. & \\
            \hline
			\inc El Operario Contable le solicita al sistema que busque la factura. & \\
            \hline
			& \inc El sistema busca la factura. \\
            \hline
			& \inc El sistema muestra los datos de la factura. \\
            \hline
			\inc El Operario le informa al cliente el importe indicado en el total de la factura. El cliente decide:
                \begin{enumerate}[label=(\alph*)]
                    \item Pagar en efectivo: ir a sección \textit{Pago en Efectivo}.
                    \item Pagar con Nota de Crédito: ir a sección \textit{Pago con Nota de Crédito}.
                    \item Pagar con Transferencia Bancaria: ir a sección \textit{Pago con Transferencia Bancaria}.
                    \item Pagar con Cheque: ir a sección \textit{Pago con Cheque}.
                    \item No paga. Ir al paso 8.
                \end{enumerate}
            & \\
            \hline
            & \inc El sistema muestra el subtotal de lo pagado y el total de lo adeudado.\\
            \hline
            & \inc Repetir pasos 5 al 6 hasta que se registre el cobro total del importe a pagar.\\
            \hline
			\inc El Operario Contable solicita al sistema buscar el Cliente que figura en la factura.& \\
            \hline
            & \inc El sistema busca el cliente.\\
            \hline
            & \inc El sistema muestra los datos del cliente.\\
            \hline
            \inc El Operario Contable registra el pago realizado en la Cuenta Corriente del cliente de la factura.&\\
            \hline
            & \inc El sistema calcula y almacena el saldo actual del cliente que figura en la factura.\\
            \hline
			\inc El Operario Contable le extiende al cliente un comprobante de pago & \\
            \hline
            & \inc Si la factura fue pagada en su totalidad, el sistema la marca como pagada. Si no, el sistema la marca como pagada parcialmente.\\
            \hline
			\inc Fin CU. & \\
		\hline
		\multicolumn{2}{|c|}{\textbf{Cursos alternos}}\\
		\hline
		\multicolumn{2}{|p{16cm}|}{\textbf{4a. } El sistema informa que la factura no existe, fin CU.}\\
		\hline	
        \multicolumn{2}{|p{16cm}|}{\textbf{4b. } El sistema informa que la factura ya fue pagada, fin CU.}\\
		\hline	
        \multicolumn{2}{|p{16cm}|}{\textbf{12. } El sistema le informa al Operario Contable que el cliente dispone de saldo a favor en su cuenta corriente. El Operario Contable genera una Nota de Crédito a nombre del Cliente con dicho saldo. Se la entrega al Cliente. Continuar con el Curso Normal.}\\
		\hline	
	\end{longtable}


    \begin{longtable}{ |p{8cm}|p{8cm}| }
        \hline
        
        \multicolumn{2}{|p{16cm}|}{\textbf{Sección}: Pago en efectivo}\\
        \hline
        \multicolumn{2}{|c|}{\textbf{Curso normal de los eventos}}\\
        \hline
        \textbf{Acción de los actores} & \textbf{Respuesta del sistema}\\
        \hline
            \inc  El cliente da un pago en efectivo& \\
            \hline
            \inc  El Operario Contable le solicita al sistema el registro del cobro del importe a pagar ingresando el monto del efectivo entregado por el cliente& \\
            \hline
            & \inc  El sistema registra el pago en efectivo\\
            \hline
            & \inc  El sistema calcula y muestra el vuelto\\
            \hline
            \inc  El Operario Contable deposita el efectivo recibido, extrae la diferencia y se la entrega al cliente (de ser necesario)&\\
            \hline
            \inc Fin Sección. & \\
        \hline
        \multicolumn{2}{|c|}{\textbf{Cursos alternos}}\\
        \hline
        \multicolumn{2}{|p{16cm}|}{\textbf{1. } El cliente no tiene suficiente efectivo como para pagar el importe. Ir a paso 5 del curso principal de CU.}\\
        \hline
    \end{longtable}
    
    \begin{longtable}{ |p{8cm}|p{8cm}| }
        \hline
        \multicolumn{2}{|p{16cm}|}{\textbf{Sección}: Pago con Nota de Crédito}\\
        \hline
        \multicolumn{2}{|c|}{\textbf{Curso normal de los eventos}}\\
        \hline
        \textbf{Acción de los actores} & \textbf{Respuesta del sistema}\\
        \hline
            \inc  El cliente le entrega al Operario Contable la Nota de Crédito con la que desea abonar.& \\
            \hline
            \inc  El Operario Contable contable busca en el sistema la Nota de Crédito, ingresando su número.& \\
            \hline
            & \inc  El sistema busca la Nota de Crédito.\\
            \hline
            & \inc  El sistema muestra los datos de la Nota de Crédito.\\
            \hline
            \inc  El Operario Contable registra en el sistema el uso de la Nota de Credito como cobro del importe a pagar.&\\
            \hline
            & \inc El Sistema registra el pago con el monto de la Nota de Crédito.\\
            \hline
            & \inc El sistema marca la nota de crédito como ya usada.\\
            \hline
            \inc Fin Sección.&\\
        \hline
        \multicolumn{2}{|c|}{\textbf{Cursos alternos}}\\
        \hline
        \multicolumn{2}{|p{16cm}|}{\textbf{4a. } La Nota de Crédito no existe. El Operario se lo informa al cliente y le da la posibilidad de elegir otra forma de pago, el cliente decide:
            
            \begin{itemize}
                \item Si desea usar otra Nota de Crédito, ir a paso 1.
                \item Si desea utilizar otra forma de pago, ir a paso 5 del curso principal de CU.
            \end{itemize}}\\
        \hline
        \multicolumn{2}{|p{16cm}|}{\textbf{4b. } La Nota de Crédito ya fue usada. El Operario se lo informa al cliente y le da la posibilidad de elegir otra forma de pago, el cliente decide:
        
            \begin{itemize}
                \item Si desea usar otra Nota de Crédito, ir a paso 1.
                \item Si desea utilizar otra forma de pago, ir a paso 5 del curso principal de CU.
            \end{itemize}}\\
        \hline
    \end{longtable}
    \begin{longtable}{ |p{8cm}|p{8cm}| }
        \hline
        \multicolumn{2}{|p{16cm}|}{\textbf{Sección}: Pago con Cheque}\\
        \hline
        \multicolumn{2}{|c|}{\textbf{Curso normal de los eventos}}\\
        \hline
        \textbf{Acción de los actores} & \textbf{Respuesta del sistema}\\
            \hline
            \inc  El Cliente le entrega al Operario Contable el cheque con el que desea abonar & \\
            \hline
            \inc  El Operario Contable solicita al sistema el registro del cobro del importe a pagar con el monto que figura en el cheque &  \\
            \hline
            & \inc El sistema registra el pago con monto del cheque. \\
            \hline
            \inc Fin Sección. & \\
            \hline
        \multicolumn{2}{|c|}{\textbf{Cursos alternos}}\\
        \hline
    \end{longtable}
    \begin{longtable}{ |p{8cm}|p{8cm}| }
        \hline
        \multicolumn{2}{|p{16cm}|}{\textbf{Sección}: Pago con Transferencia Bancaria}\\
        \hline
        \multicolumn{2}{|c|}{\textbf{Curso normal de los eventos}}\\
        \hline
        \textbf{Acción de los actores} & \textbf{Respuesta del sistema}\\
            \hline
            \inc El cliente le entrega el comprobante de la Transferencia Bancaria al Operario Contable& \\
            \hline
            \inc El Operario contable busca en el registro de la organización la transferencia &\\
            \hline
            \inc El Operario Contable confirma la acreditación de la transferencia & \\
            \hline
            \inc El Operario Contable solicita al sistema que registre el cobro del importe a pagar con el monto que figura en la Transferencia& \\
            \hline
            & \inc  El sistema registra el pago con monto de la transferencia\\
            \hline
            \inc Fin Sección &\\
            \hline
        \multicolumn{2}{|c|}{\textbf{Cursos alternos}}\\
        \hline
        \multicolumn{2}{|p{16cm}|}{\textbf{3. } La transferencia no se ha acreditado. El Operario Contable le infoma la situación al Cliente y le da la posibilidad de elegir otra forma de pago. El cliente decide:
        \begin{enumerate}[label=(\alph*)]
            \item Indicar otra transferencia. Ir a paso 1.
            \item Elegir otra forma de pago. Ir al paso 5 del Curso Normal de CU. 
        \end{enumerate}}\\
        \hline
    \end{longtable}

    \resetinc{}
    \raya{}

    \textbf{Caso de uso}: \textbf{66}. Retirar Equipo\\
    \textbf{Tipo}: Primario Esencial\\
    \textbf{Actores}: Técnico\\
    \textbf{Descripción}: Un cliente desea retirar su equipo del taller. El empleado le devuelve su equpo y le solicita al sistema registrar la salida del mismo.\\
    \textbf{Precondiciones}: Cliente existe. Equipo se encuentra en el taller. OT que tenga el equipo cerrada (y pagada si el cliente es particular)\\
    \textbf{Postcondiciones}: Equipo retirado
\end{document}
