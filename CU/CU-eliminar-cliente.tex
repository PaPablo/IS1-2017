\documentclass[12pt]{extarticle}
\usepackage[utf8]{inputenc}
\usepackage{caption}
\usepackage{subcaption}
\usepackage[spanish]{babel}
\usepackage{multicol}
\usepackage{fancyhdr}
\usepackage{longtable}
\usepackage{enumitem}

\usepackage{url}
\usepackage[a4paper]{geometry}
\usepackage{float}
\usepackage{setspace}
\usepackage{color}   %May be necessary if you want to color links
\usepackage{hyperref}
\hypersetup{
    colorlinks=true, %set true if you want colored links
    linktoc=all,     %set to all if you want both sections and subsections linked
    linkcolor=blue,  %choose some color if you want links to stand out
}
\usepackage{graphicx}
\graphicspath{ {images/} }


\begin{document}

	%DEFINIMOS UN CONTADOR
    %\newcounter{step}
    %\newcommand\inc{\stepcounter{step}\textbf{\thestep. }}


    %DEFINIMOS UN CONTADOR
    \newcounter{step}
    \newcommand\inc{\stepcounter{step}\textbf{\thestep. }}
    \newcommand\resetinc{\setcounter{step}{0}}
    
    \newcommand\raya{\noindent\rule{169mm}{0.8mm}\\}

%/******************************************************/
	%/*PLANTILLA PARA CU EXPANDIDO**************************/
	%/*IR ELIMINANDO LOS \inc QUE NO CORRESPONDAN*/
    %/*para copiar la tabla, hay que copiar desde el
    %\begin{longtable} hasta el \setcounter que esta al final*/
	%/*el \inc es el contador que está definido en el informe
	%/******************************************************/
	\begin{longtable}{ |p{8cm}|p{8cm}| }
		\hline
		\multicolumn{2}{|p{16cm}|}{\textbf{Caso de uso}: Eliminar cliente}\\
		\multicolumn{2}{|p{16cm}|}{\textbf{Tipo}: Primario Esencial}\\
		\multicolumn{2}{|p{16cm}|}{\textbf{Actores}: Jefe de Taller}\\
		\multicolumn{2}{|p{16cm}|}{\textbf{Descripción}: El Jefe de Taller da de baja un cliente, de modo que no se abrirán nuevas órdenes de trabajo a su nombre.}\\
		\multicolumn{2}{|p{16cm}|}{\textbf{Precondiciones}: -}\\
		\multicolumn{2}{|p{16cm}|}{\textbf{Postcondiciones}: Cliente dado de baja}\\
		\hline
		\multicolumn{2}{|c|}{\textbf{Curso normal de los eventos}}\\
		\hline
		\textbf{Acción de los actores} & \textbf{Respuesta del sistema}\\
		\hline
			\inc Este CU comienza cuando el Jefe de Taller decide eliminar un cliente.& \\
			\hline
			\inc  El Jefe de Taller le solicita al sistema buscar al cliente, ingresando su DNI/CUIT/CUIL. & \\
			\hline
			& \inc Busca al cliente. \\
			\hline
			& \inc Verifica el estado de la cuenta del cliente. \\
			\hline


            & \inc Verifica que no tenga \OTs{} sin finalizar a su nombre. \\
			\hline
			& \inc Da de baja al cliente. \\
			\hline
			\inc Fin CU. & \\
		\hline
		\multicolumn{2}{|c|}{\textbf{Cursos alternos}}\\
		\hline
		\multicolumn{2}{|p{16cm}|}{\textbf{4. }El sistema informa que el cliente no existe. Fin CU.}\\
		\hline
		\multicolumn{2}{|p{16cm}|}{\textbf{5. }El sistema informa que el cliente posee deuda. Fin CU.}\\
		\hline	
        \multicolumn{2}{|p{16cm}|}{\textbf{6. }El sistema informa que el cliente tiene \OTs{} sin finalizar a su nombre. Fin CU.}\\
		\hline	
	\end{longtable}

    %LO RESETEAMOS A 0
    \setcounter{step}{0}

    \noindent\rule{169mm}{0.8mm}\\
    %LA LINEA SEPARA CU, NO SECCIONES

\end{document}
