\documentclass[12pt]{extarticle}
\usepackage[utf8]{inputenc}
\usepackage[spanish]{babel}
\usepackage{multicol}
\usepackage{longtable}
\usepackage{enumitem}

\begin{document}
	%DEFINIMOS UN CONTADOR
    \newcounter{step}
    \newcommand\inc{\stepcounter{step}\textbf{\thestep. }}
    %LO RESETEAMOS A 0
    \newcommand\resetinc{\setcounter{step}{0}}
    \newcommand\raya{\noindent\rule{169mm}{0.8mm}\\}


%/******************************************************/
	%/*PLANTILLA PARA CU EXPANDIDO**************************/
	%/*IR ELIMINANDO LOS \inc QUE NO CORRESPONDAN*/
    %/*para copiar la tabla, hay que copiar desde el
    %\begin{longtable} hasta el \setcounter que esta al final*/
	%/*el \inc es el contador que está definido en el informe
	%/******************************************************/
	\begin{longtable}{ |p{8cm}|p{8cm}| }
		\hline
		\multicolumn{2}{|p{16cm}|}{\textbf{Caso de uso}: Agregar Tipo de Equipo}\\
		\multicolumn{2}{|p{16cm}|}{\textbf{Tipo}: Primario Escencial}\\
		\multicolumn{2}{|p{16cm}|}{\textbf{Actores}: Jefe de Taller}\\
		\multicolumn{2}{|p{16cm}|}{\textbf{Descripción}: El Jefe de Taller crea un nuevo tipo de equipo}\\
		\multicolumn{2}{|p{16cm}|}{\textbf{Precondiciones}: -}\\
		\multicolumn{2}{|p{16cm}|}{\textbf{Postcondiciones}: Tipo de equipo creado}\\
		\hline
		\multicolumn{2}{|c|}{\textbf{Curso normal de los eventos}}\\
		\hline
		\textbf{Acción de los actores} & \textbf{Respuesta del sistema}\\
		\hline
			\inc Este CU comienza cuando el Jefe de Taller desea agregar un nuevo tipo de equipo & \\
			\hline
            \inc  El Jefe de Taller ingresa un nombre y una descripción del tipo de equipo a dar de alta y le solicita al sistema que lo cree & \\
			\hline
            & \inc El sistema verifica que el tipo de equipo no exista previamente \\
			\hline
			& \inc El sistema crea el tipo de equipo \\
			\hline
            

			\inc El Jefe de Taller ingresa el nombre de un tipo de tarea para vincular con el tipo de equipo & \\
			\hline
			& \inc El sistema busca el tipo de tarea \\
			\hline
            & \inc Vincula el tipo de tarea con el tipo de equipo \\
			\hline
            \inc El Jefe de Taller ingresa el nombre de un tipo de servicio para vincular con el tipo de tarea del tipo de equipo &\\
			\hline
            & \inc El sistema busca el tipo de servicio \\
			\hline
            & \inc Vincula el tipo de servicio con el tipo de tarea del tipo de equipo \\
			\hline
            \inc El Jefe de Taller ingresa un precio de tarifa para el tipo de tarea con el tipo de equipo y el tipo de servicio &\\
			\hline
            & \inc El sistema asigna la tarifa ingresada al tipo de tarea con el tipo de equipo y el tipo de servicio \\
			\hline
            \inc Repetir pasos 5 al 12 hasta que el Jefe de Taller no desee crear más tarifas & \\
			\hline
			\inc Fin CU. & \\
		\hline
		\multicolumn{2}{|c|}{\textbf{Cursos alternos}}\\
		\hline
		\multicolumn{2}{|p{16cm}|}{\textbf{4. }El sistema informa que el tipo de equipo ingresado ya existe. Fin CU.}\\
		\hline
        \multicolumn{2}{|p{16cm}|}{\textbf{5. }El Jefe de Taller no desea crear tarifas. Fin CU}\\
		\hline
		\multicolumn{2}{|p{16cm}|}{\textbf{7a. }El sistema informa que el tipo de tarea ingresado no existe. Ir al paso 13}\\
		\hline	
        \multicolumn{2}{|p{16cm}|}{\textbf{7a. }El sistema informa que el tipo de tarea no está vinculado con el tipo de equipo. Ir al paso 13}\\
		\hline	
		\multicolumn{2}{|p{16cm}|}{\textbf{10a. }El sistema informa que el tipo de servicio ingresado no existe. Ir al paso 13}\\
		\hline	
		\multicolumn{2}{|p{16cm}|}{\textbf{10b. }El sistema informa que ya existe una tarifa para ese tipo de tarea de ese tipo de equipo con ese tipo de servicio. Ir al paso 13}\\
		\hline	
	\end{longtable}

    \resetinc{}
    \raya{}
\end{document}
