\documentclass[12pt]{extarticle}
\usepackage[utf8]{inputenc}
\usepackage[spanish]{babel}
\usepackage{multicol}
\usepackage{longtable}
\usepackage{enumitem}

\begin{document}
	%DEFINIMOS UN CONTADOR
    \newcounter{step}
    \newcommand\inc{\stepcounter{step}\textbf{\thestep. }}
    %LO RESETEAMOS A 0
    \newcommand\resetinc{\setcounter{step}{0}}
    \newcommand\raya{\noindent\rule{169mm}{0.8mm}\\}


%/******************************************************/
	%/*PLANTILLA PARA CU EXPANDIDO**************************/
	%/*IR ELIMINANDO LOS \inc QUE NO CORRESPONDAN*/
    %/*para copiar la tabla, hay que copiar desde el
    %\begin{longtable} hasta el \setcounter que esta al final*/
	%/*el \inc es el contador que está definido en el informe
	%/******************************************************/




	\begin{longtable}{ |p{8cm}|p{8cm}| }
		\hline
		\multicolumn{2}{|p{16cm}|}{\textbf{Caso de uso}: Agregar tipo de equipo}\\
		\multicolumn{2}{|p{16cm}|}{\textbf{Tipo}: Primario Esencial}\\
		\multicolumn{2}{|p{16cm}|}{\textbf{Actores}: Jefe de Taller}\\
        \multicolumn{2}{|p{16cm}|}{\textbf{Descripción}: El Jefe de Taller registra un nuevo tipo de equipo, indicando su precio de \textit{RDyP} según el tipo de servicio, también le agrega por lo menos una tarea que se le podrá realizar. El sistema crea el tipo de equipo y las tareas correspondientes, también genera los vínculos con los tipos de servicio}.\\
		\multicolumn{2}{|p{16cm}|}{\textbf{Precondiciones}: -}\\
		\multicolumn{2}{|p{16cm}|}{\textbf{Postcondiciones}: Tipo de equipo creado}\\
		\hline
		\multicolumn{2}{|c|}{\textbf{Curso normal de los eventos}}\\
		\hline
		\textbf{Acción de los actores} & \textbf{Respuesta del sistema}\\
		\hline
			\inc Este CU comienza cuando el Jefe de Taller desea agregar un nuevo tipo de equipo & \\
			\hline
            \inc El Jefe de Taller ingresa un nombre y una descripción del tipo de equipo a dar de alta y le solicita al sistema que lo cree & \\
			\hline
			& \inc Crea el tipo de equipo \\
			\hline
			& \inc Muestra los datos del tipo de equipo creado\\
			\hline


            %\inc El Jefe de Taller ingresa el nombre de un tipo de servicio para indicar que se podrá trabajar con el tipo de equipo en \OTs{} de ese tipo de servicio &\\
            \inc El Jefe de Taller ingresa el nombre de un tipo de servicio y un precio de \textit{RDyP}&\\
			\hline
			& \inc Busca el tipo de servicio \\
			\hline
            & \inc Crea la \textit{RDyP} para el tipo de equipo en ese tipo de servicio\\
			\hline
            & \inc Muestra los datos de la \textit{RDyP}\\
			\hline


            \inc Repetir pasos 5 a 8 hasta que el Jefe de Taller no desee crear más \textit{RDyP} para el tipo de equipo&\\
			\hline
			\inc El Jefe de Taller ingresa el nombre de un tipo de tarea para indicar que el tipo de tarea se podrá realizar para el tipo de equipo & \\
			\hline
			& \inc Crea el tipo de tarea para el tipo de equipo\\
			\hline
			& \inc Muestra los datos del tipo de tarea\\
			\hline


            \inc Repetir pasos 10 a 12 hasta que el Jefe de Taller no desee crear más tipos de tarea para el tipo de equipo&\\
			\hline
			\inc Fin CU. & \\
		\hline
		\multicolumn{2}{|c|}{\textbf{Cursos alternos}}\\
		\hline
		\multicolumn{2}{|p{16cm}|}{\textbf{4. }El sistema informa que el tipo de equipo ingresado ya existe. Fin CU.}\\
		\hline
        \multicolumn{2}{|p{16cm}|}{\textbf{7. }El sistema informa que el tipo de servicio no existe. Ir al paso 9}\\
		\hline	
        \multicolumn{2}{|p{16cm}|}{\textbf{8. }El sistema informa que el tipo de equipo ya tiene un precio de \textit{RDyP} para ese tipo de servicio. Ir al paso 9}\\
		\hline	
		\multicolumn{2}{|p{16cm}|}{\textbf{12. }El sistema informa que el tipo de tarea ya existe. Ir al paso 13}\\
		\hline	
	\end{longtable}

    \resetinc{}
    \raya{}
\end{document}
