\documentclass[12pt]{extarticle}
\usepackage[utf8]{inputenc}
\usepackage[spanish]{babel}
\usepackage{multicol}
\usepackage{longtable}
\usepackage{enumitem}

\begin{document}
	%DEFINIMOS UN CONTADOR
    \newcounter{step}
    \newcommand\inc{\stepcounter{step}\textbf{\thestep. }}
    %LO RESETEAMOS A 0
    \newcommand\resetinc{\setcounter{step}{0}}
    \newcommand\raya{\noindent\rule{169mm}{0.8mm}\\}


%/******************************************************/
	%/*PLANTILLA PARA CU EXPANDIDO**************************/
	%/*IR ELIMINANDO LOS \inc QUE NO CORRESPONDAN*/
    %/*para copiar la tabla, hay que copiar desde el
    %\begin{longtable} hasta el \setcounter que esta al final*/
	%/*el \inc es el contador que está definido en el informe
	%/******************************************************/
	\begin{longtable}{ |p{8cm}|p{8cm}| }
		\hline
		\multicolumn{2}{|p{16cm}|}{\textbf{Caso de uso}: Comenzar Tarea}\\
		\multicolumn{2}{|p{16cm}|}{\textbf{Tipo}: Primario Esencial}\\
		\multicolumn{2}{|p{16cm}|}{\textbf{Actores}: Técnico}\\
		\multicolumn{2}{|p{16cm}|}{\textbf{Descripción}: El Técnico asignado a una Tarea decide comenzarla, marca su comienzo en el sistema.}\\
		\multicolumn{2}{|p{16cm}|}{\textbf{Precondiciones}: -}\\
		\multicolumn{2}{|p{16cm}|}{\textbf{Postcondiciones}: Tarea comenzada}\\
		\hline
		\multicolumn{2}{|c|}{\textbf{Curso normal de los eventos}}\\
		\hline
		\textbf{Acción de los actores} & \textbf{Respuesta del sistema}\\
		\hline

			\inc Este CU comienza cuando un Técnico decide comenzar a trabajar en una Tarea& \\
			\hline
            \inc El Técnico le solicita al sistema buscar la \OT{} ingresando su número.& \\
			\hline
            & \inc Busca la \OT{}\\
			\hline
			& \inc Muestra los datos de la \OT{}\\
			\hlineo


			\inc El Técnico le solicita al sistema buscar la Tarea que desea comenzar, ingresando el número de Tarea& \\
			\hline
			& \inc Busca la Tarea\\
			\hline
			& \inc Muestra los datos de la Tarea\\
			\hline
            \inc El técnico le solicita al sistema registrar el comienzo de la Tarea.&\\
			\hline


            & \inc Marca la tarea como comenzada, registrando la fecha y hora actual.\\
			\hline
			& \inc Muestra los datos de la Tarea comenzada.\\
			\hline
			\inc Fin CU. & \\
        \hline
		\multicolumn{2}{|c|}{\textbf{Cursos alternos}}\\
		\hline
        \multicolumn{2}{|p{16cm}|}{\textbf{4a. }El sistema informa que la \OT{} no existe. Fin CU.}\\
		\hline
        \multicolumn{2}{|p{16cm}|}{\textbf{4b. }El sistema informa que la \OT{} no posee Tareas pendientes de realización. Fin CU.}\\
		\hline
        \multicolumn{2}{|p{16cm}|}{\textbf{7a. }El sistema informa que la Tarea no existe. Fin CU.}\\
		\hline	
        \multicolumn{2}{|p{16cm}|}{\textbf{7c. }El sistema informa que el Técnico no es el asignado para la Tarea. Fin CU.}\\
		\hline	
        \multicolumn{2}{|p{16cm}|}{\textbf{7d. }El sistema informa que el técnico no está capacitado para llevar a cabo la Tarea. Fin CU.}\\
		\hline	
        \multicolumn{2}{|p{16cm}|}{\textbf{7e. }El sistema informa que no se cuenta con el stock necesario del producto a utilizar en la tarea. Fin CU.}\\
		\hline	
		\multicolumn{2}{|p{16cm}|}{\textbf{10. }El Sistema informa que la Tarea ya ha sido comenzada. Fin CU}\\
		\hline	
	\end{longtable}

    \resetinc{}
    \raya{}
\end{document}
