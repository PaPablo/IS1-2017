\documentclass[12pt]{extarticle}
\usepackage[utf8]{inputenc}
\usepackage[spanish]{babel}
\usepackage{multicol}
\usepackage{longtable}
\usepackage{enumitem}

\begin{document}
	%DEFINIMOS UN CONTADOR
    \newcounter{step}
    \newcommand\inc{\stepcounter{step}\textbf{\thestep. }}
    %LO RESETEAMOS A 0
    \newcommand\resetinc{\setcounter{step}{0}}
    \newcommand\raya{\noindent\rule{169mm}{0.8mm}\\}


%/******************************************************/
	%/*PLANTILLA PARA CU EXPANDIDO**************************/
	%/*IR ELIMINANDO LOS \inc QUE NO CORRESPONDAN*/
    %/*para copiar la tabla, hay que copiar desde el
    %\begin{longtable} hasta el \setcounter que esta al final*/
	%/*el \inc es el contador que está definido en el informe
	%/******************************************************/



        %despues de crear la tarifa, vamos a tener que crear las vinculaciones con tipoServicio y tipoEquipo
        %despues vamos a tener que crear tarifas para ese tipo de tarea indicando tipoServicio y tipoEquipo


	\begin{longtable}{ |p{8cm}|p{8cm}| }
		\hline
		\multicolumn{2}{|p{16cm}|}{\textbf{Caso de uso}: Agregar tipo de tarea}\\
		\multicolumn{2}{|p{16cm}|}{\textbf{Tipo}: Primario Esencial}\\
		\multicolumn{2}{|p{16cm}|}{\textbf{Actores}: Jefe de Taller}\\
		\multicolumn{2}{|p{16cm}|}{\textbf{Descripción}: El Jefe de Taller crea un nuevo tipo de tarea, indicando los tipos de equipos para los cuales se podrá realizar, en qué tipo de servicio podrá ser incluído, y creando tarifas con el tipo de tarea}\\
		\multicolumn{2}{|p{16cm}|}{\textbf{Precondiciones}: -}\\
		\multicolumn{2}{|p{16cm}|}{\textbf{Postcondiciones}: Tipo de tarea creado}\\
		\hline
		\multicolumn{2}{|c|}{\textbf{Curso normal de los eventos}}\\
		\hline
		\textbf{Acción de los actores} & \textbf{Respuesta del sistema}\\
		\hline
			\inc Este CU comienza cuando el Jefe de Taller desea agregar un nuevo tipo de tarea & \\
			\hline
            \inc El Jefe de Taller ingresa un nombre y una descripción del tipo de tarea a dar de alta y le solicita al sistema que lo cree & \\
			\hline
			& \inc Crea el tipo de tarea \\
			\hline
			& \inc Muestra los datos del tipo de tarea creado\\
			\hline


			\inc El Jefe de Taller ingresa el nombre de un tipo de equipo para indicar que el tipo de tarea se podrá realizar para ese tipo de equipo & \\
			\hline
			& \inc Busca el tipo de equipo \\
			\hline
            & \inc Vincula el tipo de equipo con el tipo de tarea \\
			\hline
            & \inc Muestra los nuevos datos del tipo de tarea\\
			\hline


            \inc Repetir pasos 5 a 8 hasta que el Jefe de Taller no desee indicar más tipos de equipo para el tipo de tarea&\\
			\hline
			\inc El Jefe de Taller ingresa el nombre de un tipo de servicio para indicar que el tipo de tarea se podrá realizar dentro de ese tipo de servicio & \\
			\hline
			& \inc Busca el tipo de servicio \\
			\hline
            & \inc Vincula el tipo de servicio con el tipo de tarea \\
			\hline


            & \inc Muestra los nuevos datos del tipo de tarea\\
			\hline
            \inc Repetir pasos 9 a 13 hasta que el Jefe de Taller no desee indicar más tipos de servicio para el tipo de tarea&\\
			\hline
            %creamos tarifas PASO 15
			\inc El Jefe de Taller ingresa el nombre de un tipo de equipo para agregar a una tarifa & \\
			\hline
			& \inc Busca el tipo de equipo \\
			\hline


            & \inc Muestra los datos del tipo de equipo\\
			\hline
            \inc El Jefe de Taller ingresa el nombre de un tipo de servicio para agregar a la tarifa&\\
			\hline
            & \inc Busca el tipo de servicio \\
			\hline
            & \inc Muestra los datos del tipo de servicio\\
			\hline


            \inc El Jefe de Taller ingresa un precio para la tarifa &\\
			\hline
            & \inc Crea la tarifa con el tipo de tarea, tipo de equipo, tipo de serivicio y el precio ingresados\\
			\hline
            & \inc Muestra los datos de la tarifa creada\\
			\hline
            \inc Repetir pasos 15 al 23 hasta que el Jefe de Taller no desee crear más tarifas para el tipo de tarea creado& \\
			\hline


			\inc Fin CU. & \\
		\hline
		\multicolumn{2}{|c|}{\textbf{Cursos alternos}}\\
		\hline
		\multicolumn{2}{|p{16cm}|}{\textbf{4. }El sistema informa que el tipo de tarea ingresado ya existe. Fin CU.}\\
		\hline
		\multicolumn{2}{|p{16cm}|}{\textbf{7. }El sistema informa que el tipo de equipo ingresado no existe. Ir al paso 9}\\
		\hline	
		\multicolumn{2}{|p{16cm}|}{\textbf{8. }El sistema informa que el tipo de equipo y el tipo de tarea ya están vinculados. Ir al paso 9}\\
		\hline	
		\multicolumn{2}{|p{16cm}|}{\textbf{12. }El sistema informa que el tipo de servicio ingresado no existe. Ir al paso 14}\\
		\hline	
		\multicolumn{2}{|p{16cm}|}{\textbf{13. }El sistema informa que el tipo de servicio y el tipo de tarea ya están vinculados. Ir al paso 14}\\
		\hline	
		\multicolumn{2}{|p{16cm}|}{\textbf{18a. }El sistema informa que el tipo de equipo ingresado no existe. Ir al paso 24}\\
		\hline	
		\multicolumn{2}{|p{16cm}|}{\textbf{18b. }El sistema informa que el tipo de tarea no se puede realizar para el tipo de equipo ingresado. Ir al paso 24}\\
		\hline	
		\multicolumn{2}{|p{16cm}|}{\textbf{20a. }El sistema informa que el tipo de servicio ingresado no existe. Ir al paso 24}\\
		\hline	
		\multicolumn{2}{|p{16cm}|}{\textbf{20b. }El sistema informa que el tipo de tarea no se puede realizar bajo el tipo de servicio ingresado. Ir al paso 24}\\
		\hline	
		\multicolumn{2}{|p{16cm}|}{\textbf{23. }El sistema informa que ya existe una tarifa con el tipo de tarea, tipo de equipo y tipo de servicio ingresados. Ir al paso 24}\\
		\hline	
	\end{longtable}

    \resetinc{}
    \raya{}
\end{document}
