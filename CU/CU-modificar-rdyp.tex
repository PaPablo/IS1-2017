
	%DEFINIMOS UN CONTADOR
    %\newcounter{step}
    %\newcommand\inc{\stepcounter{step}\textbf{\thestep. }}


%/******************************************************/
	%/*PLANTILLA PARA CU EXPANDIDO**************************/
	%/*IR ELIMINANDO LOS \inc QUE NO CORRESPONDAN*/
    %/*para copiar la tabla, hay que copiar desde el
    %\begin{longtable} hasta el \setcounter que esta al final*/
	%/*el \inc es el contador que está definido en el informe
	%/******************************************************/
	\begin{longtable}{ |p{8cm}|p{8cm}| }
		\hline
        \multicolumn{2}{|p{16cm}|}{\textbf{Caso de uso}: Modificar precio de \textit{RDyP}}\\
		\multicolumn{2}{|p{16cm}|}{\textbf{Tipo}: Primario Esencial}\\
		\multicolumn{2}{|p{16cm}|}{\textbf{Actores}: Jefe de Taller}\\
        \multicolumn{2}{|p{16cm}|}{\textbf{Descripción}: El Jefe de Taller modifica el precio de un \textit{RDyP}. El sistema registra el cambio de la \textit{RDyP}}\\
		\multicolumn{2}{|p{16cm}|}{\textbf{Precondiciones}: -}\\
        \multicolumn{2}{|p{16cm}|}{\textbf{Postcondiciones}: precio de \textit{RDyP} modificado}\\
		\hline
		\multicolumn{2}{|c|}{\textbf{Curso normal de los eventos}}\\
		\hline
		\textbf{Acción de los actores} & \textbf{Respuesta del sistema}\\
		\hline
            \inc Este CU comienza cuando el Jefe de Taller se dispone a modificar el precio de un \textit{RDyP}& \\
			\hline
			\inc El Jefe de Taller ingresa el nombre de un rubro y le solicita al sistema que lo busque& \\
			\hline
			& \inc Busca el rubro \\
			\hline
			& \inc Muestra los datos del rubro \\
			\hline


            \inc El Jefe de Taller ingresa el nombre de un tipo de servicio y el nuevo precio de \textit{RDyP} para el rubro en ese tipo de servicio& \\
			\hline
			& \inc Busca el tipo de servicio \\
			\hline
            & \inc Busca la \textit{RDyP} del rubro en el tipo de servicio\\
			\hline
            & \inc Modifica el precio de la \textit{RDyP} \\
			\hline


            & \inc Muestra los datos de la \textit{RDyP} \\
			\hline
            \inc Repetir pasos 5 al 9 hasta que el Jefe de Taller no desee modificar más precios de \textit{RDyP} para el rubro& \\
			\hline
			\inc Fin CU. & \\
		\hline
		\multicolumn{2}{|c|}{\textbf{Cursos alternos}}\\
		\hline
		\multicolumn{2}{|p{16cm}|}{\textbf{4. }El sistema informa que el rubro no existe. Fin CU.}\\
		\hline
		\multicolumn{2}{|p{16cm}|}{\textbf{7. }El sistema informa que el tipo de servicio no existe. Ir al paso 10.}\\
		\hline
        \multicolumn{2}{|p{16cm}|}{\textbf{8a. }El sistema informa que no existe una \texit{RDyP} para el tipo de servicio y el rubro. Ir al paso 10.}\\
		\hline
        \multicolumn{2}{|p{16cm}|}{\textbf{8b. }El sistema informa que hay \OTs{} abiertas con la \textit{RDyP} ingresada, por lo tanto no se puede modificar. Ir al paso 10.}\\
		\hline
	\end{longtable}


    %LO RESETEAMOS A 0
    \setcounter{step}{0}

    \noindent\rule{169mm}{0.8mm}\\
    %LA LINEA SEPARA CU, NO SECCIONES
